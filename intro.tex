%-------------------------------------------------------
% THE TITLEPAGE
%-------------------------------------------------------

{\1% % this is the name of the PDF file for the background
\begin{frame}[plain,noframenumbering] % the plain option removes the header from the title page, noframenumbering removes the numbering of this frame only
  \titlepage % call the title page information from above
\end{frame}}

%-------------------------------------------------------
% INTRODUCTION: PRESENTATION PLAN
%-------------------------------------------------------

\begin{frame}{1 Introduction}{\visible<2->{2 Problem} \visible<3->{- 3 Surfing} \visible<4->{ - 4 Performance} \visible<5->{- 5 Relevance} \visible<6->{ - 6 Perspectives}}
	\centering
	\vspace{15pt}
	\textbf{\Large Outline}

	\vspace{15pt}

	\large
	\begin{enumerate}
		\setlength\itemsep{10pt}
		\item<1-> \textbf{Introduction} to the world of plankton
		\item<2-> \alt<7->{\textcolor{gray}{Formulation of the \textbf{navigation problem}}}{Formulation of the \textbf{navigation problem}}
		\item<3-> \alt<7->{\textcolor{gray}{\textbf{Solution}: The \textbf{surfing} strategy}}{Solution: The \textbf{surfing} strategy}
		\item<4-> \alt<7->{\textcolor{gray}{Evaluation of surfing \textbf{performance}}}{Evaluation of surfing \textbf{performance}}
		\item<5-> \alt<7->{\textcolor{gray}{Biophysical \textbf{relevance} of the strategy}}{Biophysical \textbf{relevance} of the strategy}
		\item<6-> \alt<7->{\textcolor{gray}{Summary and \textbf{perspectives}}}{Summary and \textbf{perspectives}}
	\end{enumerate}

\end{frame}

%-------------------------------------------------------
% INTRODUCTION: WHAT ARE PLANKTON?
%-------------------------------------------------------

\begin{frame}{1 Introduction}{2 Problem - 3 Surf - 4 Performance - 5 Relevance - 6 Perspectives}
	\centering
	\vspace{4pt}
	\textbf{\Large What are plankton?}

	\pause
	\begin{tikzpicture}
		\node[inner sep=0pt] (copepod) at (0,2.8) {
			\parbox{1.0\textwidth}{
				\begin{figure}
					\centering
					\def\svgwidth{1.0\textwidth}
					\input{parts/intro/schemes/plankton_definition.pdf_tex}
					\captionsetup{skip=5pt}
					\caption{
						Plankters are defined as being \textbf{carried} by (ocean) \textbf{currents}.
					}
				\end{figure}
			}
		};
		\visible<3->{
			\node[inner sep=0pt] (copepod) at (0,0) {
				\parbox{1.0\textwidth}{
					\begin{figure}
						\centering
						\def\svgwidth{1.0\textwidth}
						\input{parts/intro/schemes/plankton_size.pdf_tex}
						\captionsetup{skip=20pt}
						\caption{
							Illustration a \textbf{plankton size range}.
						}
					\end{figure}
				}
			};
			\draw[-stealth, ColorSurf, very thick] (-5.5,-0.6) -- (5.5,-0.6);
		}
		\visible<5->{
			\node[inner sep=0pt] (copepod) at (0,0.6) {
				\def\svgwidth{1.0\textwidth}
				\input{parts/intro/schemes/plankton_size_shadow.pdf_tex}
			};
			\draw[-stealth, ColorSurf, very thick] (-5.5,-0.6) -- (5.5,-0.6);
		}
	\end{tikzpicture}

	\vspace{-10pt}
		
	\large
	\begin{itemize}
		\item <4-> Many plankters are \textbf{actually able to swim} !
	\end{itemize}
\end{frame}

% %-------------------------------------------------------
% % INTRODUCTION: PLAKNTON SENSING
% %-------------------------------------------------------
%
% \begin{frame}{1 Introduction}{1.2 Plankters can display complex behaviors!}
	% \centering
	% \vspace{2pt}
	% \textbf{\Large Plankter can display complex behaviors!}
%
	% \vspace{2pt}
	% \begin{tikzpicture}
		% \visible<2->{
			% \node[inner sep=0pt] (copepod) at (0,0) {
				% \parbox{0.5\textwidth}{
					% \begin{figure}
						% \centering
						% \def\svgwidth{0.35\textwidth}
						% \input{parts/intro/schemes/copepod_measure_strain.pdf_tex}
						% \captionsetup{skip=5pt, width=0.4\textwidth}
						% \caption{
							% Simplified illustration of the \textbf{bioligical pump}.
						% }
					% \end{figure}
				% }
			% };
		% }
		% \visible<3->{
			% \node[inner sep=0pt] (copepod) at (0.5\textwidth,0) {
				% \parbox{0.5\textwidth}{
					% \begin{figure}
						% \centering
						% \def\svgwidth{0.35\textwidth}
						% \input{parts/intro/schemes/larval_measure_vorticity.pdf_tex}
						% \captionsetup{skip=5pt, width=0.4\textwidth}
						% \caption{
							% Simplified illustration of the \textbf{bioligical pump}.
						% }
					% \end{figure}
				% }
			% };
		% }
	% \end{tikzpicture}
%
	% %\vskip0pt plus 1filll
%
	% \large
	% \begin{itemize}
		% \item measure the \textbf{gradients} $\Gradients ( \ParticlePosition, t )$, not the velocity $\FlowVelocity (\vec{x}, t)$
	% \end{itemize}
	% %\vspace{15pt}
% \end{frame}

%-------------------------------------------------------
% INTRODUCTION: WHY ARE PLANKTON IMPORTANT?
%-------------------------------------------------------

\begin{frame}{1 Introduction}{2 Problem - 3 Surf - 4 Performance - 5 Relevance - 6 Perspectives}
	\centering
	\vspace{10pt}
	\textbf{\Large Why are plankton important?}

	\pause
	\vspace{10pt}
	\begin{figure}
		\begin{tikzpicture}
			\visible<2>{
				\node[inner sep=0pt] (copepod) at (0,0) {
					\def\svgwidth{0.4\textwidth}
					\input{parts/intro/schemes/plankton_importance_0.pdf_tex}
				};
			}
			\visible<3>{
				\node[inner sep=0pt] (copepod) at (0,0) {
					\def\svgwidth{0.4\textwidth}
					\input{parts/intro/schemes/plankton_importance_1.pdf_tex}
				};
			}
			% \visible<4>{
				% \node[inner sep=0pt] (copepod) at (0,0) {
					% \def\svgwidth{0.4\textwidth}
					% \input{parts/intro/schemes/plankton_importance_2.pdf_tex}
				% };
			% }
			\visible<4>{
				\node[inner sep=0pt] (copepod) at (0,0) {
					\def\svgwidth{0.4\textwidth}
					\input{parts/intro/schemes/plankton_importance_3.pdf_tex}
				};
			}
			% \visible<7>{
				% \node[inner sep=0pt] (copepod) at (0,0) {
					% \def\svgwidth{0.4\textwidth}
					% \input{parts/intro/schemes/plankton_importance_4.pdf_tex}
				% };
			% }
			\node[inner sep=0pt] (copepod) at (6,0) {
				\parbox{0.6\textwidth}{
					\large
					\begin{itemize}
						\setlength\itemsep{20pt}
						\large
						\item<2-> \textbf{Marine ecology:} \textbf{basis} of the \textbf{marine food web}
						\item<4-> \textbf{Climate:} carbon trapping
					\end{itemize}
				}
			};
		\end{tikzpicture}
		\captionsetup{skip=7pt, margin={0.0\textwidth, 0.0\textwidth}}
		\caption{
			Illustration of the importance of plankton dynamics. \visible<4->{Ducklow et al. (2001).}
		}
	\end{figure}

	\vskip0pt plus 1filll

	\large
	\vspace{15pt}
\end{frame}

%-------------------------------------------------------
% INTRODUCTION: WHAT ARE PLANKTON?: NET PLANKTON
%-------------------------------------------------------

\begin{frame}{1 Introduction}{2 Problem - 3 Surf - 4 Performance - 5 Relevance - 6 Perspectives}
	\vspace{6pt}
	\centering
	\leavevmode\hidewidth\begin{tikzpicture}
		\node[inner sep=0pt] (copepod) at (0,0) {
		\parbox{1.0\textwidth}{
			\begin{figure}
				\centering
				\def\svgwidth{0.8\textwidth}
				\input{parts/intro/images/marine_plankton.pdf_tex}
				%\captionsetup{skip=5pt, width=0.4\textwidth}
				\caption{
					Marine plankters. Adapted from Nadeau et al. \ccbysa ~ v4.0.
				}
			\end{figure}
		}};
		\visible<4->{
			\draw[-stealth, ColorSurf, very thick] (-4.7,1.5) -- (-4,1.5) node[pos=0, anchor=east]{\textcolor{ColorSurf}{\small Copepods}};
			\draw[-stealth, ColorSurf, very thick] (-4.7,1.5) -- (-3.5,0.4);
			\draw[-stealth, ColorSurf, very thick] (-4.7,1.5) -- (-2.6,2.3);
		}
		\visible<3->{
			\draw[-stealth, ColorSurf, very thick] (4.5,-1.6) -- (4.0,-1.6) node[pos=0, anchor=west]{\textcolor{ColorSurf}{\small Crab larva}};
		}
		\visible<2->{
			\draw[-stealth, ColorSurf, very thick] (4.7,0.85) -- (4.1,0.85) node[pos=0, anchor=west]{\textcolor{ColorSurf}{\small Eggs}};
			\draw[-stealth, ColorSurf, very thick] (4.7,0.85) -- (4.05,0.25);
			\draw[-stealth, ColorSurf, very thick] (4.7,0.85) -- (3.4,1.2);
		}
		\node[opacity=0.0] (copepod) at (5.0,0) {phantom};
	\end{tikzpicture}\hidewidth\null
\end{frame}

%-------------------------------------------------------
% INTRODUCTION: A COPEPOD, AS AN EXAMPLE OF PLANKTER
%-------------------------------------------------------

\begin{frame}{1 Introduction}{2 Problem - 3 Surf - 4 Performance - 5 Relevance - 6 Perspectives}
	\centering
	\vspace{4pt}
	\textbf{\Large Copepods: an example of plankton}

	\vspace{0pt}
	\begin{figure}
		\leavevmode\hidewidth\begin{tikzpicture}
			\visible<1>{
				\node[inner sep=0pt] (copepod) at (0,-1) {
					\def\svgwidth{0.4\textwidth}
					\input{parts/intro/schemes/copepod_picture_0.pdf_tex}
				};
			}
			\visible<2-3>{
				\node[inner sep=0pt] (copepod) at (0,-1) {
					\def\svgwidth{0.4\textwidth}
					\input{parts/intro/schemes/copepod_picture_1.pdf_tex}
				};
			}
			\visible<4-5>{
				\node[inner sep=0pt] (copepod) at (0,-1) {
					\def\svgwidth{0.4\textwidth}
					\input{parts/intro/schemes/copepod_picture_2.pdf_tex}
				};
			}
			\only<6->{
				\node[inner sep=0pt] (copepod) at (0,0) {
					% \movie[width=0.320\textwidth, height=0.340\textwidth, autostart, loop]{}{parts/intro/videos/copepod_escape.gif}
					\movie[width=0.480\textwidth, height=0.510\textwidth, autostart, loop]{}{parts/intro/videos/copepod_ambush_attack.mp4}
				};
			}
			\node[inner sep=0pt] (copepod) at (6,0) {
				\parbox{0.6\textwidth}{
					\large
					\begin{itemize}
						\setlength\itemsep{10pt}
						\large
						\item<2-> \textbf{Sensing}
						\begin{itemize}
							\setlength\itemsep{2pt}
							\normalsize
							\item<3->[$\bullet$] light intensity
							\item<5->[$\bullet$] local hydrodynamic signals
						\end{itemize}
						\item<6-> \textbf{Behavior}
						\begin{itemize}
							\setlength\itemsep{2pt}
							\normalsize
							\item<6->[$\bullet$] neurons
							\item<7->[$\bullet$] catch preys and escape predators
						\end{itemize}
					\end{itemize}
				}
			};
			\node[opacity=0.0] (copepod) at (9.0,0) {phantom};
		\end{tikzpicture}\hidewidth\null
		\captionsetup{skip=6pt, margin={-0.0\textwidth, -0.0\textwidth}}
		\caption{
			\alt<6->{\textit{Acartia Tonsa} ambush attack (270 x real time). Adapted from Kiørboe T. et al. (2009).}{\textit{Acartia tonsa}, a calanoid copepod. \copyright Micropolitan.org}
		}
	\end{figure}
\end{frame}

%-------------------------------------------------------
% INTRODUCTION: Vertical migration of plankton
%-------------------------------------------------------

\begin{frame}{1 Introduction}{2 Problem - 3 Surf - 4 Performance - 5 Relevance - 6 Perspectives}
	\centering
	\vspace{5pt}
	\textbf{\Large Diel vertical migration}

	\pause
	\vspace{0pt}
	\begin{figure}
		\leavevmode\hidewidth\begin{tikzpicture}
			\visible<2->{
				\node[inner sep=0pt] (copepod) at (0,0) {
					\def\svgwidth{0.35\textwidth}
					\input{parts/intro/schemes/vertical_migration_0.pdf_tex}
				};
				\visible<3->{
					\node[inner sep=0pt] (copepod) at (0,0) {
						\def\svgwidth{0.35\textwidth}
						\input{parts/intro/schemes/vertical_migration_1.pdf_tex}
					};
				}
				\visible<4->{
					\node[inner sep=0pt] (copepod) at (0,0) {
						\def\svgwidth{0.35\textwidth}
						\input{parts/intro/schemes/vertical_migration_2.pdf_tex}
					};
				}
				\visible<6->{
					\node[inner sep=0pt] (copepod) at (0,0) {
						\def\svgwidth{0.35\textwidth}
						\input{parts/intro/schemes/vertical_migration_4.pdf_tex}
					};
				}
				\visible<5->{
					\node[inner sep=0pt] (copepod) at (0,0) {
						\def\svgwidth{0.35\textwidth}
						\input{parts/intro/schemes/vertical_migration_3.pdf_tex}
					};
				}
				%\node[opacity=0.0] (copepod) at (9,0) {phantom};
			}
		\end{tikzpicture}\hidewidth\null
		\captionsetup{skip=5pt, margin={0.0\textwidth, 0.0\textwidth}}
		\caption{
			Illustration of plankton \textbf{vertical migrations}. Bandara et al., (2021).
			}
	\end{figure}

	\vskip0pt plus 1filll

	\large
	\begin{itemize}
		%\setlength\itemsep{10pt}
		\item<7-> \textbf{Distance:} $\sim$100\unit{\meter} $\gg$ 1\unit{\milli\meter} ~~ $\sim$10 marathons a day!
		%\item<8-> \textbf{Observation:} Copepods can react to flow sensing!
		%\item<9-> \textbf{Question:} \textbf{How could \textbf{plankters} use \textbf{hydrodynamic cues}\\ to optimize their \textbf{navigation}?}
	\end{itemize}
	\vspace{15pt}
\end{frame}


\begin{frame}{1 Introduction}{2 Problem - 3 Surf - 4 Performance - 5 Relevance - 6 Perspectives}
	\centering
	\vspace{0pt}
	\begin{multicols}{2}
		\begin{itemize}
			\large
			\setlength\itemsep{4pt}
			\item <1-> diel vertical migration
			\item <2-> react to the local flow
		\end{itemize}
	\end{multicols}

	\vspace{6pt}
	\begin{figure}
		\begin{tikzpicture}
			\visible<3->{
				\node[inner sep=0pt] (copepod) at (0,0) {
					\def\svgwidth{0.45\textwidth}
					\input{parts/intro/schemes/problem.pdf_tex}
				};
			}
		\end{tikzpicture}
		% \captionsetup{skip=5pt, margin={0.0\textwidth, 0.0\textwidth}}
		% \caption{
			% Illustration of problem investigated in this study.
		% }
	\end{figure}

	\vskip0pt plus 1filll

	\pause
	\vspace{-5pt}
	\visible<3->{
		\begin{center}
			\Large
			How could \textbf{plankters} use their \textbf{flow sensing}\\ to optimize their \textbf{navigation}?
		\end{center}
	}
	\vspace{15pt}
\end{frame}

%-------------------------------------------------------
% INTRODUCTION: LITERATURE
%-------------------------------------------------------

\begin{frame}{1 Introduction}{2 Problem - 3 Surf - 4 Performance - 5 Relevance - 6 Perspectives}
	\vspace{0pt}
	\begin{center}
		\Large
		How could \textbf{plankters} use their \textbf{flow sensing}\\ to optimize their \textbf{navigation}?
		\pause
		\vspace{10pt}
		\large
		\begin{itemize}
			%\item<2-> Zermelo navigation problem or bird soaring problems.
			%    \begin{itemize}
			%        \item But at most, planktoners only measure local velocity gradients
			%    \end{itemize}
			\item<2-> Plankton observation: data-driven approach
			\item<3-> Machine learning: reinforcement learning approach
		\end{itemize}
	\end{center}
	
	\vspace{0pt}
	\begin{itemize}
		\item<4-> \textbf{Physics-based} approach: \textbf{approximate analytical solution}.
	\end{itemize}
	
	\vskip0pt plus 1filll
	\visible<2->{
		\scriptsize
		\begin{multicols}{2}
			\begin{itemize}
				\item[] Koehl et al., Mar. Ecol. Prog. Ser. (2007)
				\item[] Michalec et al., Elife (2020)
				\item[] Montgomery et al., J. Mar. Biolog. Assoc. (2019)
			\end{itemize}
		\end{multicols}
	}
	\visible<3->{
		\scriptsize
		\begin{multicols}{2}
			\begin{itemize}
				\item[] Alageshan et al., Phys. Rev. E (2020)
				\item[] Gustavsson et al., Eur Phys J E (2017)
				\item[] Cichos et al., Nat. Mach. Intell. (2020)
				%\item[] Daddi-Moussa-Ider et al., Commun. Phys. (2021)
				\item[] Gunnarson et al., Nature (2021)
				\item[] Biferale et al., Chaos (2019)
				\item[] Reddy et al., PNAS (2016)
			\end{itemize}
		\end{multicols}
	}
	\visible<4->{
		\scriptsize
		\begin{multicols}{2}
			\begin{itemize}
				\item[] Bollt et al., J. Fluid Mech. (2021)
				\item[] Liebchen et al., EPL (2019)
			\end{itemize}
		\end{multicols}
	}
\end{frame}
