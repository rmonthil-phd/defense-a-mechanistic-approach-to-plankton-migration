%-------------------------------------------------------
% FORMULATION: PRESENTATION PLAN
%-------------------------------------------------------

\begin{frame}{4 Evaluation of surfing \textbf{performance}}{1 Introduction - 2 Problem - 3 Surfing - 5 Relevance - 6 Perspectives}
	\centering
	\vspace{15pt}
	\textbf{\Large Outline}

	\vspace{15pt}

	\large
	\begin{enumerate}
		\setlength\itemsep{10pt}
		\item \textcolor{gray}{\textbf{Introduction} to the world of plankton}
		\item \textcolor{gray}{Formulation of the \textbf{navigation problem}}
		\item \textcolor{gray}{Solution: The \textbf{surfing} strategy}
		\item \textcolor{white}{Evaluation of surfing \textbf{performance}}
		\item \textcolor{gray}{Biophysical \textbf{relevance} of the strategy}
		\item \textcolor{gray}{Summary and \textbf{perspectives}}
	\end{enumerate}

\end{frame}

%-------------------------------------------------------
% SURFING STRATEGY: Taylor-Green Vortices: Illustration
%-------------------------------------------------------

\begin{frame}{4 Evaluation of surfing \textbf{performance}}{1 Introduction - 2 Problem - 3 Surfing - 5 Relevance - 6 Perspectives}
	\centering
	\vspace{0pt}
	\begin{figure}
		\begin{tikzpicture}
			\node<1>[inner sep=0pt] (copepod) at (0,0) {
				\tikzexternalenable
\begin{tikzpicture}[
		arrow/.style={
			insert path={
				coordinate[pos=#1,sloped]  (aux-1)
				coordinate[pos=#1+\pgfkeysvalueof{/tikz/ga/length},sloped] (aux-2)
				(aux-1) edge[/tikz/ga/arrow] 
				(aux-2) %node[] {\scriptsize #1}
			}
		},
		marrow/.style={
			insert path={
				coordinate[pos=#1,sloped]  (aux-1)
				coordinate[pos=#1-\pgfkeysvalueof{/tikz/ga/length},sloped] (aux-2)
				(aux-1) edge[/tikz/ga/arrow] 
				(aux-2) %node[] {\scriptsize #1}
			}
		},
		ga/.cd,
		length/.initial=0.0001,
		arrow/.style={-stealth,white!50!black,solid},
		marrow/.style={-stealth,white!50!black,solid},
		]
	% plot
	\begin{axis}[
		axis line style={black},
		%width=0.85\linewidth,
		axis equal image,
		view={0}{90},
		% x
		xmin=-3*pi/2,
		xmax=3*pi/2,
		%xlabel=$x$,
		xticklabel=\empty,
		% y
		ymin=-pi/2,
		ymax=5*pi/2,
		%ylabel=$y$,
		yticklabel=\empty,
		% ticks
		tickwidth=0,
		% legend
		legend style={
			draw=none, 
			fill=none, 
			/tikz/every even column/.append style={column sep=4pt}, 
			at={(0.5, 1.025)}, 
			anchor=south,
			opacity=0.0,
		},
   		legend cell align=left,
   		legend columns=-1,
	]
		\addlegendimage{ColorBh,mark=*,mark options={mark indices={3}}}
		\addlegendentry{\NameBh}
		\addlegendimage{ColorSurf,mark=square*,mark options={mark indices={3}}}
		\addlegendentry{\NameSurf}
		% flow
		\addplot3 [
			thick,
			domain=-3*pi/2:3*pi/2,
			domain y=-pi/2:5*pi/2,
			samples=50,
			contour gnuplot={levels={-0.8, -0.6, -0.4, -0.2, 0.2, 0.4, 0.6, 0.8}, labels=false, draw color=white!50!black},
			forget plot,
		] {cos(deg(x)) * cos(deg(y))}
		[arrow/.list={0.46,0.48,0.5,0.52,0.54,0.56,0.58,0.6,0.62,0.64,0.66,0.68,0.7,0.72,0.74,0.76,0.78,0.8,0.82,0.84,0.86,0.88,0.90,0.92,0.94,0.96,0.98,1.0}] [marrow/.list={0.0,0.02,0.04,0.06,0.08,0.1,0.12,0.14,0.16,0.18,0.2,0.22,0.24,0.26,0.28,0.3,0.32,0.34,0.36,0.38,0.4,0.42,0.44}];
		% start position
		%\addplot[mark=*, mark size=1.2mm] coordinates {(0,0)};
		% end axis
	\end{axis}

	% % % plot
	% \begin{axis}[
		% at={(0.5\linewidth, 0)},
		% axis line style={black},
		% width=0.85\linewidth,
		% axis equal image,
		% view={0}{90},
		% % x
		% xmin=-pi/2,
		% xmax=4*pi/2,
		% %xlabel=$x$,
		% xticklabel=\empty,
		% % y
		% ymin=-1.0,
		% ymax=10.0,
		% %ylabel=$y$,
		% yticklabel=\empty,
		% % ticks
		% tickwidth=0,
	% ]
		% % flow
		% \addplot3 [
			% thick,
			% domain=-pi/2:4*pi/2,
			% domain y=-1.0:10.0,
			% samples=50,
			% contour gnuplot={levels={-0.8, -0.6, -0.4, -0.2, 0.2, 0.4, 0.6, 0.8}, labels=false, draw color=white!50!black},
			% forget plot,
		% ] {cos(deg(x)) * cos(deg(y))}
		% [arrow/.list={0.28,0.42,0.44,0.52,0.54,0.56,0.58,0.62,0.64,0.66,0.68,0.7,0.8,0.72,0.76,0.78,0.82,0.84,0.86,0.88,0.90,0.92,0.94,0.96,0.98,1.0}] [marrow/.list={0.0,0.02,0.04,0.06,0.08,0.1,0.12,0.14,0.16,0.18,0.2,0.22,0.24,0.26,0.3,0.32,0.34,0.36,0.38,0.4,0.46,0.48,0.5,0.6,0.74}];
		% % start position
		% \addplot[mark=*, mark size=1.2mm] coordinates {(-pi/4,0)};
		% % end axis
	% \end{axis}
\end{tikzpicture}
\tikzexternaldisable

			};
			\node<2>[inner sep=0pt] (copepod) at (0,0) {
				\movie[width=0.5328\textwidth, height=0.6\textwidth, autostart, poster, loop]{}{parts/surf/plots/main_bh_low.mp4}
			};
			\node<3>[inner sep=0pt] (copepod) at (0,0) {
				\movie[width=0.5328\textwidth, height=0.6\textwidth, autostart, poster, loop]{}{parts/surf/plots/main_low.mp4}
			};
		\end{tikzpicture}
		\captionsetup{skip=5pt, margin={0.0\textwidth, 0.0\textwidth}}
		\caption{
			Illustration of surfing in a 2D Taylor-Green flow.
		}
	\end{figure}
\end{frame}

%-------------------------------------------------------
% TURBULENCE: NUMERICAL SIMULATIONS
%-------------------------------------------------------

\begin{frame}{4 Evaluation of surfing \textbf{performance}}{1 Introduction - 2 Problem - 3 Surfing - 5 Relevance - 6 Perspectives}
	\centering
	\visible<1->{
		Online open-access simulation: \textbf{Johns Hopkins turbulence database}
	}

	% $Re_{\lambda} \approx 418$, $L / \eta \approx 710$, $T_L / \tau_\eta \approx 47$, $u_{rms} / u_{\eta} \approx 10$, $t_{max} \approx 5 T_L$

	\vspace{2pt}
	\begin{figure}
		\begin{tikzpicture}
			\only<1->{
				\node[inner sep=0pt] (copepod) at (0,0) {
					%% \movie[width=0.720\textwidth, height=1.280\textwidth, autostart, poster, loop]{}{parts/turbulence/videos/jhtdb_vertical_velocity.mp4}
					\movie[width=0.24\textwidth, height=0.43\textwidth, autostart, poster, loop]{}{parts/turbulence/videos/jhtdb_vertical_velocity.mp4}
				};
			}
			\only<2->{
				\node[inner sep=0pt] (copepod) at (5,0) {
					\movie[width=0.24\textwidth, height=0.43\textwidth, autostart, poster, loop]{}{parts/turbulence/videos/jhtdb_vorticity.mp4}
				};
			}
			\node[opacity=0.0] (phantom) at (5.7,0) {phantom};
		\end{tikzpicture}
		\captionsetup{skip=6pt, margin={0.0\textwidth, 0.0\textwidth}}
		\caption{
			\visible<1->{Quasi-stationary \textbf{Homogeneous Isotropic Turbulence} ($Re_{\lambda} \approx 418$). Vertical flow velocity: \textcolor{ColorSurf}{red positive}, \textcolor{ColorBh}{blue negative}.}\only<2->{Vorticity: \textcolor{ColorSurf}{red positive}, \textcolor{ColorBh}{blue negative} and \textcolor{yellow!80!black}{yellow null}.} Rendering by Gudmundur Adalsteinsson.
		}
	\end{figure}

	\vspace{-10pt}
	\begin{itemize}
		\large
		\item<3-> Kolmogorov scales: $\eta$, $\tau_\eta$ and $u_{\eta}$ used to \textbf{scale our results}
	\end{itemize}
\end{frame}

%-------------------------------------------------------
% TURBULENCE: SHELD0N PRINCIPLE
%-------------------------------------------------------

\begin{frame}{4 Evaluation of surfing \textbf{performance}}{1 Introduction - 2 Problem - 3 Surfing - 5 Relevance - 6 Perspectives}
	\centering
	\vspace{10pt}
	{\large Trajectory integration: \textbf{Sheld0n} \\ \textit{https://github.com/C0PEP0D/sheld0n}}
	
	\vspace{10pt}
	\leavevmode\hidewidth\begin{figure}
		\begin{tikzpicture}
			\visible<1->{
				\node[inner sep=0pt] (trajs) at (0,0) {
					\def\svgwidth{0.5\textwidth}
					\input{parts/turbulence/schemes/code_0.pdf_tex}
				};
			}
			\visible<2->{
				\node[inner sep=0pt] (trajs) at (0,0) {
					\def\svgwidth{0.5\textwidth}
					\input{parts/turbulence/schemes/code_1.pdf_tex}
				};
			}
			\visible<3->{
				\node[inner sep=0pt] (trajs) at (0,0) {
					\def\svgwidth{0.5\textwidth}
					\input{parts/turbulence/schemes/code_2.pdf_tex}
				};
			}
			\visible<4->{
				\node[inner sep=0pt] (trajs) at (0,0) {
					\def\svgwidth{0.5\textwidth}
					\input{parts/turbulence/schemes/code_3.pdf_tex}
				};
			}
			\node[inner sep=0pt] (copepod) at (6,0) {
				\parbox{0.5\textwidth}{
					\large
					\begin{itemize}
						\setlength\itemsep{5pt}
						\normalsize
						\item<1-> open-source solver, modern c++(17), 20000+ lines of code
						\item<1-> \textbf{fl0p}(interpolated flow), \textbf{fl0w}(flow abstraction), \textbf{l0ad}(vtk/txt/binary loader), \textbf{m0sh}(mesh utility), \textbf{p0l}(Laplace interpolator), ...
						\item<5-> 4$^{\text{th}}$ order Runge-Kutta
						\item<6-> own local flow database
					\end{itemize}
				}
			};
		\end{tikzpicture}
		\captionsetup{skip=6pt, margin={0.0\textwidth, 0.4\textwidth}}
		\caption{
			General principal of the solver.
		}
	\end{figure}\hidewidth\null
\end{frame}

%-------------------------------------------------------
% TURBULENCE: SHELD0N TRAJECTORIES
%-------------------------------------------------------

\begin{frame}{4 Evaluation of surfing \textbf{performance}}{1 Introduction - 2 Problem - 3 Surfing - 5 Relevance - 6 Perspectives}
	\centering
	\vspace{5pt}
	{\large \textbf{Visualisation of trajectories}}
	
	\centering
	\vspace{-25pt}
	\leavevmode\hidewidth\begin{figure}
		\begin{tikzpicture}
			\node<1>[inner sep=0pt] (trajs) at (0,0) {\movie[width=\textwidth, height=0.5\textwidth, autostart, poster, loop]{}{parts/turbulence/videos/trajectories_0.mp4}};
			\node<2>[inner sep=0pt] (trajs) at (0,0) {\movie[width=\textwidth, height=0.5\textwidth, autostart, poster, loop]{}{parts/turbulence/videos/trajectories_1.mp4}};
			\node<3->[inner sep=0pt] (trajs) at (0,0) {\movie[width=\textwidth, height=0.5\textwidth, autostart, poster, loop]{}{parts/turbulence/videos/trajectories_2.mp4}};
			\draw<1->[-stealth] (-5.5,0) -- (-5.5,1) node[midway, anchor=east]{$\Direction$};
			%\node<2->[opacity=0.0] (copepod) at (9.0,0) {phantom};
			\node<1->[inner sep=0pt, ColorPassive] (passives) at (-3.5,-3) {passive};
			\node<2->[inner sep=0pt, ColorBh] (risers) at (0,-3) {bottom-heavy};
			\node<3->[inner sep=0pt, ColorSurf] (surfers) at (3.5,-3) {surfers ($\TimeHorizon = 2 \KolmogorovTimeScale$)};
		\end{tikzpicture}
		\captionsetup{skip=8pt, margin={-0.2\textwidth, -0.2\textwidth}}
		\caption{
			Visualisation of trajectories in a turbulent flow ($\mathit{Re}_{\lambda} \approx 11$).
		}
	\end{figure}\hidewidth\null
\end{frame}

%-------------------------------------------------------
% TURBULENCE: RESULTS
%-------------------------------------------------------

\begin{frame}{4 Evaluation of surfing \textbf{performance}}{1 Introduction - 2 Problem - 3 Surfing - 5 Relevance - 6 Perspectives}
	\centering
	\vspace{-0pt}
	\begin{figure}
		%\tikzexternalenable
\begin{tikzpicture}
	% gain as a function of the free parameter $\tau$
	\begin{axis} [
		axis on top,
		% size
		width=0.6\textwidth,
		%height=0.45\textwidth,
		% y
		ymin=0,
		ymax=2.5,
		ylabel={$\left\langle \Performance \right\rangle / \SwimmingVelocity$},
		y label style={yshift=-4pt},
		extra y ticks={0.5, 1.5, 2.5},
		% x
		xlabel=$\TimeHorizon / \KolmogorovTimeScale$,
		x label style={yshift=4pt},
		xmin=0,
		xmax=10,
		% layers
		set layers,
		% legend
		legend style={
			draw=none,
			fill=none, 
			/tikz/every even column/.append style={column sep=4pt}, 
			at={(0.1, 1.05)},
			anchor=south west
		},
   		legend cell align=left,
   		legend columns=-1,
	]
		\addlegendimage{empty legend}\addlegendentry{$\SwimmingVelocity =$}
		% %% us 0.5
		% %%% 95 CI
		% \addplot[name path=A, draw=none, forget plot] table [
			% x index=3,
			% y expr={(\thisrowno{0} - \thisrowno{1}) / (\thisrowno{2} * 0.066)}, %u_\eta = 0.066
			% col sep=comma,
			% comment chars=\#,
			% restrict expr to domain={\thisrowno{2}}{0.5:0.5},
			% unbounded coords=discard,
		% ]{data/jhtdb/final_merge.csv};
		% \addplot[name path=B, draw=none, forget plot] table [
			% x index=3,
			% y expr={(\thisrowno{0} + \thisrowno{1}) / (\thisrowno{2} * 0.066)}, %u_\eta = 0.066
			% col sep=comma,
			% comment chars=\#,
			% restrict expr to domain={\thisrowno{2}}{0.5:0.5},
			% unbounded coords=discard,
		% ]{data/jhtdb/final_merge.csv};
		% \addplot[ColorSurf!100!ColorVs, opacity=0.4, forget plot, on layer=axis background, visible on=<4->] fill between[of=A and B];
		% %%% average
		% \addplot
		% [
		% color=ColorSurf!100!ColorVs,
		% opacity=1.0,
		% only marks,%solid
		% mark=pentagon,
		% visible on=<4->,
		% ]
		% table[
			% x index=3,
			% y expr={\thisrowno{0} / (\thisrowno{2} * 0.066)}, %u_\eta = 0.066
			% col sep=comma,
			% comment chars=\#,
			% restrict expr to domain={\thisrowno{2}}{0.5:0.5},
			% unbounded coords=discard,
		% ]{data/jhtdb/final_merge.csv};
		% \addlegendentry{$\KolmogorovVelocityScale/2$}
		%% us 1.0
		%%% 95 CI
		\addplot[name path=A, draw=none, forget plot] table [
			x index=3,
			y expr={(\thisrowno{0} - \thisrowno{1}) / (\thisrowno{2} * 0.066)}, %u_\eta = 0.066
			col sep=comma, 
			comment chars=\#,
			restrict expr to domain={\thisrowno{2}}{1.0:1.0},
			unbounded coords=discard,
		]{parts/turbulence/data/jhtdb/final_merge.csv};
		\addplot[name path=B, draw=none, forget plot] table [
			x index=3, 
			y expr={(\thisrowno{0} + \thisrowno{1}) / (\thisrowno{2} * 0.066)}, %u_\eta = 0.066
			col sep=comma, 
			comment chars=\#,
			restrict expr to domain={\thisrowno{2}}{1.0:1.0},
			unbounded coords=discard,
		]{parts/turbulence/data/jhtdb/final_merge.csv};
		\addplot[ColorSurf!100!ColorVs, opacity=0.4, forget plot, on layer=axis background, visible on=<2->,] fill between[of=A and B];
		%%% average
		\addplot
		[
		color=ColorSurf!100!ColorVs,
		opacity=1.0,
		only marks,%solid
		mark=square*,
		visible on=<2-3>,
		]
		table[
			x index=3, 
			y expr={\thisrowno{0} / (\thisrowno{2} * 0.066)}, %u_\eta = 0.066
			col sep=comma, 
			comment chars=\#,
			restrict expr to domain={\thisrowno{2}}{1.0:1.0},
			unbounded coords=discard,
		]{parts/turbulence/data/jhtdb/final_merge.csv};
		\addlegendentry{$\KolmogorovVelocityScale$}
		%%% max
		\addplot
		[
		color=ColorSurf!100!ColorVs,
		opacity=1.0,
		only marks,%solid
		mark=square*,
		visible on=<4>,
		forget plot,
		]
		table[
			x index=3, 
			y expr={\thisrowno{0} / (\thisrowno{2} * 0.066)}, %u_\eta = 0.066
			col sep=comma, 
			comment chars=\#,
			restrict expr to domain={\thisrowno{2}}{1.0:1.0},
			restrict expr to domain={\thisrowno{3}}{4.0:4.0},
			unbounded coords=discard,
		]{parts/turbulence/data/jhtdb/final_merge.csv};
		% %%% fit
		% \addplot
		% [
		% color=ColorSurf!100!ColorVs,
		% opacity=1.0,
		% solid,
		% forget plot,
		% visible on=<3->,
		% ]
		% table[
			% x index=0,
			% y expr={\thisrowno{1} / (1.0 * 0.066)}, %u_\eta = 0.066
			% col sep=comma,
			% comment chars=\#,
			% unbounded coords=discard,
		% ]{data/jhtdb_low/fits_average_velocity_axis_0__agent.csv};
		% %% us 4.0
		% %%% 95 CI
		% \addplot[name path=A, draw=none, forget plot] table [
			% x index=3,
			% y expr={(\thisrowno{0} - \thisrowno{1}) / (\thisrowno{2} * 0.066)}, %u_\eta = 0.066
			% col sep=comma,
			% comment chars=\#,
			% restrict expr to domain={\thisrowno{2}}{4.0:4.0},
			% unbounded coords=discard,
		% ]{data/jhtdb/final_merge.csv};
		% \addplot[name path=B, draw=none, forget plot] table [
			% x index=3,
			% y expr={(\thisrowno{0} + \thisrowno{1}) / (\thisrowno{2} * 0.066)}, %u_\eta = 0.066
			% col sep=comma,
			% comment chars=\#,
			% restrict expr to domain={\thisrowno{2}}{4.0:4.0},
			% unbounded coords=discard,
		% ]{data/jhtdb/final_merge.csv};
		% \addplot[ColorSurf!00!ColorVs, opacity=0.4, forget plot, on layer=axis background, visible on=<3->] fill between[of=A and B];
		% %%% average
		% \addplot
		% [
		% color=ColorSurf!00!ColorVs,
		% opacity=1.0,
		% only marks,%solid
		% mark=o,
		% visible on=<3->,
		% ]
		% table[
			% x index=3,
			% y expr={\thisrowno{0} / (\thisrowno{2} * 0.066)}, %u_\eta = 0.066
			% col sep=comma,
			% comment chars=\#,
			% restrict expr to domain={\thisrowno{2}}{4.0:4.0},
			% unbounded coords=discard,
		% ]{data/jhtdb/final_merge.csv};
		% \addlegendentry{$4 \KolmogorovVelocityScale$}
		% %%% fit
		% \addplot
		% [
		% color=ColorSurf!50!ColorVs,
		% opacity=1.0,
		% solid,
		% forget plot,
		% visible on=<4->,
		% ]
		% table[
			% x index=0,
			% y expr={\thisrowno{2} / (4.0 * 0.066)}, %u_\eta = 0.066
			% col sep=comma,
			% comment chars=\#,
			% unbounded coords=discard,
		% ]{data/jhtdb_more/fits_average_velocity_axis_0__agent.csv};
		%% us 8.0
		%%% 95 CI
		\addplot[name path=A, draw=none, forget plot] table [
			x index=3,
			y expr={(\thisrowno{0} - \thisrowno{1}) / (\thisrowno{2} * 0.066)}, %u_\eta = 0.066
			col sep=comma,
			comment chars=\#,
			restrict expr to domain={\thisrowno{2}}{8.0:8.0},
			unbounded coords=discard,
		]{parts/turbulence/data/jhtdb/final_merge.csv};
		\addplot[name path=B, draw=none, forget plot] table [
			x index=3,
			y expr={(\thisrowno{0} + \thisrowno{1}) / (\thisrowno{2} * 0.066)}, %u_\eta = 0.066
			col sep=comma,
			comment chars=\#,
			restrict expr to domain={\thisrowno{2}}{8.0:8.0},
			unbounded coords=discard,
		]{parts/turbulence/data/jhtdb/final_merge.csv};
		\addplot[ColorSurf!0!ColorVs, opacity=0.4, forget plot, on layer=axis background, visible on=<3->,] fill between[of=A and B];
		%%% average
		\addplot
		[
		color=ColorSurf!0!ColorVs,
		opacity=1.0,
		only marks,%solid
		mark=o,
		visible on=<3>,
		]
		table[
			x index=3,
			y expr={\thisrowno{0} / (\thisrowno{2} * 0.066)}, %u_\eta = 0.066
			col sep=comma,
			comment chars=\#,
			restrict expr to domain={\thisrowno{2}}{8.0:8.0},
			unbounded coords=discard,
		]{parts/turbulence/data/jhtdb/final_merge.csv};
		\addlegendentry{$8 \KolmogorovVelocityScale$}
		%% max
		\addplot
		[
		color=ColorSurf!0!ColorVs,
		opacity=1.0,
		only marks,%solid
		mark=o,
		visible on=<4>,
		forget plot
		]
		table[
			x index=3,
			y expr={\thisrowno{0} / (\thisrowno{2} * 0.066)}, %u_\eta = 0.066
			col sep=comma,
			comment chars=\#,
			restrict expr to domain={\thisrowno{2}}{8.0:8.0},
			restrict expr to domain={\thisrowno{3}}{3.0:3.0},
			unbounded coords=discard,
		]{parts/turbulence/data/jhtdb/final_merge.csv};
		% %%% fit
		% \addplot
		% [
		% color=ColorSurf!0!ColorVs,
		% opacity=1.0,
		% solid,
		% forget plot,
		% visible on=<6->,
		% ]
		% table[
			% x index=0,
			% y expr={\thisrowno{6} / (8.0 * 0.066)}, %u_\eta = 0.066
			% col sep=comma,
			% comment chars=\#,
			% unbounded coords=discard,
		% ]{data/jhtdb_even_more/fits_average_velocity_axis_0__agent.csv};
		%% y = x
        \addplot
        [
        color=white,
        opacity=1.0,
        solid, 
        on layer=axis background,
        domain=0:10,
        forget plot,
        ]{1};
	\end{axis}
\end{tikzpicture}
%\tikzexternaldisable

		\captionsetup{skip=2pt, margin={0.0\textwidth, -0.0\textwidth}}
		\caption{
			Surfing performance in turbulence ($\mathit{Re}_{\lambda} \approx 418$) as a function of the surfing parameter $\TimeHorizon$. Monthiller et al. (2022).
		}
	\end{figure}

	\vskip0pt plus 1filll

	\vspace{-5pt}
	\visible<2->{
		\begin{itemize}
			\item Surfing can \textbf{double} migration speed in turbulence!
		\end{itemize}
	}
	\vspace{10pt}
\end{frame}

\begin{frame}{4 Evaluation of surfing \textbf{performance}}{1 Introduction - 2 Problem - 3 Surfing - 5 Relevance - 6 Perspectives}
	\centering
	\vspace{-10pt}
	\begin{figure}
		%\tikzexternalenable
\begin{tikzpicture}
    % gain as a function of swimming velocity
    \begin{axis}[
        axis on top,
        % size
        width=0.6\linewidth,
        % y
        ylabel={$\left\langle \Performance \right\rangle / \SwimmingVelocity$},
        ymin=0.0,
        ymax=2.5,
        extra y ticks={2.5},
        y label style={yshift=-4pt},
        % x
        xlabel=$\SwimmingVelocity / \KolmogorovVelocityScale$,
        xmode=log,
        xticklabels={0.1,1,10},
        extra x ticks={0.5, 20},
        extra x tick labels={0.5,20},
        xmin=0.5,
        xmax=20,
        x label style={yshift=4pt},
        % layers
        set layers,
        % legend
        legend style={draw=none, fill=none, /tikz/every even column/.append style={column sep=8pt}},
        legend cell align={left},
        legend pos=south east,
        legend columns=-1,
    ]
        % tss
        \addplot[name path=A, draw=none, forget plot] table [
            x index=2, 
            y expr={(\thisrowno{0} - \thisrowno{1}) / (\thisrowno{2} * 0.066)}, %u_\eta = 0.066
            col sep=comma,
            comment chars=\#,
        ]{parts/turbulence/data/jhtdb/final_max.csv};
        \addplot[name path=B, draw=none, forget plot] table [
            x index=2, 
            y expr={(\thisrowno{0} + \thisrowno{1}) / (\thisrowno{2} * 0.066)}, %u_\eta = 0.066
            col sep=comma, 
            comment chars=\#,
        ]{parts/turbulence/data/jhtdb/final_max.csv};
        \addplot[ColorSurf, opacity=0.4, forget plot, on layer=axis background, visible on=<1->] fill between[of=A and B];
        \addplot
        [
        color=ColorSurf,
        opacity=1.0,
        only marks,%solid, 
        mark=square*,
        visible on=<1->
        ]
        table[
            x index=2,
            y expr={\thisrowno{0} / (\thisrowno{2} * 0.066)}, %u_\eta = 0.066
            col sep=comma, 
            comment chars=\#,
        ]{parts/turbulence/data/jhtdb/final_max.csv};
        %\addlegendentry{\NameSurf}
        % % straight
        % \addplot[name path=A, draw=none, forget plot] table [
            % x index=2,
            % y expr={(\thisrowno{0} - \thisrowno{1}) / (\thisrowno{2} * 0.066)}, %u_\eta = 0.066
            % col sep=comma,
            % comment chars=\#,
            % restrict expr to domain={\thisrowno{3}}{0.0:0.0},
            % unbounded coords=discard,
        % ]{data/jhtdb/final_merge.csv};
        % \addplot[name path=B, draw=none, forget plot] table [
            % x index=2,
            % y expr={(\thisrowno{0} + \thisrowno{1}) / (\thisrowno{2} * 0.066)}, %u_\eta = 0.066
            % col sep=comma,
            % comment chars=\#,
            % restrict expr to domain={\thisrowno{3}}{0.0:0.0},
            % unbounded coords=discard,
        % ]{data/jhtdb/final_merge.csv};
        % \addplot[ColorBh, opacity=0.4, forget plot, on layer=axis background, visible on=<1->] fill between[of=A and B];
        % \addplot
        % [
        % color=ColorBh,
        % opacity=1.0,
        % only marks,%solid,
        % mark=o,
        % visible on=<1->
        % ]
        % table[
            % x index=2,
            % y expr={\thisrowno{0} / (\thisrowno{2} * 0.066)}, %u_\eta = 0.066
            % col sep=comma,
            % comment chars=\#,
            % restrict expr to domain={\thisrowno{3}}{0.0:0.0},
            % unbounded coords=discard,
        % ]{data/jhtdb/final_merge.csv};
        % \addlegendentry{\NameBh}
        %% y = x
        \addplot
        [
        color=white,
        opacity=1.0,
        solid, 
        on layer=axis background,
        domain=0.5:20,
        ]{1};
        %% model
        %\def\modlambda{0.05}
        %\def\modomega{0.2}
        %\addplot
        %[
        %color=black,
        %opacity=1.0,
        %on layer=axis foreground,
        %dashed,
        %]
        %table[
        %    x index=0,
        %    y expr={e^(\modlambda * (0.11 * \thisrowno{1} / 0.0424)) / cos(deg(\modomega * (0.11 * \thisrowno{1} / 0.0424)))}, % \tau_\eta = 0.0424
        %    col sep=comma,
        %    comment chars=\#,
        %]{data/jhtdb/final_time.csv};
    \end{axis}
    \begin{axis}[
	    % size
	    width=0.6\linewidth,
	    % x
	    xmin=0.05,
	    xmax=2,
	    xlabel=$\SwimmingVelocity / u_{\mathrm{rms}}$,
	    xmode=log,
	    x label style={yshift=-4pt},
	    axis x line*=top,
	    xtickten={-2, -1, 0},
	    xticklabels={0.01,0.1,1},
	    extra x ticks={0.5, 2},
	    extra x tick labels={0.5,2},
	    % y
	    axis y line=none,
	    ymin=0,
	    ymax=2.5,
    ]
    \addplot
    [
	    color=white,
	    opacity=1.0,
	    solid, 
	    on layer=axis background,
	    domain=0.05:2,
    ]
    {1};
    \end{axis}
\end{tikzpicture}
%\tikzexternaldisable

		\captionsetup{skip=2pt, margin={0.0\textwidth, -0.0\textwidth}}
		\caption{
			Surfing performance in turbulence ($\mathit{Re}_{\lambda} \approx 418$) as a function of swimming velocity $\SwimmingVelocity / \KolmogorovVelocityScale$. Monthiller et al. (2022).
		}
	\end{figure}

	\vspace{-10pt}
	\visible<2->{
		\begin{itemize}
			\item Migration performance decrease with the ratio $\SwimmingVelocity / \KolmogorovVelocityScale$
		\end{itemize}
	}
\end{frame}

%-------------------------------------------------------
% TURBULENCE: REORIENTATION TIME
%-------------------------------------------------------

\begin{frame}{4 Evaluation of surfing \textbf{performance}}{1 Introduction - 2 Problem - 3 Surfing - 5 Relevance - 6 Perspectives}
	\centering
	\vspace{5pt}
	\textbf{\large Robustness: finite-time reorientation}
	
	\vspace{10pt}
	\begin{tikzpicture}
		\visible<2>{
			\node[inner sep=0pt] (trajs) at (0,0) {
				\def\svgwidth{0.75\textwidth}
				\input{parts/turbulence/schemes/reorientation_time_0.pdf_tex}
			};
		}
		\visible<3>{
			\node[inner sep=0pt] (trajs) at (0,0) {
				\def\svgwidth{0.75\textwidth}
				\input{parts/turbulence/schemes/reorientation_time_1.pdf_tex}
			};
		}
		\visible<4>{
			\node[inner sep=0pt] (trajs) at (0,0) {
				\def\svgwidth{0.75\textwidth}
				\input{parts/turbulence/schemes/reorientation_time_2.pdf_tex}
			};
		}
		\visible<5>{
			\node[inner sep=0pt] (trajs) at (0,0) {
				\def\svgwidth{0.75\textwidth}
				\input{parts/turbulence/schemes/reorientation_time_3.pdf_tex}
			};
		}
		\visible<6>{
			\node[inner sep=0pt] (trajs) at (0,0) {
				\def\svgwidth{0.75\textwidth}
				\input{parts/turbulence/schemes/reorientation_time_4.pdf_tex}
			};
		}
		\visible<7->{
			\node[inner sep=0pt] (trajs) at (0,0) {
				\def\svgwidth{0.75\textwidth}
				\input{parts/turbulence/schemes/reorientation_time_5.pdf_tex}
			};
		}
	\end{tikzpicture}

	\vspace{-0pt}
	\visible<7->{
		\begin{equation*}
			\frac{d \SwimmingDirection}{d t} = \frac{1}{2} \FlowVorticity (\ParticlePosition, t) \times \SwimmingDirection + \frac{1}{2 \ReorientationTime} \left[ \ControlDirection - (\ControlDirection \cdot \SwimmingDirection) \SwimmingDirection \right]
		\end{equation*}

		\vspace{-10pt}
		\scriptsize Pedley et al. (1992)
	}
\end{frame}

\begin{frame}{4 Evaluation of surfing \textbf{performance}}{1 Introduction - 2 Problem - 3 Surfing - 5 Relevance - 6 Perspectives}
	\centering
	\vspace{5pt}
	\centering
	\begin{figure}
		%\tikzexternalenable
\begin{tikzpicture}
    \begin{groupplot}[
            group style={
                group size=1 by 1,
            },
            % size
            width=0.6\textwidth,
            % y
            ylabel={$\left\langle \Performance \right\rangle / \SwimmingVelocity$},
            y label style={yshift=-4pt},
            ymin=0.0,
            ymax=2.5,
            % x
            x label style={yshift=4pt},
            xmin=0.0,
            % layers
            set layers,
            % legend
            legend style={draw=none, fill=none},
            legend cell align=left,
        ]
    \nextgroupplot[
        %width=0.33\linewidth,
        % x
        extra y ticks={0.5, 1.5, 2.5},
        xlabel=$\ReorientationTime / \KolmogorovTimeScale$,
        xmax=4,
        % legend
        legend pos=north east,
        legend style={fill opacity=0.5, text opacity=1},
    ]
        % shade
        \addplot[name path=A, draw=none, forget plot] table [
            x index=4, 
            y expr={(\thisrowno{0} - \thisrowno{1}) / 0.066},
            col sep=comma, 
            comment chars=\#,
        ]{parts/turbulence/data/jhtdb_reorientation_time/max.csv};
        \addplot[name path=B, draw=none, forget plot] table [
            x index=4, 
            y expr={(\thisrowno{0} + \thisrowno{1}) / 0.066},
            col sep=comma,
            comment chars=\#,
        ]{parts/turbulence/data/jhtdb_reorientation_time/max.csv};
        \addplot[ColorSurf, opacity=0.4, forget plot, visible on=<1->] fill between[of=A and B];
        \addplot[name path=A, draw=none, forget plot] table [
            x index=4, 
            y expr={(\thisrowno{0} - \thisrowno{1}) / 0.066},
            col sep=comma, 
            comment chars=\#,
            restrict expr to domain={\thisrowno{3}}{0.0:0.0},
        ]{parts/turbulence/data/jhtdb_reorientation_time/merge.csv};
        \addplot[name path=B, draw=none, forget plot] table [
            x index=4, 
            y expr={(\thisrowno{0} + \thisrowno{1}) / 0.066},
            col sep=comma,
            comment chars=\#,
            restrict expr to domain={\thisrowno{3}}{0.0:0.0},
        ]{parts/turbulence/data/jhtdb_reorientation_time/merge.csv};
        \addplot[ColorBh, opacity=0.4, forget plot, visible on=<1->] fill between[of=A and B];
        % plot
        \addplot
        [
        color=ColorSurf,
        opacity=1.0,
        line width=1pt, 
        only marks,%solid,
        mark=square*,
        visible on=<1->,
        ]
        table[
            x index=4, 
            y expr={\thisrowno{0} / 0.066},
            col sep=comma, 
            comment chars=\#,
            unbounded coords=discard,
        ]{parts/turbulence/data/jhtdb_reorientation_time/max.csv};
        \addlegendentry{\NameSurf}
        %\addplot
        %[
        %color=pink!50!black,
        %opacity=1.0,
        %line width=1pt, 
        %dashed,
        %]
        %table[
        %    x index=4, 
        %    y expr={1.0},
        %    col sep=comma, 
        %    comment chars=\#,
        %    restrict expr to domain={\thisrowno{3}}{0.0:0.0},
        %    unbounded coords=discard,
        %]{parts/turbulence/data/jhtdb_reorientation_time/final_merge.csv};
        %\addlegendentry{\NameBh}
        \addplot
        [
        color=ColorBh,
        opacity=1.0,
        line width=1pt, 
        only marks,%solid,
        mark=o,
        visible on=<1->,
        ]
        table[
            x index=4, 
            y expr={\thisrowno{0} / 0.066},
            col sep=comma, 
            comment chars=\#,
            restrict expr to domain={\thisrowno{3}}{0.0:0.0},
            unbounded coords=discard,
        ]{parts/turbulence/data/jhtdb_reorientation_time/merge.csv};
        \addlegendentry{bottom-heavy}
        \addplot
        [
        color=white,
        opacity=1.0,
        solid, 
        on layer=axis background,
        domain=0.0:4,
        ]{1};
    \end{groupplot}
\end{tikzpicture}
%\tikzexternaldisable

		\captionsetup{skip=2pt, margin={0.0\textwidth, -0.0\textwidth}}
		\caption{
			Migration performance in turbulence ($\mathit{Re}_{\lambda} \approx 418$) as a function of the reorientation time $\ReorientationTime / \KolmogorovTimeScale$. Monthiller et al. (2022).
		}
	\end{figure}

	\vskip0pt plus 1filll

	\vspace{0pt}
	\visible<2->{
		\begin{itemize}
			\large
			\item Surfing is \textbf{robust} to finite reorientation time.
		\end{itemize}
	}
	\vspace{30pt}
\end{frame}
