%-------------------------------------------------------
% CONCLUSION: 
%-------------------------------------------------------

\begin{frame}{6 Summary and \textbf{perspectives}}{1 Introduction - 2 Problem - 3 Surfing - 4 Performance - 5 Relevance}
	\centering
	\vspace{15pt}
	\textbf{\Large Outline}

	\vspace{15pt}

	\large
	\begin{enumerate}
		\setlength\itemsep{10pt}
		\item \textcolor{gray}{\textbf{Introduction} to the world of plankton}
		\item \textcolor{gray}{Formulation of the \textbf{navigation problem}}
		\item \textcolor{gray}{Solution: The \textbf{surfing} strategy}
		\item \textcolor{gray}{Evaluation of surfing \textbf{performance}}
		\item \textcolor{gray}{Biophysical \textbf{relevance} of the strategy}
		\item \textcolor{white}{Summary and \textbf{perspectives}}
	\end{enumerate}
\end{frame}

\begin{frame}{6 Summary and \textbf{perspectives}}{1 Introduction - 2 Problem - 3 Surfing - 4 Performance - 5 Relevance}
	\centering
	\vspace{5pt}
	\textbf{\Large Summary}

	\vspace{10pt}
	\centering
	\begin{itemize}
		\setlength\itemsep{2pt}
		\item<1-> Formalized the \textbf{navigation problem} of plankton \textbf{vertical migration}
		\item<2-> We derived an \textbf{approximate analytical solution}
		\item<3-> We demonstrated its \textbf{performance in turbulence}
		\item<4-> We demonstrated its \textbf{relevance} for actual plankters in their habitat
	\end{itemize}

	\vspace{40pt}
	\begin{tikzpicture}
		\node<1>[inner sep=0pt] (copepod) at (0,0) {
			\def\svgwidth{0.9\textwidth}
			\input{parts/upscaling/summary_0.pdf_tex}
		};
		\node<2>[inner sep=0pt] (copepod) at (0,0) {
			\def\svgwidth{0.9\textwidth}
			\input{parts/upscaling/summary_1.pdf_tex}
		};
		\node<3>[inner sep=0pt] (copepod) at (0,0) {
			\def\svgwidth{0.9\textwidth}
			\input{parts/upscaling/summary_2.pdf_tex}
		};
		\node<4->[inner sep=0pt] (copepod) at (0,0) {
			\def\svgwidth{0.9\textwidth}
			\input{parts/upscaling/summary_3.pdf_tex}
		};
	\end{tikzpicture}
\end{frame}

\begin{frame}{6 Summary and \textbf{perspectives}}{1 Introduction - 2 Problem - 3 Surfing - 4 Performance - 5 Relevance}
	\centering
	\vspace{-10pt}
	\small
	\begin{multicols}{2}
		\begin{itemize}
			%\setlength\itemsep{2pt}
			\item \textcolor{ColorSurf}{Plankton}
				\begin{itemize}
					\scriptsize
					\item[$\bullet$] \textcolor{ColorSurf}{What are plankton}
					\item[$\bullet$] \textcolor{ColorSurf}{Role in marine food web}
					\item<2->[$\bullet$] Threat of climate change for plankton
					\item<2->[$\bullet$] Biological pump
				\end{itemize}
			\item \textcolor{ColorSurf}{Plankton navigation problem}
				\begin{itemize}
					\scriptsize
					\item[$\bullet$] \textcolor{ColorSurf}{Vertical migration}
					\item<2->[$\bullet$] Horizontal dispersion
					\item<2->[$\bullet$] Odor and light tracking
				\end{itemize}
			\item \textcolor{ColorSurf}{Surfing strategy}
				\begin{itemize}
					\scriptsize
					\item[$\bullet$] \textcolor{ColorSurf}{Derivation}
					\item<2->[$\bullet$] Characterisation in linear flows
					\item[$\bullet$] \textcolor{ColorSurf}{Taylor-Green illustration}
					\item<2->[$\bullet$] Surfing in a Poiseuille flow
				\end{itemize}
			\item \textcolor{ColorSurf}{Surfing on turbulence}
				\begin{itemize}
					\scriptsize
					\item[$\bullet$] \textcolor{ColorSurf}{Performance}
					\item<2->[$\bullet$] Flow sampled by surfers
					\item<2->[$\bullet$] Influence of $\mathit{Re}_{\lambda}$
					\item<2->[$\bullet$] Estimation of performance
				\end{itemize}
			\item \textcolor{ColorSurf}{Robustness}
				\begin{itemize}
					\scriptsize
					\item[$\bullet$] \textcolor{ColorSurf}{Reorientation time}
					\item<2->[$\bullet$] Turbulence fluctuations
					\item<2->[$\bullet$] Noise
					\item<2->[$\bullet$] Filtering
					\item<2->[$\bullet$] Partial measure
					\item<2->[$\bullet$] Processing power
				\end{itemize}
			\item<2-> Other navigation approaches
				\begin{itemize}
					\scriptsize
					\item<2->[$\bullet$] Zermelo equation
					\item<2->[$\bullet$] Dijkstra algorithm
					\item<2->[$\bullet$] Reinforcement learning
				\end{itemize}
			\item \textcolor{ColorSurf}{Biophysical relevance}
				\begin{itemize}
					\scriptsize
					\item[$\bullet$] \textcolor{ColorSurf}{Demonstration of relevance}
					\item<2->[$\bullet$] Optimality in marine ecology
				\end{itemize}
			\item<2-> Additional motion dynamics
			\item<2-> Horizontal dispersion problem
			\item<2-> Energey efficient navigation
		\end{itemize}
	\end{multicols}
\end{frame}

\begin{frame}{6 Summary and \textbf{perspectives}}{1 Introduction - 2 Problem - 3 Surfing - 4 Performance - 5 Relevance}
	\centering
	\vspace{4pt}
	\textbf{\Large Perspectives}

	\pause
	\vspace{2pt}
	\centering
	\begin{itemize}
		\setlength\itemsep{10pt}
		\item<1-> \textbf{Generalization}
			\begin{align*}
				\ControlDirectionOpt^* = \frac{\ControlDirectionOptNN^*}{\norm{\ControlDirectionOptNN^*}}, \quad \text{with} \quad \ControlDirectionOptNN^* &= \exp \left[ \TimeHorizon (\Gradients)^\top + \frac{\TimeHorizon^{2}}{2} \frac{d (\Gradients)^\top}{dt} + \cdots \right] \cdot \Direction \\
				 &= \exp \left[ \sum_{k=0}^\infty \frac{\TimeHorizon^{k+1}}{(k+1)!} \frac{d^k (\Gradients)^\top}{dt^k} \right] \cdot \Direction
			\end{align*}
		\item<1-> \st{Approximate} \textbf{Actual optimal solution}
		\item<1-> $\ControlDirectionOpt$ is a first order approximation of $\ControlDirectionOpt^*$
		\item<1-> function of \textbf{local information} contrary to the Zermelo equation or the Dijkstra algorithm $\to$ animal navigation
		\item<1-> \textbf{Positional target:} $\Direction(t)$ \quad $\Direction \to \Direction(\TimeHorizon) \approx \Direction_0 + \TimeHorizon (\partial \Direction/ \partial t)_0 + \cdots$
	\end{itemize}
\end{frame}

\begin{frame}{6 Summary and \textbf{perspectives}}{1 Introduction - 2 Problem - 3 Surfing - 4 Performance - 5 Relevance}
	\centering
	\vspace{5pt}
	\textbf{\Large Perspectives}

	\vspace{15pt}
	\centering
	\leavevmode\hidewidth\begin{tikzpicture}
		\node[inner sep=0pt] (copepod) at (0,0) {
			\def\svgwidth{1.0\textwidth}
			\input{parts/upscaling/perspectives.pdf_tex}
		};
	\end{tikzpicture}\hidewidth\null

	\vspace{0pt}
	\centering
	\begin{multicols}{2}
		\begin{itemize}
			\scriptsize
			\setlength\itemsep{2pt}
			\item \textbf{Sheld0n:} https://github.com/C0PEP0D/sheld0n
			\item \textbf{Thesis:} https://github.com/rmonthil-phd
			\item \textbf{Defense:} https://github.com/rmonthil-phd
			\item \textbf{Video-game:} https://akarius.itch.io/floward
		\end{itemize}
	\end{multicols}
\end{frame}

%-------------------------------------------------------
% MORE: 
%-------------------------------------------------------

\begin{frame}[noframenumbering]{3 Solution: The \textbf{surfing} strategy}{1 Introduction - 2 Problem - 4 Performance - 5 Relevance - 6 Perspectives}
	\centering
	\textbf{\Large Definition of the gradient}

	\Large
	\begin{center}
		\begin{equation*}
			\Gradients = \begin{pmatrix}
				\displaystyle \frac{\partial \FlowVelocityScalar_x}{\partial x} & \displaystyle \frac{\partial \FlowVelocityScalar_x}{\partial y} & \displaystyle \frac{\partial \FlowVelocityScalar_x}{\partial z} \\[20pt]
				\displaystyle \frac{\partial \FlowVelocityScalar_y}{\partial x} & \displaystyle \frac{\partial \FlowVelocityScalar_y}{\partial y} & \displaystyle \frac{\partial \FlowVelocityScalar_y}{\partial z} \\[20pt]
				\displaystyle \frac{\partial \FlowVelocityScalar_z}{\partial x} & \displaystyle \frac{\partial \FlowVelocityScalar_z}{\partial y} & \displaystyle \frac{\partial \FlowVelocityScalar_z}{\partial z}
			\end{pmatrix}
		\end{equation*}
	\end{center}
\end{frame}

%-------------------------------------------------------
% SURFING STRATEGY: INTERPRETATION
%-------------------------------------------------------

\begin{frame}[noframenumbering]{3 Solution: The \textbf{surfing} strategy}{1 Introduction - 2 Problem - 4 Performance - 5 Relevance - 6 Perspectives}
	\centering
	\textbf{\Large Physical interpretation of \textit{surfing}}
	\begin{equation*}
		\ControlDirectionOpt = \frac{\ControlDirectionOptNN}{\norm{\ControlDirectionOptNN}}, \quad \text{with} \quad
		\ControlDirectionOptNN = \Direction + \TimeHorizon \, \vec{\nabla} \FlowVelocityScalar_z + \frac{1}{2} \TimeHorizon^2 \, \vec{\nabla} ( \FlowVelocity \cdot \vec{\nabla} \FlowVelocityScalar_\DirectionScalar ) + \dotsb
	\end{equation*}

	\begin{figure}
		\begin{tikzpicture}
			\node[inner sep=0pt] (copepod) at (0,0) {
				\def\svgwidth{0.3\textwidth}
				\input{parts/surf/schemes/physical_interpretation_3.pdf_tex}
			};
		\end{tikzpicture}
		\captionsetup{skip=10pt}
		\caption{
			Illustration of the surfing strategy in a simple vortex flow.
		}
	\end{figure}

	\vskip0pt plus 1filll

	\large
	\begin{itemize}
		\item<1-> succession of \textbf{``gradient ascents''} controlled by $\TimeHorizon$
	\end{itemize}
	\vspace{15pt}
\end{frame}

% %-------------------------------------------------------
% % SURFING STRATEGY: INTERPRETATION
% %-------------------------------------------------------
% 
% \begin{frame}[noframenumbering]{3 Solution: The \textbf{surfing} strategy}{1 Introduction - 2 Problem - 4 Performance - 5 Relevance - 6 Perspectives}
	% \centering
	% \textbf{\Large Physical interpretation of \textit{surfing}}
	% \only<1>{
		% \begin{equation*}
			% \ControlDirection = \frac{\ControlDirectionOptNN}{\norm{\ControlDirectionOptNN}}, \quad \text{with} \quad \ControlDirectionOptNN = \exp \left[ \TimeHorizon (\Gradients)^\top \right] \, \Direction
		% \end{equation*}
	% }
	% % \only<2>{
		% % \begin{equation*}
			% % \ControlDirectionOpt = \frac{\ControlDirectionOptNN}{\norm{\ControlDirectionOptNN}}, \quad \text{with} \quad
			% % \ControlDirectionOptNN = \sum_{k = 0}^{\infty} \frac{\TimeHorizon^k}{k!}  \left[ (\Gradients)^\top \right]^k \cdot \Direction
		% % \end{equation*}
	% % }
	% % \only<4>{
		% % \begin{equation*}
			% % \ControlDirectionOpt = \frac{\ControlDirectionOptNN}{\norm{\ControlDirectionOptNN}}, \quad \text{with} \quad
			% % \ControlDirectionOptNN = \Direction + \TimeHorizon \, (\Gradients)^\top \cdot \Direction + \frac{1}{2} \TimeHorizon^2 [ (\Gradients)^\top ]^2 \cdot \Direction + \dotsb
		% % \end{equation*}
	% % }
	% \only<2->{
		% \begin{equation*}
			% \ControlDirectionOpt = \frac{\ControlDirectionOptNN}{\norm{\ControlDirectionOptNN}}, \quad \text{with} \quad
			% \ControlDirectionOptNN = \alt<3>{\textcolor{ColorSurf}{\Direction}}{\Direction} + \TimeHorizon \, \alt<4>{\textcolor{ColorSurf}{\vec{\nabla} \FlowVelocityScalar_z}}{\vec{\nabla} \FlowVelocityScalar_z} + \frac{1}{2} \TimeHorizon^2 \, \alt<6->{\textcolor{ColorSurf}{\vec{\nabla} ( \FlowVelocity \cdot \vec{\nabla} \FlowVelocityScalar_\DirectionScalar )}}{\vec{\nabla} ( \FlowVelocity \cdot \vec{\nabla} \FlowVelocityScalar_\DirectionScalar )} + \dotsb
		% \end{equation*}
	% }
% 
	% \visible<2->{
		% \begin{figure}
			% \begin{tikzpicture}
				% \visible<2>{
					% \node[inner sep=0pt] (copepod) at (0,0) {
						% \def\svgwidth{0.3\textwidth}
						% \input{parts/surf/schemes/physical_interpretation_0.pdf_tex}
					% };
				% }
				% \visible<3>{
					% \node[inner sep=0pt] (copepod) at (0,0) {
						% \def\svgwidth{0.3\textwidth}
						% \input{parts/surf/schemes/physical_interpretation_1.pdf_tex}
					% };
				% }
				% \visible<4>{
					% \node[inner sep=0pt] (copepod) at (0,0) {
						% \def\svgwidth{0.3\textwidth}
						% \input{parts/surf/schemes/physical_interpretation_2.pdf_tex}
					% };
				% }
				% \visible<5>{
					% \node[inner sep=0pt] (copepod) at (0,0) {
						% \def\svgwidth{0.3\textwidth}
						% \input{parts/surf/schemes/physical_interpretation_25.pdf_tex}
					% };
				% }
				% \visible<6->{
					% \node[inner sep=0pt] (copepod) at (0,0) {
						% \def\svgwidth{0.3\textwidth}
						% \input{parts/surf/schemes/physical_interpretation_3.pdf_tex}
					% };
				% }
			% \end{tikzpicture}
			% \captionsetup{skip=10pt}
			% \caption{
				% Illustration of the surfing strategy in a simple vortex flow.
			% }
		% \end{figure}
	% }
% 
	% \vskip0pt plus 1filll
% 
	% \large
	% \begin{itemize}
		% \item<7-> succession of \textbf{``gradient ascents''} controlled by $\TimeHorizon$
	% \end{itemize}
	% \vspace{15pt}
% \end{frame}

% %-------------------------------------------------------
% % SURFING STRATEGY: Taylor-Green Vortices: Definition
% %-------------------------------------------------------
% 
% \begin{frame}{3 Solution: The \textbf{surfing} strategy}{1 Introduction - 2 Problem - 4 Performance - 5 Relevance - 6 Perspectives}
	% \vspace{5pt}
	% \centering
	% \textbf{\Large Taylor-Green Vortices}
% 
	% \pause
	% \begin{equation*}\label{eq:taylor_green_vortex_velocity}
		% \FlowVelocity(\vec{x}) = \FlowVelocityScalar_{\mathrm{max}}
		% \begin{pmatrix}
			% \cos (x/L) \, \sin (y/L) \\
			% -\sin (x/L) \, \cos (y/L)
		% \end{pmatrix},
		% \quad \text{with} \quad
		% \vec{x} =
		% \begin{pmatrix}
			% x \\
			% y
		% \end{pmatrix}
	% \end{equation*}
% 
	% \input{parts/surf/plots/tgv.tex}
% 
	% \vskip0pt plus 1filll
	% 
	% \pause
	% \large
	% \begin{itemize}
		% \item $\FlowVelocityScalar_{\max}$ and $\FlowVorticityScalar_{\max} = 2 \FlowVelocityScalar_{\max} / L$ are used to \textbf{scale} the results
	% \end{itemize}
	% \vspace{5pt}
% \end{frame}
