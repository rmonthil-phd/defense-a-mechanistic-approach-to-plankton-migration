%-------------------------------------------------------
% FORMULATION: PRESENTATION PLAN
%-------------------------------------------------------

\begin{frame}{5 Biophysical \textbf{relevance} of the strategy}{1 Introduction - 2 Problem - 3 Surfing - 4 Performance - 6 Perspectives}
	\centering
	\vspace{15pt}
	\textbf{\Large Outline}

	\vspace{15pt}

	\large
	\begin{enumerate}
		\setlength\itemsep{10pt}
		\item \textcolor{gray}{\textbf{Introduction} to the world of plankton}
		\item \textcolor{gray}{Formulation of the \textbf{navigation problem}}
		\item \textcolor{gray}{Solution: The \textbf{surfing} strategy}
		\item \textcolor{gray}{Evaluation of surfing \textbf{performance}}
		\item \textcolor{white}{Biophysical \textbf{relevance} of the strategy}
		\item \textcolor{gray}{Summary and \textbf{perspectives}}
	\end{enumerate}

\end{frame}

%-------------------------------------------------------
% RELEVANCE: SETUP
%-------------------------------------------------------

\begin{frame}{5 Biophysical \textbf{relevance} of the strategy}{1 Introduction - 2 Problem - 3 Surfing - 4 Performance - 6 Perspectives}
	\centering
	\vspace{5pt}
	\textbf{\large How beneficial would the surfing strategy be \\ for actual plankters in their habitat?}

	\pause
	\vspace{10pt}
	\begin{itemize}
		\item Surfing performance is impacted by $\displaystyle \frac{\SwimmingVelocity}{\KolmogorovVelocityScale}$ and $\displaystyle \frac{\tau_{\mathrm{align, \NameSurfShort}}}{\KolmogorovTimeScale}$.
	\end{itemize}

	\pause
	\vspace{5pt}
	\begin{figure}
		\begin{tikzpicture}
			\visible<3> {
				\node[inner sep=0pt] (copepod) at (0,0) {
					\def\svgwidth{0.9\textwidth}
					\input{parts/relevance/schemes/typical_plankton_0.pdf_tex}
				};
			}
			\visible<4> {
				\node[inner sep=0pt] (copepod) at (0,0) {
					\def\svgwidth{0.9\textwidth}
					\input{parts/relevance/schemes/typical_plankton_1.pdf_tex}
				};
			}
			\visible<5> {
				\node[inner sep=0pt] (copepod) at (0,0) {
					\def\svgwidth{0.9\textwidth}
					\input{parts/relevance/schemes/typical_plankton_2.pdf_tex}
				};
			}
			\visible<6> {
				\node[inner sep=0pt] (copepod) at (0,0) {
					\def\svgwidth{0.9\textwidth}
					\input{parts/relevance/schemes/typical_plankton_3.pdf_tex}
				};
			}
		\end{tikzpicture}
		\captionsetup{skip=5pt, margin={0.0\textwidth, 0.0\textwidth}}
		\caption{
			Typical plankter parameters. Size ratios are respected. $\delta = d /200$.
		}
	\end{figure}
\end{frame}

%-------------------------------------------------------
% RELEVANCE: RESULTS
%-------------------------------------------------------

\begin{frame}{5 Biophysical \textbf{relevance} of the strategy}{1 Introduction - 2 Problem - 3 Surfing - 4 Performance - 6 Perspectives}
	\centering
	\vspace{-5pt}
	\begin{figure}
		\begin{tikzpicture}
			\visible<1> {
				\node[inner sep=0pt] (copepod) at (0,0) {
					\def\svgwidth{0.48\textwidth}
					\input{parts/relevance/schemes/bio_relevance_0.pdf_tex}
				};
			}
			\visible<2> {
				\node[inner sep=0pt] (copepod) at (0,0) {
					\def\svgwidth{0.48\textwidth}
					\input{parts/relevance/schemes/bio_relevance_1.pdf_tex}
				};
			}
			\visible<3> {
				\node[inner sep=0pt] (copepod) at (0,0) {
					\def\svgwidth{0.48\textwidth}
					\input{parts/relevance/schemes/bio_relevance_2.pdf_tex}
				};
			}
			\visible<4> {
				\node[inner sep=0pt] (copepod) at (0,0) {
					\def\svgwidth{0.48\textwidth}
					\input{parts/relevance/schemes/bio_relevance_3.pdf_tex}
				};
			}
			\visible<5> {
				\node[inner sep=0pt] (copepod) at (0,0) {
					\def\svgwidth{0.48\textwidth}
					\input{parts/relevance/schemes/bio_relevance_4.pdf_tex}
				};
			}
		\end{tikzpicture}
		\captionsetup{skip=20pt}%, margin={0.0\textwidth, 0.0\textwidth}}
		\caption{
			Estimation of surfing performance as a function of plankton habitat.
		}
	\end{figure}
\end{frame}

%-------------------------------------------------------
% RELEVANCE: DINSTINGUISH
%-------------------------------------------------------

\begin{frame}{5 Biophysical \textbf{relevance} of the strategy}{1 Introduction - 2 Problem - 3 Surfing - 4 Performance - 6 Perspectives}
	\centering
	\vspace{5pt}
	\textbf{\large Surfing would be beneficial for actual plankters. \\ \visible<2->{But are plankters actually surfing?}}

	\vspace{15pt}
	\begin{columns}[c]
		\column{.55\textwidth}
		\centering
		\begin{itemize}
			\large
			\setlength\itemsep{10pt}
			\item<3-> DiBenedetto et al. (2022)
			\item<3-> \textit{Crepidula fornicata} larvae
			\item<4-> migrate upward to \textbf{disperse}
			\item<5-> migrate downward to \textbf{settle}
			\item<6-> \textbf{ongoing collaboration} with \\ Michelle DiBenedetto
		\end{itemize}

		\column{.4\textwidth}
		\centering
		\visible<3->{
			\begin{figure}
				\def\svgwidth{1.0\textwidth}
				\input{parts/relevance/schemes/shell.pdf_tex}
				\caption{
					Shell of \textit{Crepidula fornicata}.
				}
			\end{figure}
		}
	\end{columns}
\end{frame}

%-------------------------------------------------------
% RELEVANCE: EXPERIMENT
%-------------------------------------------------------

\begin{frame}{5 Biophysical \textbf{relevance} of the strategy}{1 Introduction - 2 Problem - 3 Surfing - 4 Performance - 6 Perspectives}
	\centering
	\vspace{5pt}
	\textbf{\large Description of the experiment}

	\vspace{0pt}
	\begin{figure}
		\begin{tikzpicture}
			\visible<1> {
				\node[inner sep=0pt] (copepod) at (0,0) {
					\def\svgwidth{0.75\textwidth}
					\input{parts/relevance/schemes/tank_0.pdf_tex}
				};
			}
			\visible<2> {
				\node[inner sep=0pt] (copepod) at (0,0) {
					\def\svgwidth{0.75\textwidth}
					\input{parts/relevance/schemes/tank_1.pdf_tex}
				};
			}
			\visible<3> {
				\node[inner sep=0pt] (copepod) at (0,0) {
					\def\svgwidth{0.75\textwidth}
					\input{parts/relevance/schemes/tank_2.pdf_tex}
				};
			}
			\visible<4-> {
				\node[inner sep=0pt] (copepod) at (0,0) {
					\def\svgwidth{0.75\textwidth}
					\input{parts/relevance/schemes/tank_3.pdf_tex}
				};
			}
		\end{tikzpicture}
		\captionsetup{skip=5pt}%, margin={0.0\textwidth, 0.0\textwidth}}
		\caption{
			Description of the experiment of DiBenedetto et al. (2022).
		}
	\end{figure}

	\vspace{-12pt}
	\only<5->{
		\begin{itemize}
			\item Slip velocity: $\vec{V}_s = (d \ParticlePosition / dt) - \FlowVelocity(\ParticlePosition, t)$ proxy of $\SwimmingVelocity \SwimmingDirection$
		\end{itemize}
	}
\end{frame}

%-------------------------------------------------------
% RELEVANCE: DISTINGUISH
%-------------------------------------------------------

\begin{frame}{5 Biophysical \textbf{relevance} of the strategy}{1 Introduction - 2 Problem - 3 Surfing - 4 Performance - 6 Perspectives}
	\centering
	\vspace{5pt}
	\textbf{\large How to differentiate surfing from bottom-heaviness?}
	
	\centering
	\vspace{10pt}
	\begin{figure}
		\begin{tikzpicture}
			\visible<1> {
				\node[inner sep=0pt] (copepod) at (0,0) {
					\def\svgwidth{0.8\textwidth}
					\input{parts/relevance/schemes/distinguish_0.pdf_tex}
				};
			}
			\visible<2> {
				\node[inner sep=0pt] (copepod) at (0,0) {
					\def\svgwidth{0.8\textwidth}
					\input{parts/relevance/schemes/distinguish_1.pdf_tex}
				};
			}
			\visible<3> {
				\node[inner sep=0pt] (copepod) at (0,0) {
					\def\svgwidth{0.8\textwidth}
					\input{parts/relevance/schemes/distinguish_2.pdf_tex}
				};
			}
			\visible<4> {
				\node[inner sep=0pt] (copepod) at (0,0) {
					\def\svgwidth{0.8\textwidth}
					\input{parts/relevance/schemes/distinguish_3.pdf_tex}
				};
			}
			\visible<5> {
				\node[inner sep=0pt] (copepod) at (0,0) {
					\def\svgwidth{0.8\textwidth}
					\input{parts/relevance/schemes/distinguish_4.pdf_tex}
				};
			}
			\visible<6> {
				\node[inner sep=0pt] (copepod) at (0,0) {
					\def\svgwidth{0.8\textwidth}
					\input{parts/relevance/schemes/distinguish_5.pdf_tex}
				};
			}
			\visible<7> {
				\node[inner sep=0pt] (copepod) at (0,0) {
					\def\svgwidth{0.8\textwidth}
					\input{parts/relevance/schemes/distinguish_6.pdf_tex}
				};
			}
		\end{tikzpicture}
		\captionsetup{skip=10pt}%, margin={0.0\textwidth, -0.2\textwidth}}
		\caption{
			Illustration of how to distinguish surfing from bottom-heaviness.
		}
	\end{figure}
\end{frame}

%-------------------------------------------------------
% RELEVANCE: NUMERICAL AND EXPERIMENTAL RESULTS
%-------------------------------------------------------

\begin{frame}{5 Biophysical \textbf{relevance} of the strategy}{1 Introduction - 2 Problem - 3 Surfing - 4 Performance - 6 Perspectives}
	\centering
	\vspace{5pt}
	\textbf{\large Numerical and experimental results}

	\vspace{-5pt}
	\begin{figure}
		\begin{tikzpicture}
	\begin{groupplot}[
			group style={
				group size=2 by 1,
				horizontal sep=0.14\linewidth,
			},
			axis on top,
			% size
			width=0.5\textwidth,
			%ymode=log,
			% layers
			set layers ,
			% legend
			legend style={draw=none, fill=none, /tikz/every even column/.append style={column sep=4pt}, at={(1.1, 1.05)}, anchor=south},
			%legend pos=north east,
			legend cell align=right,
			legend columns=-1,
		]
		% n_{surf, x}
		\nextgroupplot[
			% x
			xlabel=$\FlowVorticityScalar_y \KolmogorovTimeScale$,
			xmin=-1,
			xmax=1,
			% y
			ylabel=$\langle \SwimmingDirection_{x} \rangle$,
			ylabel shift = -5pt,
			ymin=-1,
			ymax=1,
			ytick={-1,0,1},
		]
			% tau 2.0
			\addplot[
				ColorSurf,
				mark=square*,
				only marks,
			] table [
				x expr={\thisrowno{0}},
				y expr={\thisrowno{1}},
				col sep=comma, 
				comment chars=\#,
				unbounded coords=discard,
			] {parts/relevance/data/control_surfers__flow__n_128__re_250/control_surfer__vs_1o0__surftimeconst_2o0__omegamax_1o0__vorticity_z__p_y.csv};
			\addlegendentry{\NameSurf}
			% reorientationtime 1.0
			\addplot[
				ColorBh,
				mark=o,
				only marks,
			] table [
				x expr={\thisrowno{0}},
				y expr={\thisrowno{1}},
				col sep=comma,
				comment chars=\#,
				unbounded coords=discard,
			] {parts/relevance/data/control_surfers__flow__n_128__re_250/spherical_riser__vs_1o0__reorientationtime_1o0__vorticity_z__p_y.csv};
			\addlegendentry{\NameBh \quad\quad}
			% ref
			\addplot
	        [
		        color=white,
		        opacity=0.5,
		        solid, 
		        on layer=axis background,
		        domain=-1:1,
		        forget plot,
	        ]{0};
	        % exp
	        \addlegendimage{ColorAlt,mark=asterisk,only marks} \addlegendentry{Data of DiBenedetto et al. (2022)}
		% n_{surf, x}
		\nextgroupplot[
			% x
			xlabel=$\FlowVorticityScalar_y \KolmogorovTimeScale$,
			xmin=-1,
			xmax=1,
			% y
			ylabel=$\langle V_{s,x} \rangle / \langle V_{s,z} \rangle$,
			ylabel shift = -15pt,
			ymin=-1,
			ymax=1,
			ytick={-1,0,1},
		]
			% shade
	        \addplot[name path=A, draw=none, forget plot] table [
	            x expr={\thisrowno{0} * 0.67},
	            y expr={\thisrowno{3} / 0.069},
	            col sep=comma, 
	            comment chars=\#,
	        ]{parts/relevance/data/exp/exp_data.csv};
	        \addplot[name path=B, draw=none, forget plot] table [
	            x expr={\thisrowno{0} * 0.67},
	           	y expr={\thisrowno{2} / 0.069},
	            col sep=comma,
	            comment chars=\#,
	        ]{parts/relevance/data/exp/exp_data.csv};
	        \addplot[ColorAlt, opacity=0.4, forget plot, visible on=<2->] fill between[of=A and B];
			% tau 1.0
			\addplot[
				ColorAlt,
				mark=asterisk,
				only marks,
				visible on=<2->
			] table [
				x expr={\thisrowno{0} * 0.67},
				y expr={\thisrowno{1} / 0.069},
				col sep=comma, 
				comment chars=\#,
				unbounded coords=discard,
			] {parts/relevance/data/exp/exp_data.csv};
			% ref
			\addplot
	        [
		        color=white,
		        opacity=0.5,
		        solid, 
		        on layer=axis background,
		        domain=-1:1,
	        ]{0};
	\end{groupplot}
\end{tikzpicture}

		\captionsetup{skip=0pt}%, margin={0.0\textwidth, 0.0\textwidth}}
		\caption{
			Orientation statistics as a function of flow vorticity $\omega_y$ obtained from \textbf{(left)} simulations \textbf{(right)} experiments of DiBenedetto et al. (2022).
		}
	\end{figure}

	\vspace{-10pt}
	\begin{itemize}
		\item<3-> Might be the \textbf{first observation} of plankter ``surfing'' (\textbf{ongoing})
	\end{itemize}
\end{frame}













% %-------------------------------------------------------
% % RELEVANCE: PHOTOTACTIC SURFING
% %-------------------------------------------------------
% 
% \begin{frame}{5 Biophysical \textbf{relevance} of the strategy}{1 Introduction - 2 Problem - 3 Surfing - 4 Performance - 6 Perspectives}
	% \centering
	% \vspace{15pt}
	% \textbf{\large Many plankters are phototactic!}
% 
	% \pause
	% \vspace{30pt}
	% \def\svgwidth{0.8\textwidth}
	% \input{parts/relevance/schemes/phototactic_tank.pdf_tex}
% \end{frame}
