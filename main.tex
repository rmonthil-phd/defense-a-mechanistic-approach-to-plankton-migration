\documentclass[t,10pt,xcolor={dvipsnames, table}]{beamer}
\usetheme[
%%% option passed to the outer theme
%    progressstyle=fixedCircCnt,   % fixedCircCnt, movingCircCnt (moving is deault)
  ]{Feather}

% Colors

\input{colors}  

% Change back color
\setbeamercolor{background canvas}{bg=black}
% Change the bar colors:
\setbeamercolor{Feather}{fg=ColorSurf!30!black,bg=ColorSurf!60!black}
% Change the color of the structural elements:
\setbeamercolor{structure}{fg=ColorSurf}
% Change the frame title text color:
\setbeamercolor{frametitle}{fg=white!5}
% Change the normal text colors:
\setbeamercolor{normal text}{fg=white!75,bg=black}
% Change the block title colors
\setbeamercolor{block title}{use=Feather,bg=Feather.fg, fg=white!90} 


% Change the logo in the upper right circle:
\renewcommand{\logofile}{imgs/logo_irphe.png} 
%% This is an image that comes with the LaTeX installation
% Adjust scale of the logo w.r.t. the circle; default is 0.875
\renewcommand{\logoscale}{0.8}

% Change the background image on the title and final page.
% It stretches to fill the entire frame!
\renewcommand{\backgroundfile}{imgs/fond_dark.pdf}
\renewcommand{\backgroundopacity}{0.0}

%-------------------------------------------------------
% INCLUDE PACKAGES
%-------------------------------------------------------

\usepackage[utf8]{inputenc}
\usepackage[english]{babel}
% \usepackage{helvet}
\usepackage{multicol}

\usepackage{soul} % st
\usepackage{cancel}

\usepackage[arrow]{hhtensor} % \matr, \vec

%\usepackage[table]{xcolor}

%-------------------------------------------------------
% DEFFINING AND REDEFINING COMMANDS
%-------------------------------------------------------

% colored hyperlinks
\newcommand{\chref}[2]{
  \href{#1}{{\usebeamercolor[bg]{Feather}#2}}
}

\graphicspath{{schemes/}{parts/intro/images}{parts/intro/schemes}{parts/formulation/schemes}{parts/surf/schemes}{parts/relevance/schemes}{parts/turbulence/schemes}{parts/upscaling}}

%-------------------------------------------------------
% PRESENTATION SPECIFIC STUFF
%-------------------------------------------------------

\usepackage{suffix}
\usepackage{tikz}
\usetikzlibrary{math}
\usetikzlibrary{external}
\tikzexternalize[prefix=./export_figures/]
\tikzexternaldisable
\makeatletter
	\tikzset{% https://tex.stackexchange.com/a/79572/39222
		beamer externalising/.style={%
			execute at end picture={%
				\tikzifexternalizing{%
					\ifbeamer@anotherslide
						\pgfexternalstorecommand{\string\global\string\beamer@anotherslidetrue}%
					\fi
				}{}%
			}%
		},
		external/optimize=false
	}
\makeatother
%\tikzset{external/force remake}
\tikzset{
	invisible/.style={opacity=0},
	visible on/.style={alt={#1{}{invisible}}},
	alt/.code args={<#1>#2#3}{%
		\alt<#1>{\pgfkeysalso{#2}}{\pgfkeysalso{#3}} % \pgfkeysalso doesn't change the path
		},
}
\usepackage{pgf}
\usepackage{pgfplots, pgfplotstable}
\usepgfplotslibrary{groupplots}
\usepgfplotslibrary{fillbetween}
\pgfplotsset{compat=newest}

\pgfplotsset{select coords between index/.style 2 args={
    x filter/.code={
        \ifnum\coordindex<#1\def\pgfmathresult{}\fi
        \ifnum\coordindex>#2\def\pgfmathresult{}\fi
    }
}}

\usepackage{multimedia}
\usepackage{animate}

\usepackage{amsmath}
\usepackage{mathtools}

\usepackage{siunitx}
\setbeamertemplate{caption}[numbered]
\usepackage{ccicons} % creative commons symbols
\usepackage{caption} % creative commons symbols

\usepackage[nomessages]{fp} % calc

\input{notations}

\usepackage{amsmath}
\usepackage{multicol}

%-------------------------------------------------------
% INFORMATION IN THE TITLE PAGE
%-------------------------------------------------------

%\title[Tracking the source of a passive scalar in turbulence] % [] is optional - is placed on the bottom of the sidebar on every slide
%{ % is placed on the title page
%      \textbf{\LARGE Tracking the source of a passive scalar in turbulence}\\
%}

\title[A mechanistic approach to plankton migration] % [] is optional - is placed on the bottom of the sidebar on every slide
{ % is placed on the title page
	\vspace{-1.5cm}
	\textbf{\LARGE A mechanistic approach to\\ plankton migration}
}

\subtitle[]
{
}

\author[Rémi Monthiller]
{   
	\vspace{-25pt}
	\large \textbf{Rémi Monthiller}\\
	\vspace{5pt}
	\normalsize Ph.D. supervisors: \textbf{Benjamin Favier and Christophe Eloy}\\
	%\vspace{20pt}
	%Aix Marseille Univ, CNRS, Centrale Méditerranée, IRPHE, Marseille, France\\
	\vspace{20pt}
	This project has received funding from the European Research Council (ERC) under the European Union’s Horizon 2020 research and innovation programme (grant agreement No 834238).
	\vspace{20pt}
}

\institute[IRPHE]
{%
}

%\date{}
\date{\large December 20, 2022}
%\date{\today}

%-------------------------------------------------------
% THE BODY OF THE PRESENTATION
%-------------------------------------------------------

%\batchmode

\begin{document}

%-------------------------------------------------------
% THE TITLEPAGE
%-------------------------------------------------------

{\1% % this is the name of the PDF file for the background
\begin{frame}[plain,noframenumbering] % the plain option removes the header from the title page, noframenumbering removes the numbering of this frame only
  \titlepage % call the title page information from above
\end{frame}}

%-------------------------------------------------------
% INTRODUCTION: PRESENTATION PLAN
%-------------------------------------------------------

\begin{frame}{1 Introduction}{\visible<2->{2 Problem} \visible<3->{- 3 Surfing} \visible<4->{ - 4 Performance} \visible<5->{- 5 Relevance} \visible<6->{ - 6 Perspectives}}
	\centering
	\vspace{15pt}
	\textbf{\Large Outline}

	\vspace{15pt}

	\large
	\begin{enumerate}
		\setlength\itemsep{10pt}
		\item<1-> \textbf{Introduction} to the world of plankton
		\item<2-> \alt<7->{\textcolor{gray}{Formulation of the \textbf{navigation problem}}}{Formulation of the \textbf{navigation problem}}
		\item<3-> \alt<7->{\textcolor{gray}{\textbf{Solution}: The \textbf{surfing} strategy}}{Solution: The \textbf{surfing} strategy}
		\item<4-> \alt<7->{\textcolor{gray}{Evaluation of surfing \textbf{performance}}}{Evaluation of surfing \textbf{performance}}
		\item<5-> \alt<7->{\textcolor{gray}{Biophysical \textbf{relevance} of the strategy}}{Biophysical \textbf{relevance} of the strategy}
		\item<6-> \alt<7->{\textcolor{gray}{Summary and \textbf{perspectives}}}{Summary and \textbf{perspectives}}
	\end{enumerate}

\end{frame}

%-------------------------------------------------------
% INTRODUCTION: WHAT ARE PLANKTON?
%-------------------------------------------------------

\begin{frame}{1 Introduction}{2 Problem - 3 Surf - 4 Performance - 5 Relevance - 6 Perspectives}
	\centering
	\vspace{4pt}
	\textbf{\Large What are plankton?}

	\pause
	\begin{tikzpicture}
		\node[inner sep=0pt] (copepod) at (0,2.8) {
			\parbox{1.0\textwidth}{
				\begin{figure}
					\centering
					\def\svgwidth{1.0\textwidth}
					\input{parts/intro/schemes/plankton_definition.pdf_tex}
					\captionsetup{skip=5pt}
					\caption{
						Plankters are defined as being \textbf{carried} by (ocean) \textbf{currents}.
					}
				\end{figure}
			}
		};
		\visible<3->{
			\node[inner sep=0pt] (copepod) at (0,0) {
				\parbox{1.0\textwidth}{
					\begin{figure}
						\centering
						\def\svgwidth{1.0\textwidth}
						\input{parts/intro/schemes/plankton_size.pdf_tex}
						\captionsetup{skip=20pt}
						\caption{
							Illustration a \textbf{plankton size range}.
						}
					\end{figure}
				}
			};
			\draw[-stealth, ColorSurf, very thick] (-5.5,-0.6) -- (5.5,-0.6);
		}
		\visible<5->{
			\node[inner sep=0pt] (copepod) at (0,0.6) {
				\def\svgwidth{1.0\textwidth}
				\input{parts/intro/schemes/plankton_size_shadow.pdf_tex}
			};
			\draw[-stealth, ColorSurf, very thick] (-5.5,-0.6) -- (5.5,-0.6);
		}
	\end{tikzpicture}

	\vspace{-10pt}
		
	\large
	\begin{itemize}
		\item <4-> Many plankters are \textbf{actually able to swim} !
	\end{itemize}
\end{frame}

% %-------------------------------------------------------
% % INTRODUCTION: PLAKNTON SENSING
% %-------------------------------------------------------
%
% \begin{frame}{1 Introduction}{1.2 Plankters can display complex behaviors!}
	% \centering
	% \vspace{2pt}
	% \textbf{\Large Plankter can display complex behaviors!}
%
	% \vspace{2pt}
	% \begin{tikzpicture}
		% \visible<2->{
			% \node[inner sep=0pt] (copepod) at (0,0) {
				% \parbox{0.5\textwidth}{
					% \begin{figure}
						% \centering
						% \def\svgwidth{0.35\textwidth}
						% \input{parts/intro/schemes/copepod_measure_strain.pdf_tex}
						% \captionsetup{skip=5pt, width=0.4\textwidth}
						% \caption{
							% Simplified illustration of the \textbf{bioligical pump}.
						% }
					% \end{figure}
				% }
			% };
		% }
		% \visible<3->{
			% \node[inner sep=0pt] (copepod) at (0.5\textwidth,0) {
				% \parbox{0.5\textwidth}{
					% \begin{figure}
						% \centering
						% \def\svgwidth{0.35\textwidth}
						% \input{parts/intro/schemes/larval_measure_vorticity.pdf_tex}
						% \captionsetup{skip=5pt, width=0.4\textwidth}
						% \caption{
							% Simplified illustration of the \textbf{bioligical pump}.
						% }
					% \end{figure}
				% }
			% };
		% }
	% \end{tikzpicture}
%
	% %\vskip0pt plus 1filll
%
	% \large
	% \begin{itemize}
		% \item measure the \textbf{gradients} $\Gradients ( \ParticlePosition, t )$, not the velocity $\FlowVelocity (\vec{x}, t)$
	% \end{itemize}
	% %\vspace{15pt}
% \end{frame}

%-------------------------------------------------------
% INTRODUCTION: WHY ARE PLANKTON IMPORTANT?
%-------------------------------------------------------

\begin{frame}{1 Introduction}{2 Problem - 3 Surf - 4 Performance - 5 Relevance - 6 Perspectives}
	\centering
	\vspace{10pt}
	\textbf{\Large Why are plankton important?}

	\pause
	\vspace{10pt}
	\begin{figure}
		\begin{tikzpicture}
			\visible<2>{
				\node[inner sep=0pt] (copepod) at (0,0) {
					\def\svgwidth{0.4\textwidth}
					\input{parts/intro/schemes/plankton_importance_0.pdf_tex}
				};
			}
			\visible<3>{
				\node[inner sep=0pt] (copepod) at (0,0) {
					\def\svgwidth{0.4\textwidth}
					\input{parts/intro/schemes/plankton_importance_1.pdf_tex}
				};
			}
			% \visible<4>{
				% \node[inner sep=0pt] (copepod) at (0,0) {
					% \def\svgwidth{0.4\textwidth}
					% \input{parts/intro/schemes/plankton_importance_2.pdf_tex}
				% };
			% }
			\visible<4>{
				\node[inner sep=0pt] (copepod) at (0,0) {
					\def\svgwidth{0.4\textwidth}
					\input{parts/intro/schemes/plankton_importance_3.pdf_tex}
				};
			}
			% \visible<7>{
				% \node[inner sep=0pt] (copepod) at (0,0) {
					% \def\svgwidth{0.4\textwidth}
					% \input{parts/intro/schemes/plankton_importance_4.pdf_tex}
				% };
			% }
			\node[inner sep=0pt] (copepod) at (6,0) {
				\parbox{0.6\textwidth}{
					\large
					\begin{itemize}
						\setlength\itemsep{20pt}
						\large
						\item<2-> \textbf{Marine ecology:} \textbf{basis} of the \textbf{marine food web}
						\item<4-> \textbf{Climate:} carbon trapping
					\end{itemize}
				}
			};
		\end{tikzpicture}
		\captionsetup{skip=7pt, margin={0.0\textwidth, 0.0\textwidth}}
		\caption{
			Illustration of the importance of plankton dynamics. \visible<4->{Ducklow et al. (2001).}
		}
	\end{figure}

	\vskip0pt plus 1filll

	\large
	\vspace{15pt}
\end{frame}

%-------------------------------------------------------
% INTRODUCTION: WHAT ARE PLANKTON?: NET PLANKTON
%-------------------------------------------------------

\begin{frame}{1 Introduction}{2 Problem - 3 Surf - 4 Performance - 5 Relevance - 6 Perspectives}
	\vspace{6pt}
	\centering
	\leavevmode\hidewidth\begin{tikzpicture}
		\node[inner sep=0pt] (copepod) at (0,0) {
		\parbox{1.0\textwidth}{
			\begin{figure}
				\centering
				\def\svgwidth{0.8\textwidth}
				\input{parts/intro/images/marine_plankton.pdf_tex}
				%\captionsetup{skip=5pt, width=0.4\textwidth}
				\caption{
					Marine plankters. Adapted from Nadeau et al. \ccbysa ~ v4.0.
				}
			\end{figure}
		}};
		\visible<4->{
			\draw[-stealth, ColorSurf, very thick] (-4.7,1.5) -- (-4,1.5) node[pos=0, anchor=east]{\textcolor{ColorSurf}{\small Copepods}};
			\draw[-stealth, ColorSurf, very thick] (-4.7,1.5) -- (-3.5,0.4);
			\draw[-stealth, ColorSurf, very thick] (-4.7,1.5) -- (-2.6,2.3);
		}
		\visible<3->{
			\draw[-stealth, ColorSurf, very thick] (4.5,-1.6) -- (4.0,-1.6) node[pos=0, anchor=west]{\textcolor{ColorSurf}{\small Crab larva}};
		}
		\visible<2->{
			\draw[-stealth, ColorSurf, very thick] (4.7,0.85) -- (4.1,0.85) node[pos=0, anchor=west]{\textcolor{ColorSurf}{\small Eggs}};
			\draw[-stealth, ColorSurf, very thick] (4.7,0.85) -- (4.05,0.25);
			\draw[-stealth, ColorSurf, very thick] (4.7,0.85) -- (3.4,1.2);
		}
		\node[opacity=0.0] (copepod) at (5.0,0) {phantom};
	\end{tikzpicture}\hidewidth\null
\end{frame}

%-------------------------------------------------------
% INTRODUCTION: A COPEPOD, AS AN EXAMPLE OF PLANKTER
%-------------------------------------------------------

\begin{frame}{1 Introduction}{2 Problem - 3 Surf - 4 Performance - 5 Relevance - 6 Perspectives}
	\centering
	\vspace{4pt}
	\textbf{\Large Copepods: an example of plankton}

	\vspace{0pt}
	\begin{figure}
		\leavevmode\hidewidth\begin{tikzpicture}
			\visible<1>{
				\node[inner sep=0pt] (copepod) at (0,-1) {
					\def\svgwidth{0.4\textwidth}
					\input{parts/intro/schemes/copepod_picture_0.pdf_tex}
				};
			}
			\visible<2-3>{
				\node[inner sep=0pt] (copepod) at (0,-1) {
					\def\svgwidth{0.4\textwidth}
					\input{parts/intro/schemes/copepod_picture_1.pdf_tex}
				};
			}
			\visible<4-5>{
				\node[inner sep=0pt] (copepod) at (0,-1) {
					\def\svgwidth{0.4\textwidth}
					\input{parts/intro/schemes/copepod_picture_2.pdf_tex}
				};
			}
			\only<6->{
				\node[inner sep=0pt] (copepod) at (0,0) {
					% \movie[width=0.320\textwidth, height=0.340\textwidth, autostart, loop]{}{parts/intro/videos/copepod_escape.gif}
					\movie[width=0.480\textwidth, height=0.510\textwidth, autostart, loop]{}{parts/intro/videos/copepod_ambush_attack.mp4}
				};
			}
			\node[inner sep=0pt] (copepod) at (6,0) {
				\parbox{0.6\textwidth}{
					\large
					\begin{itemize}
						\setlength\itemsep{10pt}
						\large
						\item<2-> \textbf{Sensing}
						\begin{itemize}
							\setlength\itemsep{2pt}
							\normalsize
							\item<3->[$\bullet$] light intensity
							\item<5->[$\bullet$] local hydrodynamic signals
						\end{itemize}
						\item<6-> \textbf{Behavior}
						\begin{itemize}
							\setlength\itemsep{2pt}
							\normalsize
							\item<6->[$\bullet$] neurons
							\item<7->[$\bullet$] catch preys and escape predators
						\end{itemize}
					\end{itemize}
				}
			};
			\node[opacity=0.0] (copepod) at (9.0,0) {phantom};
		\end{tikzpicture}\hidewidth\null
		\captionsetup{skip=6pt, margin={-0.0\textwidth, -0.0\textwidth}}
		\caption{
			\alt<6->{\textit{Acartia Tonsa} ambush attack (270 x real time). Adapted from Kiørboe T. et al. (2009).}{\textit{Acartia tonsa}, a calanoid copepod. \copyright Micropolitan.org}
		}
	\end{figure}
\end{frame}

%-------------------------------------------------------
% INTRODUCTION: Vertical migration of plankton
%-------------------------------------------------------

\begin{frame}{1 Introduction}{2 Problem - 3 Surf - 4 Performance - 5 Relevance - 6 Perspectives}
	\centering
	\vspace{5pt}
	\textbf{\Large Diel vertical migration}

	\pause
	\vspace{0pt}
	\begin{figure}
		\leavevmode\hidewidth\begin{tikzpicture}
			\visible<2->{
				\node[inner sep=0pt] (copepod) at (0,0) {
					\def\svgwidth{0.35\textwidth}
					\input{parts/intro/schemes/vertical_migration_0.pdf_tex}
				};
				\visible<3->{
					\node[inner sep=0pt] (copepod) at (0,0) {
						\def\svgwidth{0.35\textwidth}
						\input{parts/intro/schemes/vertical_migration_1.pdf_tex}
					};
				}
				\visible<4->{
					\node[inner sep=0pt] (copepod) at (0,0) {
						\def\svgwidth{0.35\textwidth}
						\input{parts/intro/schemes/vertical_migration_2.pdf_tex}
					};
				}
				\visible<6->{
					\node[inner sep=0pt] (copepod) at (0,0) {
						\def\svgwidth{0.35\textwidth}
						\input{parts/intro/schemes/vertical_migration_4.pdf_tex}
					};
				}
				\visible<5->{
					\node[inner sep=0pt] (copepod) at (0,0) {
						\def\svgwidth{0.35\textwidth}
						\input{parts/intro/schemes/vertical_migration_3.pdf_tex}
					};
				}
				%\node[opacity=0.0] (copepod) at (9,0) {phantom};
			}
		\end{tikzpicture}\hidewidth\null
		\captionsetup{skip=5pt, margin={0.0\textwidth, 0.0\textwidth}}
		\caption{
			Illustration of plankton \textbf{vertical migrations}. Bandara et al., (2021).
			}
	\end{figure}

	\vskip0pt plus 1filll

	\large
	\begin{itemize}
		%\setlength\itemsep{10pt}
		\item<7-> \textbf{Distance:} $\sim$100\unit{\meter} $\gg$ 1\unit{\milli\meter} ~~ $\sim$10 marathons a day!
		%\item<8-> \textbf{Observation:} Copepods can react to flow sensing!
		%\item<9-> \textbf{Question:} \textbf{How could \textbf{plankters} use \textbf{hydrodynamic cues}\\ to optimize their \textbf{navigation}?}
	\end{itemize}
	\vspace{15pt}
\end{frame}


\begin{frame}{1 Introduction}{2 Problem - 3 Surf - 4 Performance - 5 Relevance - 6 Perspectives}
	\centering
	\vspace{0pt}
	\begin{multicols}{2}
		\begin{itemize}
			\large
			\setlength\itemsep{4pt}
			\item <1-> diel vertical migration
			\item <2-> react to the local flow
		\end{itemize}
	\end{multicols}

	\vspace{6pt}
	\begin{figure}
		\begin{tikzpicture}
			\visible<3->{
				\node[inner sep=0pt] (copepod) at (0,0) {
					\def\svgwidth{0.45\textwidth}
					\input{parts/intro/schemes/problem.pdf_tex}
				};
			}
		\end{tikzpicture}
		% \captionsetup{skip=5pt, margin={0.0\textwidth, 0.0\textwidth}}
		% \caption{
			% Illustration of problem investigated in this study.
		% }
	\end{figure}

	\vskip0pt plus 1filll

	\pause
	\vspace{-5pt}
	\visible<3->{
		\begin{center}
			\Large
			How could \textbf{plankters} use their \textbf{flow sensing}\\ to optimize their \textbf{navigation}?
		\end{center}
	}
	\vspace{15pt}
\end{frame}

%-------------------------------------------------------
% INTRODUCTION: LITERATURE
%-------------------------------------------------------

\begin{frame}{1 Introduction}{2 Problem - 3 Surf - 4 Performance - 5 Relevance - 6 Perspectives}
	\vspace{0pt}
	\begin{center}
		\Large
		How could \textbf{plankters} use their \textbf{flow sensing}\\ to optimize their \textbf{navigation}?
		\pause
		\vspace{10pt}
		\large
		\begin{itemize}
			%\item<2-> Zermelo navigation problem or bird soaring problems.
			%    \begin{itemize}
			%        \item But at most, planktoners only measure local velocity gradients
			%    \end{itemize}
			\item<2-> Plankton observation: data-driven approach
			\item<3-> Machine learning: reinforcement learning approach
		\end{itemize}
	\end{center}
	
	\vspace{0pt}
	\begin{itemize}
		\item<4-> \textbf{Physics-based} approach: \textbf{approximate analytical solution}.
	\end{itemize}
	
	\vskip0pt plus 1filll
	\visible<2->{
		\scriptsize
		\begin{multicols}{2}
			\begin{itemize}
				\item[] Koehl et al., Mar. Ecol. Prog. Ser. (2007)
				\item[] Michalec et al., Elife (2020)
				\item[] Montgomery et al., J. Mar. Biolog. Assoc. (2019)
			\end{itemize}
		\end{multicols}
	}
	\visible<3->{
		\scriptsize
		\begin{multicols}{2}
			\begin{itemize}
				\item[] Alageshan et al., Phys. Rev. E (2020)
				\item[] Gustavsson et al., Eur Phys J E (2017)
				\item[] Cichos et al., Nat. Mach. Intell. (2020)
				%\item[] Daddi-Moussa-Ider et al., Commun. Phys. (2021)
				\item[] Gunnarson et al., Nature (2021)
				\item[] Biferale et al., Chaos (2019)
				\item[] Reddy et al., PNAS (2016)
			\end{itemize}
		\end{multicols}
	}
	\visible<4->{
		\scriptsize
		\begin{multicols}{2}
			\begin{itemize}
				\item[] Bollt et al., J. Fluid Mech. (2021)
				\item[] Liebchen et al., EPL (2019)
			\end{itemize}
		\end{multicols}
	}
\end{frame}

%-------------------------------------------------------
% FORMULATION: PRESENTATION PLAN
%-------------------------------------------------------

\begin{frame}{2 Formulation of the \textbf{navigation problem}}{1 Introduction - 3 Surfing - 4 Performance - 5 Relevance - 6 Perspectives}
	\centering
	\vspace{15pt}
	\textbf{\Large Outline}

	\vspace{15pt}

	\large
	\begin{enumerate}
		\setlength\itemsep{10pt}
		\item \textcolor{gray}{\textbf{Introduction} to the world of plankton}
		\item \textcolor{white}{Formulation of the \textbf{navigation problem}}
		\item \textcolor{gray}{Solution: The \textbf{surfing} strategy}
		\item \textcolor{gray}{Evaluation of surfing \textbf{performance}}
		\item \textcolor{gray}{Biophysical \textbf{relevance} of the strategy}
		\item \textcolor{gray}{Summary and \textbf{perspectives}}
	\end{enumerate}

\end{frame}

%-------------------------------------------------------
% FORMULATION: PLANKTER MODEL
%-------------------------------------------------------

\begin{frame}{2 Formulation of the \textbf{navigation problem}}{1 Introduction - 3 Surfing - 4 Performance - 5 Relevance - 6 Perspectives}
	% \centering
	% \vspace{5pt}
	% \textbf{\Large Plankter model}

	\vspace{-10pt}
	\begin{figure}
		\leavevmode\hidewidth\begin{tikzpicture}
			\visible<1-2>{ % small
				\node[inner sep=0pt] (copepod) at (0,0) {
					\def\svgwidth{0.35\textwidth}
					\input{parts/formulation/schemes/plankter_model_0.pdf_tex}
				};
				\node[ColorBh] (copepod) at (-0.7,-0.7) {
					$\FlowVelocity(\vec{x}, t)$
				};
			}
			\visible<3>{ % copepod focus
				\node[inner sep=0pt] (copepod) at (0,0) {
					\def\svgwidth{0.5\textwidth}
					\input{parts/formulation/schemes/plankter_model_1.pdf_tex}
				};
			}
			\visible<4>{ % notations
				\node[inner sep=0pt] (copepod) at (0,0) {
					\def\svgwidth{0.5\textwidth}
					\input{parts/formulation/schemes/plankter_model_2.pdf_tex}
				};
			}
			\visible<5>{ % sphere
				\node[inner sep=0pt] (copepod) at (0,0) {
					\def\svgwidth{0.5\textwidth}
					\input{parts/formulation/schemes/plankter_model_4.pdf_tex}
				};
			}
			\visible<6-7>{ % neutrally buoyant
				\node[inner sep=0pt] (copepod) at (0,0) {
					\def\svgwidth{0.5\textwidth}
					\input{parts/formulation/schemes/plankter_model_5.pdf_tex}
				};
			}
			\visible<8>{ % swimming velocity
				\node[inner sep=0pt] (copepod) at (0,0) {
					\def\svgwidth{0.5\textwidth}
					\input{parts/formulation/schemes/plankter_model_6.pdf_tex}
				};
			}
			\visible<9>{ % sensing
				\node[inner sep=0pt] (copepod) at (0,0) {
					\def\svgwidth{0.5\textwidth}
					\input{parts/formulation/schemes/plankter_model_7.pdf_tex}
				};
			}
			\visible<10>{ % preferred direction
				\node[inner sep=0pt] (copepod) at (0,0) {
					\def\svgwidth{0.5\textwidth}
					\input{parts/formulation/schemes/plankter_model_8.pdf_tex}
				};
			}
			\visible<11-12>{ % instantaneous reorientation
				\node[inner sep=0pt] (copepod) at (0,0) {
					\def\svgwidth{0.5\textwidth}
					\input{parts/formulation/schemes/plankter_model_9.pdf_tex}
				};
			}
			\visible<13>{ % memoryless
				\node[inner sep=0pt] (copepod) at (0,0) {
					\def\svgwidth{0.5\textwidth}
					\input{parts/formulation/schemes/plankter_model_10.pdf_tex}
				};
			}
				
			\node[inner sep=0pt] (copepod) at (6.5,0) {
				\parbox{0.7\textwidth}{
					\vspace{10pt}
					\begin{itemize}
						\large
						\setlength\itemsep{5pt}
						\item<1-> Flow velocity field: \textcolor{ColorBh}{$\FlowVelocity(\vec{x}, t)$}
						\item<2-> Plankter physical properties
						\begin{itemize}
							\normalsize
							\setlength\itemsep{2pt}
							\item<2->[$\bullet$] small compared to the smallest flow scale
							\item<5->[$\bullet$] spherical
							\item<6->[$\bullet$] neutrally buoyant
							\item<7->[$\bullet$] inertialess
						\end{itemize}
						\item<8-> Active behavior
						\begin{itemize}
							\normalsize
							\setlength\itemsep{2pt}
							\item<8->[$\bullet$] constant swimming velocity $\SwimmingVelocity$
							\item<9->[$\bullet$] sensing: \st{$\FlowVelocity(\vec{x}, t)$}, $\Gradients$ and the vertical $\Direction$
							\item<10->[$\bullet$] preferred direction $\ControlDirection$
							\item<11->[$\bullet$] instantaneous reorientation
							\item<13->[$\bullet$] memoryless
						\end{itemize}
					\end{itemize}
				}
			};
			\node[opacity=0.0] (copepod) at (10.5,0) {phantom};
		\end{tikzpicture}\hidewidth\null
		\captionsetup{skip=-15pt, margin={-0.8\textwidth, 0.0\textwidth}}
		\caption{
			Plankter model.
		}
	\end{figure}

	% \only<11>{
		% \vspace{-18pt}
		% \begin{equation*}
			% \frac{d \ParticlePosition}{d t} = \FlowVelocity(\ParticlePosition, t) + \SwimmingVelocity \,\SwimmingDirection(t), \quad\quad \SwimmingDirection = \ControlDirection
		% \end{equation*}
	% }
	\only<12->{
		\vspace{-18pt}
		\begin{equation*}
			\frac{d \ParticlePosition}{d t} = \FlowVelocity(\ParticlePosition, t) + \SwimmingVelocity \,\ControlDirection(t)
		\end{equation*}
	}
\end{frame}

\begin{frame}{2 Formulation of the \textbf{navigation problem}}{1 Introduction - 3 Surfing - 4 Performance - 5 Relevance - 6 Perspectives}
	\centering
	\vspace{10pt}
	\textbf{\Large Navigation problem}

	\pause
	\vspace{-10pt}
	\begin{equation*}
		\large
		\text{find} ~ \ControlDirection ~ \text{such that} ~ \ParticlePosition(\FinalTime) \cdot \Direction ~ \text{is maximum}
	\end{equation*}

	\vspace{5pt}
	\begin{figure}
		\begin{tikzpicture}
			\visible<2->{ % small
				\node[inner sep=0pt] (copepod) at (0,0) {
					\def\svgwidth{0.35\textwidth}
					\input{parts/formulation/schemes/navigation_problem.pdf_tex}
				};
			}

			\visible<3->{ % small
				\node[inner sep=0pt] (copepod) at (5.5,0) {
					\large
					\textbf{Performance:} $\displaystyle \Performance = \frac{\ParticlePosition(T) \cdot \Direction}{T}$
				};
			}
		\end{tikzpicture}
		\captionsetup{skip=10pt, margin={0.0\textwidth, 0.0\textwidth}}
		\caption{
			Navigation problem: maximizing vertical displacement.
		}
	\end{figure}
\end{frame}

% \begin{frame}{2 Formulation of the \textbf{navigation problem}}{1 Introduction - 3 Surfing - 4 Performance - 5 Relevance - 6 Perspectives}
	% \centering
	% \vspace{10pt}
	% \textbf{\Large Navigation baseline}
% 
	% \vspace{20pt}
	% \begin{figure}
		% \begin{tikzpicture}
			% \visible<1>{ % plankter
				% \node[inner sep=0pt] (copepod) at (0,0) {
					% \def\svgwidth{0.4\textwidth}
					% \input{parts/formulation/schemes/plankter_bh_0.pdf_tex}
				% };
			% }
			% \visible<2>{ % straight
				% \node[inner sep=0pt] (copepod) at (0,0) {
					% \def\svgwidth{0.4\textwidth}
					% \input{parts/formulation/schemes/plankter_bh_1.pdf_tex}
				% };
			% }
			% \visible<3>{ % reorientation time
				% \node[inner sep=0pt] (copepod) at (0,0) {
					% \def\svgwidth{0.4\textwidth}
					% \input{parts/formulation/schemes/plankter_bh_2.pdf_tex}
				% };
			% }
			% \visible<4->{ % instantaneous reorientation
				% \node[inner sep=0pt] (copepod) at (0,0) {
					% \def\svgwidth{0.4\textwidth}
					% \input{parts/formulation/schemes/plankter_bh_3.pdf_tex}
				% };
			% }
% 
			% \node[inner sep=0pt] (copepod) at (6.5,0.0) {
				% \parbox{0.6\textwidth}{
					% \large
					% \begin{itemize}
						% \setlength\itemsep{10pt}
						% \item<2-> no flow sensing: swim \textbf{straight} upwards
						% \item<3-> solution: \textbf{bottom-heavyness}
						% \item<4-> instantaneous reorientation
					% \end{itemize}
				% }
			% };
			% \node[opacity=0.0] (copepod) at (10.5,0) {phantom};
		% \end{tikzpicture}
		% \captionsetup{skip=10pt, margin={0.0\textwidth, 0.5\textwidth}}
		% \caption{
			% Illustration of our navigation baseline: the \textbf{bottom-heavy} strategy.
		% }
	% \end{figure}
% \end{frame}

%-------------------------------------------------------
% FORMULATION: PRESENTATION PLAN
%-------------------------------------------------------

\begin{frame}{3 Solution: The \textbf{surfing} strategy}{1 Introduction - 2 Problem - 4 Performance - 5 Relevance - 6 Perspectives}
	\centering
	\vspace{15pt}
	\textbf{\Large Outline}

	\vspace{15pt}

	\large
	\begin{enumerate}
		\setlength\itemsep{10pt}
		\item \textcolor{gray}{\textbf{Introduction} to the world of plankton}
		\item \textcolor{gray}{Formulation of the \textbf{navigation problem}}
		\item \textcolor{white}{Solution: The \textbf{surfing} strategy}
		\item \textcolor{gray}{Evaluation of surfing \textbf{performance}}
		\item \textcolor{gray}{Biophysical \textbf{relevance} of the strategy}
		\item \textcolor{gray}{Summary and \textbf{perspectives}}
	\end{enumerate}

\end{frame}

%-------------------------------------------------------
% SURFING STRATEGY: DERIVATION
%-------------------------------------------------------

% \begin{frame}{3 Solution: The \textbf{surfing} strategy}{1 Introduction - 2 Problem - 4 Performance - 5 Relevance - 6 Perspectives}
	% \centering
	% \vspace{2.5pt}
% %-------------------------------------------------------
	% \pause
	% \begin{equation}
		% \label{eq:motion}
		% \text{Motion: } ~~~ \frac{d \ParticlePosition}{dt} = \FlowVelocity (\ParticlePosition, t) + \SwimmingVelocity \, \ControlDirection
	% \end{equation}
	% \pause
	% \begin{equation*}
		% \label{eq:linear}
		% \text{Linearization: } ~~~ \FlowVelocity (\vec{x}, \TimeHorizon) \approx \FlowVelocity_0 + (\Gradients)_0 \, \left(\vec{x}  - \ParticlePosition_0 \right) + \, \left(\frac{\partial \FlowVelocity}{\partial t} \right)_0 \left( \TimeHorizon - t_0 \right)
	% \end{equation*}
	% \pause
	% \begin{equation*}
		% \label{eq:optimal_swimming_direction}
		% \text{Solution: } ~~~
		% \boxed{
			% \ControlDirectionOpt = \frac{\ControlDirectionOptNN}{\norm{\ControlDirectionOptNN}}, \quad \text{with} \quad \ControlDirectionOptNN = \left[ \exp \left( \TimeHorizon \Gradients \right) \right]^T \, \Direction.
		% }
	% \end{equation*}
% 
	% \pause
	% \vspace{3pt}
	% \begin{itemize}
		% \item $\TimeHorizon$ duration when the linearization starts to break down
		% \pause
		% \item $\TimeHorizonOpt$ linked to \textbf{correlation times}
	% \end{itemize}
	% \pause
	% \vspace{3pt}
	% \begin{equation}
		% \ControlDirectionOpt \underset{\TimeHorizon \to 0}{\longrightarrow} \Direction \pause = \ControlDirection_{\mathrm{b-h}}
	% \end{equation}
% 
	% \vskip0pt plus 1filll
% 
	% \pause
	% \begin{itemize}
		% \item $\ControlDirectionOpt (t)$ : local optimum over a time $\TimeHorizon$, \textbf{only free parameter}
	% \end{itemize}
	% \vspace{3pt}
% \end{frame}

\begin{frame}{3 Solution: The \textbf{surfing} strategy}{1 Introduction - 2 Problem - 4 Performance - 5 Relevance - 6 Perspectives}
	\vspace{-20pt}
	\centering
	\begin{equation*}
		\large
		\text{find} ~ \ControlDirection ~ \text{such that} ~ \ParticlePosition(\FinalTime) \cdot \Direction ~ \text{is maximum}
	\end{equation*}

	\vspace{-5pt}
	\begin{figure}
		\begin{tikzpicture}
			\visible<1-2>{ % small
				\node[inner sep=0pt] (copepod) at (0,0) {
					\def\svgwidth{0.35\textwidth}
					\input{parts/surf/schemes/navigation_problem_0.pdf_tex}
				};
			}

			\visible<3->{ % small
				\node[inner sep=0pt] (copepod) at (0,0) {
					\def\svgwidth{0.35\textwidth}
					\input{parts/surf/schemes/navigation_problem_1.pdf_tex}
				};
			}
		\end{tikzpicture}
		% \captionsetup{skip=10pt, margin={0.0\textwidth, 0.0\textwidth}}
		% \caption{
			% Navigation problem: maximizing vertical displacement.
		% }
	\end{figure}
	\pause
	\vspace{0pt}
	\begin{equation*}
		\label{eq:motion}
		\frac{d \ParticlePosition}{dt} = \FlowVelocity (\ParticlePosition, t) + \SwimmingVelocity \, \ControlDirection
	\end{equation*}
	\pause
	\begin{equation*}
		\label{eq:linear}
		\FlowVelocity (\vec{x}, t) \approx \FlowVelocity_0 + (\Gradients) \, \vec{x} + \, \left(\frac{\partial \FlowVelocity}{\partial t} \right) t
	\end{equation*}	
\end{frame}

\begin{frame}{3 Solution: The \textbf{surfing} strategy}{1 Introduction - 2 Problem - 4 Turbulence - 5 Relevance - 6 Perspectives}
	\centering
	\vspace{-15pt}
	\begin{multline*}
		\ParticlePosition(\TimeHorizon) \cdot \Direction =
		\left[ \exp \left[ \TimeHorizon \Gradients \right] - \matr{I} \right] (\Gradients)^{-1} \, \left[ \FlowVelocity \, + (\Gradients)^{-1} \left(\frac{\partial \FlowVelocity}{\partial t} \right) \, \right] \cdot \Direction \\
		- \, \TimeHorizon (\Gradients)^{-1} \left(\frac{\partial \FlowVelocity}{\partial t} \right) \cdot \Direction
		+ \, \SwimmingVelocity \int_{0}^{\TimeHorizon} \left( \exp \left[ (\TimeHorizon - t) \Gradients \right] \ControlDirection(t) \right) \cdot \Direction \, dt \,
	\end{multline*}
	\vspace{2pt}
	\pause
	\begin{equation*}
		\text{find} ~ \ControlDirection ~ \text{such that} ~ \int_{0}^{\TimeHorizon} \left( \exp \left[ (\TimeHorizon - t) \Gradients \right] \ControlDirection(t) \right) \cdot \Direction \, dt ~ \text{is maximum}
	\end{equation*}
	\pause
	\begin{equation*}
		\text{find} ~ \ControlDirection ~ \text{such that} ~ \left( \exp \left[ (\TimeHorizon - t) \Gradients \right] \ControlDirection(t) \right) \cdot \Direction ~ \text{is maximum}
	\end{equation*}
	\pause
	\begin{equation*}
		\text{$t=0$: find} ~ \ControlDirection ~ \text{such that} ~ \left( \exp \left[ \TimeHorizon \Gradients \right] \ControlDirection(t) \right) \cdot \Direction ~ \text{is maximum}
	\end{equation*}
	\pause
	\begin{equation*}
		\text{find} ~ \ControlDirection ~ \text{such that} ~ \left( \exp \left[ \TimeHorizon (\Gradients)^\top \right] \Direction \right) \cdot \ControlDirection ~ \text{is maximum}
	\end{equation*}
	\pause
	\begin{equation*}
		\ControlDirectionOpt = \frac{\ControlDirectionOptNN}{\norm{\ControlDirectionOptNN}}, \quad \text{with} \quad \ControlDirectionOptNN = \exp \left[ \TimeHorizon (\Gradients)^\top \right] \, \Direction.
	\end{equation*}
\end{frame}


\begin{frame}{3 Solution: The \textbf{surfing} strategy}{1 Introduction - 2 Problem - 4 Turbulence - 5 Relevance - 6 Perspectives}
	\centering
	\vspace{20pt}
	\begin{equation*}
		\label{eq:optimal_swimming_direction}
		\ControlDirectionOpt = \frac{\ControlDirectionOptNN}{\norm{\ControlDirectionOptNN}}, \quad \text{with} \quad \ControlDirectionOptNN = \exp \left[ \TimeHorizon (\Gradients)^\top \right] \, \Direction.
	\end{equation*}

	\vskip0pt plus 1filll
	
	\begin{itemize}
		\large
		\setlength\itemsep{10pt}
		\item<1-> local optimization over a time horizon $\TimeHorizon$
		\item<2-> main result of this study
		\item<3-> $\TimeHorizon$ is the \textbf{only} free parameter
		\item<4-> \textbf{brute force} approach
	\end{itemize}
	\vspace{25pt}
\end{frame}

% %-------------------------------------------------------
% % SURFING STRATEGY: Taylor-Green Vortices: Performance
% %-------------------------------------------------------
% 
% \begin{frame}{3 Solution: The \textbf{surfing} strategy}{1 Introduction - 2 Problem - 4 Performance - 5 Relevance - 6 Perspectives}
	% \vspace{5pt}
	% \centering
% 
	% %\tikzexternalenable
\begin{tikzpicture}
	\begin{groupplot}[
		group style={
			group size=1 by 1,
			y descriptions at=edge left,
			horizontal sep=0.06\linewidth,
		},
		% size
		width=0.65\textwidth,
		% y
		ymin=0,
		ymax=2.5,
		ylabel={$\left\langle \Performance \right\rangle / \SwimmingVelocity$},
		% x
		xlabel=$\TimeHorizon \FlowVorticityScalar_{\mathrm{max}}$,
		x label style={yshift=4pt},
		xmin=0,
		xmax=2.0001*pi,
		xtick={0, pi/2.0, pi, 3*pi/2.0, 2.0*pi},
		xticklabels={0,$\pi/2$,$\pi$,$3\pi/2$,$2\pi$},
		% layers
		set layers,
		% legend
		legend style={
			draw=none, 
			fill=none, 
			/tikz/every even column/.append style={column sep=4pt}, 
			at={(-0.2, 1.05)},
			anchor=south west
		},
   		legend cell align=left,
   		legend columns=-1,
	]
		\nextgroupplot[
			ylabel={$\left\langle \Performance \right\rangle_{N,\Direction} / \SwimmingVelocity$},
			ylabel style={yshift=-0.02\textwidth},
		]
			\addlegendimage{empty legend}\addlegendentry{$\SwimmingVelocity =$}
			%% us 0.25
			%%% 95 CI
			\addplot[name path=A, draw=none, forget plot] table [
				x expr={\thisrowno{0} * 2 * (1.0 + 0.0)},
				y expr={(\thisrowno{1} - \thisrowno{2}) / (0.0 + 0.25)},
				col sep=comma,
				comment chars=\#,
				unbounded coords=discard,
			]{parts/surf/data/flow_taylor_green_vortex__angle_0o0__l_1o0__T_200.0__swimming_speed_0.25__surfers__average_effective_velocity.csv};
			\addplot[name path=B, draw=none, forget plot] table [
				x expr={\thisrowno{0} * 2 * (1.0 + 0.0)},
				y expr={(\thisrowno{1} + \thisrowno{2}) / (0.0 + 0.25)},
				col sep=comma,
				comment chars=\#,
				unbounded coords=discard,
			]{parts/surf/data/flow_taylor_green_vortex__angle_0o0__l_1o0__T_200.0__swimming_speed_0.25__surfers__average_effective_velocity.csv};
			\addplot[ColorSurf!100!ColorVs, opacity=0.4, forget plot, on layer=axis background, visible on=<3->,] fill between[of=A and B];
			%%% average
			\addplot
			[
			color=ColorSurf!100!ColorVs,
			only marks,%solid
			mark=pentagon,
			visible on=<3->
			]
			table[
				x expr={\thisrowno{0} * 2 * (1.0 + 0.0)},
				y expr={(\thisrowno{1}) / (0.0 + 0.25)},
				col sep=comma,
				comment chars=\#,
				unbounded coords=discard,
			]{parts/surf/data/flow_taylor_green_vortex__angle_0o0__l_1o0__T_200.0__swimming_speed_0.25__surfers__average_effective_velocity.csv};
			\addlegendentry{$\FlowVelocityScalar_{\mathrm{max}}/4$}
			%% us 0.5
			%%% 95 CI
			\addplot[name path=A, draw=none, forget plot] table [
				x expr={\thisrowno{0} * 2 * (1.0 + 0.0)},
				y expr={(\thisrowno{1} - \thisrowno{2}) / (0.0 + 0.5)},
				col sep=comma, 
				comment chars=\#,
				unbounded coords=discard,
			]{parts/surf/data/flow_taylor_green_vortex__angle_0o0__l_1o0__T_200.0__swimming_speed_0.5__surfers__average_effective_velocity.csv};
			\addplot[name path=B, draw=none, forget plot] table [
				x expr={\thisrowno{0} * 2 * (1.0 + 0.0)},
				y expr={(\thisrowno{1} + \thisrowno{2}) / (0.0 + 0.5)},
				col sep=comma,
				comment chars=\#,
				unbounded coords=discard,
			]{parts/surf/data/flow_taylor_green_vortex__angle_0o0__l_1o0__T_200.0__swimming_speed_0.5__surfers__average_effective_velocity.csv};
			\addplot[ColorSurf!66!ColorVs, opacity=0.4, forget plot, on layer=axis background, visible on=<2->] fill between[of=A and B];
			%%% average
			\addplot
			[
			color=ColorSurf!66!ColorVs,
			only marks,%solid
			mark=square*,
			visible on=<2->
			]
			table[
				x expr={\thisrowno{0} * 2 * (1.0 + 0.0)},
				y expr={(\thisrowno{1}) / (0.0 + 0.5)},
				col sep=comma, 
				comment chars=\#,
				unbounded coords=discard,
			]{parts/surf/data/flow_taylor_green_vortex__angle_0o0__l_1o0__T_200.0__swimming_speed_0.5__surfers__average_effective_velocity.csv};
			\addlegendentry{$\FlowVelocityScalar_{\mathrm{max}}/2$}
			%% us 1
			%%% 95 CI
			\addplot[name path=A, draw=none, forget plot] table [
				x expr={\thisrowno{0} * 2 * (1.0 + 0.0)},
				y expr={(\thisrowno{1} - \thisrowno{2}) / (0.0 + 1.0)},
				col sep=comma,
				comment chars=\#,
				unbounded coords=discard,
			]{parts/surf/data/flow_taylor_green_vortex__angle_0o0__l_1o0__T_200.0__swimming_speed_1.0__surfers__average_effective_velocity.csv};
			\addplot[name path=B, draw=none, forget plot] table [
				x expr={\thisrowno{0} * 2 * (1.0 + 0.0)},
				y expr={(\thisrowno{1} + \thisrowno{2}) / (0.0 + 1.0)},
				col sep=comma,
				comment chars=\#,
				unbounded coords=discard,
			]{parts/surf/data/flow_taylor_green_vortex__angle_0o0__l_1o0__T_200.0__swimming_speed_1.0__surfers__average_effective_velocity.csv};
			\addplot[ColorSurf!33!ColorVs, opacity=0.4, forget plot, on layer=axis background, visible on=<4->] fill between[of=A and B];
			%%% average
			\addplot
			[
			color=ColorSurf!33!ColorVs,
			only marks,%solid
			mark=o,
			visible on=<4->
			]
			table[
				x expr={\thisrowno{0} * 2 * (1.0 + 0.0)},
				y expr={(\thisrowno{1}) / (0.0 + 1.0)},
				col sep=comma,
				comment chars=\#,
				unbounded coords=discard,
			]{parts/surf/data/flow_taylor_green_vortex__angle_0o0__l_1o0__T_200.0__swimming_speed_1.0__surfers__average_effective_velocity.csv};
			\addlegendentry{$\FlowVelocityScalar_{\mathrm{max}}$}
			%% us 2
			%%% 95 CI
			\addplot[name path=A, draw=none, forget plot] table [
				x expr={\thisrowno{0} * 2 * (1.0 + 0.0)},
				y expr={(\thisrowno{1} - \thisrowno{2}) / (0.0 + 2.0)},
				col sep=comma,
				comment chars=\#,
				unbounded coords=discard,
			]{parts/surf/data/flow_taylor_green_vortex__angle_0o0__l_1o0__T_200.0__swimming_speed_2.0__surfers__average_effective_velocity.csv};
			\addplot[name path=B, draw=none, forget plot] table [
				x expr={\thisrowno{0} * 2 * (1.0 + 0.0)},
				y expr={(\thisrowno{1} + \thisrowno{2}) / (0.0 + 2.0)},
				col sep=comma,
				comment chars=\#,
				unbounded coords=discard,
			]{parts/surf/data/flow_taylor_green_vortex__angle_0o0__l_1o0__T_200.0__swimming_speed_2.0__surfers__average_effective_velocity.csv};
			\addplot[ColorSurf!00!ColorVs, opacity=0.4, forget plot, on layer=axis background, visible on=<4->] fill between[of=A and B];
			%%% average
			\addplot
			[
			color=ColorSurf!00!ColorVs,
			only marks,%solid
			mark=*,
			visible on=<4->
			]
			table[
				x expr={\thisrowno{0} * 2 * (1.0 + 0.0)},
				y expr={(\thisrowno{1}) / (0.0 + 2.0)},
				col sep=comma,
				comment chars=\#,
				unbounded coords=discard,
			]{parts/surf/data/flow_taylor_green_vortex__angle_0o0__l_1o0__T_200.0__swimming_speed_2.0__surfers__average_effective_velocity.csv};
			\addlegendentry{$2\FlowVelocityScalar_{\mathrm{max}}$}
			%% y = x
			\addplot
			[
			color=gray!50!white,
			opacity=1.0,
			%line width=1pt, 
			solid, 
			on layer=axis background,
			domain=0:2*pi,
			]{1};
	\end{groupplot}
\end{tikzpicture}
%\tikzexternaldisable

% 
	% \vskip0pt plus 1filll
	% 
	% %\large
	% \begin{itemize}
		% \item<5-> \textbf{surfing is beneficial}, performance decreases with $\SwimmingVelocity / \KolmogorovVelocityScale$
	% \end{itemize}
	% \vspace{15pt}
% \end{frame}

%-------------------------------------------------------
% FORMULATION: PRESENTATION PLAN
%-------------------------------------------------------

\begin{frame}{4 Evaluation of surfing \textbf{performance}}{1 Introduction - 2 Problem - 3 Surfing - 5 Relevance - 6 Perspectives}
	\centering
	\vspace{15pt}
	\textbf{\Large Outline}

	\vspace{15pt}

	\large
	\begin{enumerate}
		\setlength\itemsep{10pt}
		\item \textcolor{gray}{\textbf{Introduction} to the world of plankton}
		\item \textcolor{gray}{Formulation of the \textbf{navigation problem}}
		\item \textcolor{gray}{Solution: The \textbf{surfing} strategy}
		\item \textcolor{white}{Evaluation of surfing \textbf{performance}}
		\item \textcolor{gray}{Biophysical \textbf{relevance} of the strategy}
		\item \textcolor{gray}{Summary and \textbf{perspectives}}
	\end{enumerate}

\end{frame}

%-------------------------------------------------------
% SURFING STRATEGY: Taylor-Green Vortices: Illustration
%-------------------------------------------------------

\begin{frame}{4 Evaluation of surfing \textbf{performance}}{1 Introduction - 2 Problem - 3 Surfing - 5 Relevance - 6 Perspectives}
	\centering
	\vspace{0pt}
	\begin{figure}
		\begin{tikzpicture}
			\node<1>[inner sep=0pt] (copepod) at (0,0) {
				\tikzexternalenable
\begin{tikzpicture}[
		arrow/.style={
			insert path={
				coordinate[pos=#1,sloped]  (aux-1)
				coordinate[pos=#1+\pgfkeysvalueof{/tikz/ga/length},sloped] (aux-2)
				(aux-1) edge[/tikz/ga/arrow] 
				(aux-2) %node[] {\scriptsize #1}
			}
		},
		marrow/.style={
			insert path={
				coordinate[pos=#1,sloped]  (aux-1)
				coordinate[pos=#1-\pgfkeysvalueof{/tikz/ga/length},sloped] (aux-2)
				(aux-1) edge[/tikz/ga/arrow] 
				(aux-2) %node[] {\scriptsize #1}
			}
		},
		ga/.cd,
		length/.initial=0.0001,
		arrow/.style={-stealth,white!50!black,solid},
		marrow/.style={-stealth,white!50!black,solid},
		]
	% plot
	\begin{axis}[
		axis line style={black},
		%width=0.85\linewidth,
		axis equal image,
		view={0}{90},
		% x
		xmin=-3*pi/2,
		xmax=3*pi/2,
		%xlabel=$x$,
		xticklabel=\empty,
		% y
		ymin=-pi/2,
		ymax=5*pi/2,
		%ylabel=$y$,
		yticklabel=\empty,
		% ticks
		tickwidth=0,
		% legend
		legend style={
			draw=none, 
			fill=none, 
			/tikz/every even column/.append style={column sep=4pt}, 
			at={(0.5, 1.025)}, 
			anchor=south,
			opacity=0.0,
		},
   		legend cell align=left,
   		legend columns=-1,
	]
		\addlegendimage{ColorBh,mark=*,mark options={mark indices={3}}}
		\addlegendentry{\NameBh}
		\addlegendimage{ColorSurf,mark=square*,mark options={mark indices={3}}}
		\addlegendentry{\NameSurf}
		% flow
		\addplot3 [
			thick,
			domain=-3*pi/2:3*pi/2,
			domain y=-pi/2:5*pi/2,
			samples=50,
			contour gnuplot={levels={-0.8, -0.6, -0.4, -0.2, 0.2, 0.4, 0.6, 0.8}, labels=false, draw color=white!50!black},
			forget plot,
		] {cos(deg(x)) * cos(deg(y))}
		[arrow/.list={0.46,0.48,0.5,0.52,0.54,0.56,0.58,0.6,0.62,0.64,0.66,0.68,0.7,0.72,0.74,0.76,0.78,0.8,0.82,0.84,0.86,0.88,0.90,0.92,0.94,0.96,0.98,1.0}] [marrow/.list={0.0,0.02,0.04,0.06,0.08,0.1,0.12,0.14,0.16,0.18,0.2,0.22,0.24,0.26,0.28,0.3,0.32,0.34,0.36,0.38,0.4,0.42,0.44}];
		% start position
		%\addplot[mark=*, mark size=1.2mm] coordinates {(0,0)};
		% end axis
	\end{axis}

	% % % plot
	% \begin{axis}[
		% at={(0.5\linewidth, 0)},
		% axis line style={black},
		% width=0.85\linewidth,
		% axis equal image,
		% view={0}{90},
		% % x
		% xmin=-pi/2,
		% xmax=4*pi/2,
		% %xlabel=$x$,
		% xticklabel=\empty,
		% % y
		% ymin=-1.0,
		% ymax=10.0,
		% %ylabel=$y$,
		% yticklabel=\empty,
		% % ticks
		% tickwidth=0,
	% ]
		% % flow
		% \addplot3 [
			% thick,
			% domain=-pi/2:4*pi/2,
			% domain y=-1.0:10.0,
			% samples=50,
			% contour gnuplot={levels={-0.8, -0.6, -0.4, -0.2, 0.2, 0.4, 0.6, 0.8}, labels=false, draw color=white!50!black},
			% forget plot,
		% ] {cos(deg(x)) * cos(deg(y))}
		% [arrow/.list={0.28,0.42,0.44,0.52,0.54,0.56,0.58,0.62,0.64,0.66,0.68,0.7,0.8,0.72,0.76,0.78,0.82,0.84,0.86,0.88,0.90,0.92,0.94,0.96,0.98,1.0}] [marrow/.list={0.0,0.02,0.04,0.06,0.08,0.1,0.12,0.14,0.16,0.18,0.2,0.22,0.24,0.26,0.3,0.32,0.34,0.36,0.38,0.4,0.46,0.48,0.5,0.6,0.74}];
		% % start position
		% \addplot[mark=*, mark size=1.2mm] coordinates {(-pi/4,0)};
		% % end axis
	% \end{axis}
\end{tikzpicture}
\tikzexternaldisable

			};
			\node<2>[inner sep=0pt] (copepod) at (0,0) {
				\movie[width=0.5328\textwidth, height=0.6\textwidth, autostart, poster, loop]{}{parts/surf/plots/main_bh_low.mp4}
			};
			\node<3>[inner sep=0pt] (copepod) at (0,0) {
				\movie[width=0.5328\textwidth, height=0.6\textwidth, autostart, poster, loop]{}{parts/surf/plots/main_low.mp4}
			};
		\end{tikzpicture}
		\captionsetup{skip=5pt, margin={0.0\textwidth, 0.0\textwidth}}
		\caption{
			Illustration of surfing in a 2D Taylor-Green flow.
		}
	\end{figure}
\end{frame}

%-------------------------------------------------------
% TURBULENCE: NUMERICAL SIMULATIONS
%-------------------------------------------------------

\begin{frame}{4 Evaluation of surfing \textbf{performance}}{1 Introduction - 2 Problem - 3 Surfing - 5 Relevance - 6 Perspectives}
	\centering
	\visible<1->{
		Online open-access simulation: \textbf{Johns Hopkins turbulence database}
	}

	% $Re_{\lambda} \approx 418$, $L / \eta \approx 710$, $T_L / \tau_\eta \approx 47$, $u_{rms} / u_{\eta} \approx 10$, $t_{max} \approx 5 T_L$

	\vspace{2pt}
	\begin{figure}
		\begin{tikzpicture}
			\only<1->{
				\node[inner sep=0pt] (copepod) at (0,0) {
					%% \movie[width=0.720\textwidth, height=1.280\textwidth, autostart, poster, loop]{}{parts/turbulence/videos/jhtdb_vertical_velocity.mp4}
					\movie[width=0.24\textwidth, height=0.43\textwidth, autostart, poster, loop]{}{parts/turbulence/videos/jhtdb_vertical_velocity.mp4}
				};
			}
			\only<2->{
				\node[inner sep=0pt] (copepod) at (5,0) {
					\movie[width=0.24\textwidth, height=0.43\textwidth, autostart, poster, loop]{}{parts/turbulence/videos/jhtdb_vorticity.mp4}
				};
			}
			\node[opacity=0.0] (phantom) at (5.7,0) {phantom};
		\end{tikzpicture}
		\captionsetup{skip=6pt, margin={0.0\textwidth, 0.0\textwidth}}
		\caption{
			\visible<1->{Quasi-stationary \textbf{Homogeneous Isotropic Turbulence} ($Re_{\lambda} \approx 418$). Vertical flow velocity: \textcolor{ColorSurf}{red positive}, \textcolor{ColorBh}{blue negative}.}\only<2->{Vorticity: \textcolor{ColorSurf}{red positive}, \textcolor{ColorBh}{blue negative} and \textcolor{yellow!80!black}{yellow null}.} Rendering by Gudmundur Adalsteinsson.
		}
	\end{figure}

	\vspace{-10pt}
	\begin{itemize}
		\large
		\item<3-> Kolmogorov scales: $\eta$, $\tau_\eta$ and $u_{\eta}$ used to \textbf{scale our results}
	\end{itemize}
\end{frame}

%-------------------------------------------------------
% TURBULENCE: SHELD0N PRINCIPLE
%-------------------------------------------------------

\begin{frame}{4 Evaluation of surfing \textbf{performance}}{1 Introduction - 2 Problem - 3 Surfing - 5 Relevance - 6 Perspectives}
	\centering
	\vspace{10pt}
	{\large Trajectory integration: \textbf{Sheld0n} \\ \textit{https://github.com/C0PEP0D/sheld0n}}
	
	\vspace{10pt}
	\leavevmode\hidewidth\begin{figure}
		\begin{tikzpicture}
			\visible<1->{
				\node[inner sep=0pt] (trajs) at (0,0) {
					\def\svgwidth{0.5\textwidth}
					\input{parts/turbulence/schemes/code_0.pdf_tex}
				};
			}
			\visible<2->{
				\node[inner sep=0pt] (trajs) at (0,0) {
					\def\svgwidth{0.5\textwidth}
					\input{parts/turbulence/schemes/code_1.pdf_tex}
				};
			}
			\visible<3->{
				\node[inner sep=0pt] (trajs) at (0,0) {
					\def\svgwidth{0.5\textwidth}
					\input{parts/turbulence/schemes/code_2.pdf_tex}
				};
			}
			\visible<4->{
				\node[inner sep=0pt] (trajs) at (0,0) {
					\def\svgwidth{0.5\textwidth}
					\input{parts/turbulence/schemes/code_3.pdf_tex}
				};
			}
			\node[inner sep=0pt] (copepod) at (6,0) {
				\parbox{0.5\textwidth}{
					\large
					\begin{itemize}
						\setlength\itemsep{5pt}
						\normalsize
						\item<1-> open-source solver, modern c++(17), 20000+ lines of code
						\item<1-> \textbf{fl0p}(interpolated flow), \textbf{fl0w}(flow abstraction), \textbf{l0ad}(vtk/txt/binary loader), \textbf{m0sh}(mesh utility), \textbf{p0l}(Laplace interpolator), ...
						\item<5-> 4$^{\text{th}}$ order Runge-Kutta
						\item<6-> own local flow database
					\end{itemize}
				}
			};
		\end{tikzpicture}
		\captionsetup{skip=6pt, margin={0.0\textwidth, 0.4\textwidth}}
		\caption{
			General principal of the solver.
		}
	\end{figure}\hidewidth\null
\end{frame}

%-------------------------------------------------------
% TURBULENCE: SHELD0N TRAJECTORIES
%-------------------------------------------------------

\begin{frame}{4 Evaluation of surfing \textbf{performance}}{1 Introduction - 2 Problem - 3 Surfing - 5 Relevance - 6 Perspectives}
	\centering
	\vspace{5pt}
	{\large \textbf{Visualisation of trajectories}}
	
	\centering
	\vspace{-25pt}
	\leavevmode\hidewidth\begin{figure}
		\begin{tikzpicture}
			\node<1>[inner sep=0pt] (trajs) at (0,0) {\movie[width=\textwidth, height=0.5\textwidth, autostart, poster, loop]{}{parts/turbulence/videos/trajectories_0.mp4}};
			\node<2>[inner sep=0pt] (trajs) at (0,0) {\movie[width=\textwidth, height=0.5\textwidth, autostart, poster, loop]{}{parts/turbulence/videos/trajectories_1.mp4}};
			\node<3->[inner sep=0pt] (trajs) at (0,0) {\movie[width=\textwidth, height=0.5\textwidth, autostart, poster, loop]{}{parts/turbulence/videos/trajectories_2.mp4}};
			\draw<1->[-stealth] (-5.5,0) -- (-5.5,1) node[midway, anchor=east]{$\Direction$};
			%\node<2->[opacity=0.0] (copepod) at (9.0,0) {phantom};
			\node<1->[inner sep=0pt, ColorPassive] (passives) at (-3.5,-3) {passive};
			\node<2->[inner sep=0pt, ColorBh] (risers) at (0,-3) {bottom-heavy};
			\node<3->[inner sep=0pt, ColorSurf] (surfers) at (3.5,-3) {surfers ($\TimeHorizon = 2 \KolmogorovTimeScale$)};
		\end{tikzpicture}
		\captionsetup{skip=8pt, margin={-0.2\textwidth, -0.2\textwidth}}
		\caption{
			Visualisation of trajectories in a turbulent flow ($\mathit{Re}_{\lambda} \approx 11$).
		}
	\end{figure}\hidewidth\null
\end{frame}

%-------------------------------------------------------
% TURBULENCE: RESULTS
%-------------------------------------------------------

\begin{frame}{4 Evaluation of surfing \textbf{performance}}{1 Introduction - 2 Problem - 3 Surfing - 5 Relevance - 6 Perspectives}
	\centering
	\vspace{-0pt}
	\begin{figure}
		%\tikzexternalenable
\begin{tikzpicture}
	% gain as a function of the free parameter $\tau$
	\begin{axis} [
		axis on top,
		% size
		width=0.6\textwidth,
		%height=0.45\textwidth,
		% y
		ymin=0,
		ymax=2.5,
		ylabel={$\left\langle \Performance \right\rangle / \SwimmingVelocity$},
		y label style={yshift=-4pt},
		extra y ticks={0.5, 1.5, 2.5},
		% x
		xlabel=$\TimeHorizon / \KolmogorovTimeScale$,
		x label style={yshift=4pt},
		xmin=0,
		xmax=10,
		% layers
		set layers,
		% legend
		legend style={
			draw=none,
			fill=none, 
			/tikz/every even column/.append style={column sep=4pt}, 
			at={(0.1, 1.05)},
			anchor=south west
		},
   		legend cell align=left,
   		legend columns=-1,
	]
		\addlegendimage{empty legend}\addlegendentry{$\SwimmingVelocity =$}
		% %% us 0.5
		% %%% 95 CI
		% \addplot[name path=A, draw=none, forget plot] table [
			% x index=3,
			% y expr={(\thisrowno{0} - \thisrowno{1}) / (\thisrowno{2} * 0.066)}, %u_\eta = 0.066
			% col sep=comma,
			% comment chars=\#,
			% restrict expr to domain={\thisrowno{2}}{0.5:0.5},
			% unbounded coords=discard,
		% ]{data/jhtdb/final_merge.csv};
		% \addplot[name path=B, draw=none, forget plot] table [
			% x index=3,
			% y expr={(\thisrowno{0} + \thisrowno{1}) / (\thisrowno{2} * 0.066)}, %u_\eta = 0.066
			% col sep=comma,
			% comment chars=\#,
			% restrict expr to domain={\thisrowno{2}}{0.5:0.5},
			% unbounded coords=discard,
		% ]{data/jhtdb/final_merge.csv};
		% \addplot[ColorSurf!100!ColorVs, opacity=0.4, forget plot, on layer=axis background, visible on=<4->] fill between[of=A and B];
		% %%% average
		% \addplot
		% [
		% color=ColorSurf!100!ColorVs,
		% opacity=1.0,
		% only marks,%solid
		% mark=pentagon,
		% visible on=<4->,
		% ]
		% table[
			% x index=3,
			% y expr={\thisrowno{0} / (\thisrowno{2} * 0.066)}, %u_\eta = 0.066
			% col sep=comma,
			% comment chars=\#,
			% restrict expr to domain={\thisrowno{2}}{0.5:0.5},
			% unbounded coords=discard,
		% ]{data/jhtdb/final_merge.csv};
		% \addlegendentry{$\KolmogorovVelocityScale/2$}
		%% us 1.0
		%%% 95 CI
		\addplot[name path=A, draw=none, forget plot] table [
			x index=3,
			y expr={(\thisrowno{0} - \thisrowno{1}) / (\thisrowno{2} * 0.066)}, %u_\eta = 0.066
			col sep=comma, 
			comment chars=\#,
			restrict expr to domain={\thisrowno{2}}{1.0:1.0},
			unbounded coords=discard,
		]{parts/turbulence/data/jhtdb/final_merge.csv};
		\addplot[name path=B, draw=none, forget plot] table [
			x index=3, 
			y expr={(\thisrowno{0} + \thisrowno{1}) / (\thisrowno{2} * 0.066)}, %u_\eta = 0.066
			col sep=comma, 
			comment chars=\#,
			restrict expr to domain={\thisrowno{2}}{1.0:1.0},
			unbounded coords=discard,
		]{parts/turbulence/data/jhtdb/final_merge.csv};
		\addplot[ColorSurf!100!ColorVs, opacity=0.4, forget plot, on layer=axis background, visible on=<2->,] fill between[of=A and B];
		%%% average
		\addplot
		[
		color=ColorSurf!100!ColorVs,
		opacity=1.0,
		only marks,%solid
		mark=square*,
		visible on=<2-3>,
		]
		table[
			x index=3, 
			y expr={\thisrowno{0} / (\thisrowno{2} * 0.066)}, %u_\eta = 0.066
			col sep=comma, 
			comment chars=\#,
			restrict expr to domain={\thisrowno{2}}{1.0:1.0},
			unbounded coords=discard,
		]{parts/turbulence/data/jhtdb/final_merge.csv};
		\addlegendentry{$\KolmogorovVelocityScale$}
		%%% max
		\addplot
		[
		color=ColorSurf!100!ColorVs,
		opacity=1.0,
		only marks,%solid
		mark=square*,
		visible on=<4>,
		forget plot,
		]
		table[
			x index=3, 
			y expr={\thisrowno{0} / (\thisrowno{2} * 0.066)}, %u_\eta = 0.066
			col sep=comma, 
			comment chars=\#,
			restrict expr to domain={\thisrowno{2}}{1.0:1.0},
			restrict expr to domain={\thisrowno{3}}{4.0:4.0},
			unbounded coords=discard,
		]{parts/turbulence/data/jhtdb/final_merge.csv};
		% %%% fit
		% \addplot
		% [
		% color=ColorSurf!100!ColorVs,
		% opacity=1.0,
		% solid,
		% forget plot,
		% visible on=<3->,
		% ]
		% table[
			% x index=0,
			% y expr={\thisrowno{1} / (1.0 * 0.066)}, %u_\eta = 0.066
			% col sep=comma,
			% comment chars=\#,
			% unbounded coords=discard,
		% ]{data/jhtdb_low/fits_average_velocity_axis_0__agent.csv};
		% %% us 4.0
		% %%% 95 CI
		% \addplot[name path=A, draw=none, forget plot] table [
			% x index=3,
			% y expr={(\thisrowno{0} - \thisrowno{1}) / (\thisrowno{2} * 0.066)}, %u_\eta = 0.066
			% col sep=comma,
			% comment chars=\#,
			% restrict expr to domain={\thisrowno{2}}{4.0:4.0},
			% unbounded coords=discard,
		% ]{data/jhtdb/final_merge.csv};
		% \addplot[name path=B, draw=none, forget plot] table [
			% x index=3,
			% y expr={(\thisrowno{0} + \thisrowno{1}) / (\thisrowno{2} * 0.066)}, %u_\eta = 0.066
			% col sep=comma,
			% comment chars=\#,
			% restrict expr to domain={\thisrowno{2}}{4.0:4.0},
			% unbounded coords=discard,
		% ]{data/jhtdb/final_merge.csv};
		% \addplot[ColorSurf!00!ColorVs, opacity=0.4, forget plot, on layer=axis background, visible on=<3->] fill between[of=A and B];
		% %%% average
		% \addplot
		% [
		% color=ColorSurf!00!ColorVs,
		% opacity=1.0,
		% only marks,%solid
		% mark=o,
		% visible on=<3->,
		% ]
		% table[
			% x index=3,
			% y expr={\thisrowno{0} / (\thisrowno{2} * 0.066)}, %u_\eta = 0.066
			% col sep=comma,
			% comment chars=\#,
			% restrict expr to domain={\thisrowno{2}}{4.0:4.0},
			% unbounded coords=discard,
		% ]{data/jhtdb/final_merge.csv};
		% \addlegendentry{$4 \KolmogorovVelocityScale$}
		% %%% fit
		% \addplot
		% [
		% color=ColorSurf!50!ColorVs,
		% opacity=1.0,
		% solid,
		% forget plot,
		% visible on=<4->,
		% ]
		% table[
			% x index=0,
			% y expr={\thisrowno{2} / (4.0 * 0.066)}, %u_\eta = 0.066
			% col sep=comma,
			% comment chars=\#,
			% unbounded coords=discard,
		% ]{data/jhtdb_more/fits_average_velocity_axis_0__agent.csv};
		%% us 8.0
		%%% 95 CI
		\addplot[name path=A, draw=none, forget plot] table [
			x index=3,
			y expr={(\thisrowno{0} - \thisrowno{1}) / (\thisrowno{2} * 0.066)}, %u_\eta = 0.066
			col sep=comma,
			comment chars=\#,
			restrict expr to domain={\thisrowno{2}}{8.0:8.0},
			unbounded coords=discard,
		]{parts/turbulence/data/jhtdb/final_merge.csv};
		\addplot[name path=B, draw=none, forget plot] table [
			x index=3,
			y expr={(\thisrowno{0} + \thisrowno{1}) / (\thisrowno{2} * 0.066)}, %u_\eta = 0.066
			col sep=comma,
			comment chars=\#,
			restrict expr to domain={\thisrowno{2}}{8.0:8.0},
			unbounded coords=discard,
		]{parts/turbulence/data/jhtdb/final_merge.csv};
		\addplot[ColorSurf!0!ColorVs, opacity=0.4, forget plot, on layer=axis background, visible on=<3->,] fill between[of=A and B];
		%%% average
		\addplot
		[
		color=ColorSurf!0!ColorVs,
		opacity=1.0,
		only marks,%solid
		mark=o,
		visible on=<3>,
		]
		table[
			x index=3,
			y expr={\thisrowno{0} / (\thisrowno{2} * 0.066)}, %u_\eta = 0.066
			col sep=comma,
			comment chars=\#,
			restrict expr to domain={\thisrowno{2}}{8.0:8.0},
			unbounded coords=discard,
		]{parts/turbulence/data/jhtdb/final_merge.csv};
		\addlegendentry{$8 \KolmogorovVelocityScale$}
		%% max
		\addplot
		[
		color=ColorSurf!0!ColorVs,
		opacity=1.0,
		only marks,%solid
		mark=o,
		visible on=<4>,
		forget plot
		]
		table[
			x index=3,
			y expr={\thisrowno{0} / (\thisrowno{2} * 0.066)}, %u_\eta = 0.066
			col sep=comma,
			comment chars=\#,
			restrict expr to domain={\thisrowno{2}}{8.0:8.0},
			restrict expr to domain={\thisrowno{3}}{3.0:3.0},
			unbounded coords=discard,
		]{parts/turbulence/data/jhtdb/final_merge.csv};
		% %%% fit
		% \addplot
		% [
		% color=ColorSurf!0!ColorVs,
		% opacity=1.0,
		% solid,
		% forget plot,
		% visible on=<6->,
		% ]
		% table[
			% x index=0,
			% y expr={\thisrowno{6} / (8.0 * 0.066)}, %u_\eta = 0.066
			% col sep=comma,
			% comment chars=\#,
			% unbounded coords=discard,
		% ]{data/jhtdb_even_more/fits_average_velocity_axis_0__agent.csv};
		%% y = x
        \addplot
        [
        color=white,
        opacity=1.0,
        solid, 
        on layer=axis background,
        domain=0:10,
        forget plot,
        ]{1};
	\end{axis}
\end{tikzpicture}
%\tikzexternaldisable

		\captionsetup{skip=2pt, margin={0.0\textwidth, -0.0\textwidth}}
		\caption{
			Surfing performance in turbulence ($\mathit{Re}_{\lambda} \approx 418$) as a function of the surfing parameter $\TimeHorizon$. Monthiller et al. (2022).
		}
	\end{figure}

	\vskip0pt plus 1filll

	\vspace{-5pt}
	\visible<2->{
		\begin{itemize}
			\item Surfing can \textbf{double} migration speed in turbulence!
		\end{itemize}
	}
	\vspace{10pt}
\end{frame}

\begin{frame}{4 Evaluation of surfing \textbf{performance}}{1 Introduction - 2 Problem - 3 Surfing - 5 Relevance - 6 Perspectives}
	\centering
	\vspace{-10pt}
	\begin{figure}
		%\tikzexternalenable
\begin{tikzpicture}
    % gain as a function of swimming velocity
    \begin{axis}[
        axis on top,
        % size
        width=0.6\linewidth,
        % y
        ylabel={$\left\langle \Performance \right\rangle / \SwimmingVelocity$},
        ymin=0.0,
        ymax=2.5,
        extra y ticks={2.5},
        y label style={yshift=-4pt},
        % x
        xlabel=$\SwimmingVelocity / \KolmogorovVelocityScale$,
        xmode=log,
        xticklabels={0.1,1,10},
        extra x ticks={0.5, 20},
        extra x tick labels={0.5,20},
        xmin=0.5,
        xmax=20,
        x label style={yshift=4pt},
        % layers
        set layers,
        % legend
        legend style={draw=none, fill=none, /tikz/every even column/.append style={column sep=8pt}},
        legend cell align={left},
        legend pos=south east,
        legend columns=-1,
    ]
        % tss
        \addplot[name path=A, draw=none, forget plot] table [
            x index=2, 
            y expr={(\thisrowno{0} - \thisrowno{1}) / (\thisrowno{2} * 0.066)}, %u_\eta = 0.066
            col sep=comma,
            comment chars=\#,
        ]{parts/turbulence/data/jhtdb/final_max.csv};
        \addplot[name path=B, draw=none, forget plot] table [
            x index=2, 
            y expr={(\thisrowno{0} + \thisrowno{1}) / (\thisrowno{2} * 0.066)}, %u_\eta = 0.066
            col sep=comma, 
            comment chars=\#,
        ]{parts/turbulence/data/jhtdb/final_max.csv};
        \addplot[ColorSurf, opacity=0.4, forget plot, on layer=axis background, visible on=<1->] fill between[of=A and B];
        \addplot
        [
        color=ColorSurf,
        opacity=1.0,
        only marks,%solid, 
        mark=square*,
        visible on=<1->
        ]
        table[
            x index=2,
            y expr={\thisrowno{0} / (\thisrowno{2} * 0.066)}, %u_\eta = 0.066
            col sep=comma, 
            comment chars=\#,
        ]{parts/turbulence/data/jhtdb/final_max.csv};
        %\addlegendentry{\NameSurf}
        % % straight
        % \addplot[name path=A, draw=none, forget plot] table [
            % x index=2,
            % y expr={(\thisrowno{0} - \thisrowno{1}) / (\thisrowno{2} * 0.066)}, %u_\eta = 0.066
            % col sep=comma,
            % comment chars=\#,
            % restrict expr to domain={\thisrowno{3}}{0.0:0.0},
            % unbounded coords=discard,
        % ]{data/jhtdb/final_merge.csv};
        % \addplot[name path=B, draw=none, forget plot] table [
            % x index=2,
            % y expr={(\thisrowno{0} + \thisrowno{1}) / (\thisrowno{2} * 0.066)}, %u_\eta = 0.066
            % col sep=comma,
            % comment chars=\#,
            % restrict expr to domain={\thisrowno{3}}{0.0:0.0},
            % unbounded coords=discard,
        % ]{data/jhtdb/final_merge.csv};
        % \addplot[ColorBh, opacity=0.4, forget plot, on layer=axis background, visible on=<1->] fill between[of=A and B];
        % \addplot
        % [
        % color=ColorBh,
        % opacity=1.0,
        % only marks,%solid,
        % mark=o,
        % visible on=<1->
        % ]
        % table[
            % x index=2,
            % y expr={\thisrowno{0} / (\thisrowno{2} * 0.066)}, %u_\eta = 0.066
            % col sep=comma,
            % comment chars=\#,
            % restrict expr to domain={\thisrowno{3}}{0.0:0.0},
            % unbounded coords=discard,
        % ]{data/jhtdb/final_merge.csv};
        % \addlegendentry{\NameBh}
        %% y = x
        \addplot
        [
        color=white,
        opacity=1.0,
        solid, 
        on layer=axis background,
        domain=0.5:20,
        ]{1};
        %% model
        %\def\modlambda{0.05}
        %\def\modomega{0.2}
        %\addplot
        %[
        %color=black,
        %opacity=1.0,
        %on layer=axis foreground,
        %dashed,
        %]
        %table[
        %    x index=0,
        %    y expr={e^(\modlambda * (0.11 * \thisrowno{1} / 0.0424)) / cos(deg(\modomega * (0.11 * \thisrowno{1} / 0.0424)))}, % \tau_\eta = 0.0424
        %    col sep=comma,
        %    comment chars=\#,
        %]{data/jhtdb/final_time.csv};
    \end{axis}
    \begin{axis}[
	    % size
	    width=0.6\linewidth,
	    % x
	    xmin=0.05,
	    xmax=2,
	    xlabel=$\SwimmingVelocity / u_{\mathrm{rms}}$,
	    xmode=log,
	    x label style={yshift=-4pt},
	    axis x line*=top,
	    xtickten={-2, -1, 0},
	    xticklabels={0.01,0.1,1},
	    extra x ticks={0.5, 2},
	    extra x tick labels={0.5,2},
	    % y
	    axis y line=none,
	    ymin=0,
	    ymax=2.5,
    ]
    \addplot
    [
	    color=white,
	    opacity=1.0,
	    solid, 
	    on layer=axis background,
	    domain=0.05:2,
    ]
    {1};
    \end{axis}
\end{tikzpicture}
%\tikzexternaldisable

		\captionsetup{skip=2pt, margin={0.0\textwidth, -0.0\textwidth}}
		\caption{
			Surfing performance in turbulence ($\mathit{Re}_{\lambda} \approx 418$) as a function of swimming velocity $\SwimmingVelocity / \KolmogorovVelocityScale$. Monthiller et al. (2022).
		}
	\end{figure}

	\vspace{-10pt}
	\visible<2->{
		\begin{itemize}
			\item Migration performance decrease with the ratio $\SwimmingVelocity / \KolmogorovVelocityScale$
		\end{itemize}
	}
\end{frame}

%-------------------------------------------------------
% TURBULENCE: REORIENTATION TIME
%-------------------------------------------------------

\begin{frame}{4 Evaluation of surfing \textbf{performance}}{1 Introduction - 2 Problem - 3 Surfing - 5 Relevance - 6 Perspectives}
	\centering
	\vspace{5pt}
	\textbf{\large Robustness: finite-time reorientation}
	
	\vspace{10pt}
	\begin{tikzpicture}
		\visible<2>{
			\node[inner sep=0pt] (trajs) at (0,0) {
				\def\svgwidth{0.75\textwidth}
				\input{parts/turbulence/schemes/reorientation_time_0.pdf_tex}
			};
		}
		\visible<3>{
			\node[inner sep=0pt] (trajs) at (0,0) {
				\def\svgwidth{0.75\textwidth}
				\input{parts/turbulence/schemes/reorientation_time_1.pdf_tex}
			};
		}
		\visible<4>{
			\node[inner sep=0pt] (trajs) at (0,0) {
				\def\svgwidth{0.75\textwidth}
				\input{parts/turbulence/schemes/reorientation_time_2.pdf_tex}
			};
		}
		\visible<5>{
			\node[inner sep=0pt] (trajs) at (0,0) {
				\def\svgwidth{0.75\textwidth}
				\input{parts/turbulence/schemes/reorientation_time_3.pdf_tex}
			};
		}
		\visible<6>{
			\node[inner sep=0pt] (trajs) at (0,0) {
				\def\svgwidth{0.75\textwidth}
				\input{parts/turbulence/schemes/reorientation_time_4.pdf_tex}
			};
		}
		\visible<7->{
			\node[inner sep=0pt] (trajs) at (0,0) {
				\def\svgwidth{0.75\textwidth}
				\input{parts/turbulence/schemes/reorientation_time_5.pdf_tex}
			};
		}
	\end{tikzpicture}

	\vspace{-0pt}
	\visible<7->{
		\begin{equation*}
			\frac{d \SwimmingDirection}{d t} = \frac{1}{2} \FlowVorticity (\ParticlePosition, t) \times \SwimmingDirection + \frac{1}{2 \ReorientationTime} \left[ \ControlDirection - (\ControlDirection \cdot \SwimmingDirection) \SwimmingDirection \right]
		\end{equation*}

		\vspace{-10pt}
		\scriptsize Pedley et al. (1992)
	}
\end{frame}

\begin{frame}{4 Evaluation of surfing \textbf{performance}}{1 Introduction - 2 Problem - 3 Surfing - 5 Relevance - 6 Perspectives}
	\centering
	\vspace{5pt}
	\centering
	\begin{figure}
		%\tikzexternalenable
\begin{tikzpicture}
    \begin{groupplot}[
            group style={
                group size=1 by 1,
            },
            % size
            width=0.6\textwidth,
            % y
            ylabel={$\left\langle \Performance \right\rangle / \SwimmingVelocity$},
            y label style={yshift=-4pt},
            ymin=0.0,
            ymax=2.5,
            % x
            x label style={yshift=4pt},
            xmin=0.0,
            % layers
            set layers,
            % legend
            legend style={draw=none, fill=none},
            legend cell align=left,
        ]
    \nextgroupplot[
        %width=0.33\linewidth,
        % x
        extra y ticks={0.5, 1.5, 2.5},
        xlabel=$\ReorientationTime / \KolmogorovTimeScale$,
        xmax=4,
        % legend
        legend pos=north east,
        legend style={fill opacity=0.5, text opacity=1},
    ]
        % shade
        \addplot[name path=A, draw=none, forget plot] table [
            x index=4, 
            y expr={(\thisrowno{0} - \thisrowno{1}) / 0.066},
            col sep=comma, 
            comment chars=\#,
        ]{parts/turbulence/data/jhtdb_reorientation_time/max.csv};
        \addplot[name path=B, draw=none, forget plot] table [
            x index=4, 
            y expr={(\thisrowno{0} + \thisrowno{1}) / 0.066},
            col sep=comma,
            comment chars=\#,
        ]{parts/turbulence/data/jhtdb_reorientation_time/max.csv};
        \addplot[ColorSurf, opacity=0.4, forget plot, visible on=<1->] fill between[of=A and B];
        \addplot[name path=A, draw=none, forget plot] table [
            x index=4, 
            y expr={(\thisrowno{0} - \thisrowno{1}) / 0.066},
            col sep=comma, 
            comment chars=\#,
            restrict expr to domain={\thisrowno{3}}{0.0:0.0},
        ]{parts/turbulence/data/jhtdb_reorientation_time/merge.csv};
        \addplot[name path=B, draw=none, forget plot] table [
            x index=4, 
            y expr={(\thisrowno{0} + \thisrowno{1}) / 0.066},
            col sep=comma,
            comment chars=\#,
            restrict expr to domain={\thisrowno{3}}{0.0:0.0},
        ]{parts/turbulence/data/jhtdb_reorientation_time/merge.csv};
        \addplot[ColorBh, opacity=0.4, forget plot, visible on=<1->] fill between[of=A and B];
        % plot
        \addplot
        [
        color=ColorSurf,
        opacity=1.0,
        line width=1pt, 
        only marks,%solid,
        mark=square*,
        visible on=<1->,
        ]
        table[
            x index=4, 
            y expr={\thisrowno{0} / 0.066},
            col sep=comma, 
            comment chars=\#,
            unbounded coords=discard,
        ]{parts/turbulence/data/jhtdb_reorientation_time/max.csv};
        \addlegendentry{\NameSurf}
        %\addplot
        %[
        %color=pink!50!black,
        %opacity=1.0,
        %line width=1pt, 
        %dashed,
        %]
        %table[
        %    x index=4, 
        %    y expr={1.0},
        %    col sep=comma, 
        %    comment chars=\#,
        %    restrict expr to domain={\thisrowno{3}}{0.0:0.0},
        %    unbounded coords=discard,
        %]{parts/turbulence/data/jhtdb_reorientation_time/final_merge.csv};
        %\addlegendentry{\NameBh}
        \addplot
        [
        color=ColorBh,
        opacity=1.0,
        line width=1pt, 
        only marks,%solid,
        mark=o,
        visible on=<1->,
        ]
        table[
            x index=4, 
            y expr={\thisrowno{0} / 0.066},
            col sep=comma, 
            comment chars=\#,
            restrict expr to domain={\thisrowno{3}}{0.0:0.0},
            unbounded coords=discard,
        ]{parts/turbulence/data/jhtdb_reorientation_time/merge.csv};
        \addlegendentry{bottom-heavy}
        \addplot
        [
        color=white,
        opacity=1.0,
        solid, 
        on layer=axis background,
        domain=0.0:4,
        ]{1};
    \end{groupplot}
\end{tikzpicture}
%\tikzexternaldisable

		\captionsetup{skip=2pt, margin={0.0\textwidth, -0.0\textwidth}}
		\caption{
			Migration performance in turbulence ($\mathit{Re}_{\lambda} \approx 418$) as a function of the reorientation time $\ReorientationTime / \KolmogorovTimeScale$. Monthiller et al. (2022).
		}
	\end{figure}

	\vskip0pt plus 1filll

	\vspace{0pt}
	\visible<2->{
		\begin{itemize}
			\large
			\item Surfing is \textbf{robust} to finite reorientation time.
		\end{itemize}
	}
	\vspace{30pt}
\end{frame}

%-------------------------------------------------------
% FORMULATION: PRESENTATION PLAN
%-------------------------------------------------------

\begin{frame}{5 Biophysical \textbf{relevance} of the strategy}{1 Introduction - 2 Problem - 3 Surfing - 4 Performance - 6 Perspectives}
	\centering
	\vspace{15pt}
	\textbf{\Large Outline}

	\vspace{15pt}

	\large
	\begin{enumerate}
		\setlength\itemsep{10pt}
		\item \textcolor{gray}{\textbf{Introduction} to the world of plankton}
		\item \textcolor{gray}{Formulation of the \textbf{navigation problem}}
		\item \textcolor{gray}{Solution: The \textbf{surfing} strategy}
		\item \textcolor{gray}{Evaluation of surfing \textbf{performance}}
		\item \textcolor{white}{Biophysical \textbf{relevance} of the strategy}
		\item \textcolor{gray}{Summary and \textbf{perspectives}}
	\end{enumerate}

\end{frame}

%-------------------------------------------------------
% RELEVANCE: SETUP
%-------------------------------------------------------

\begin{frame}{5 Biophysical \textbf{relevance} of the strategy}{1 Introduction - 2 Problem - 3 Surfing - 4 Performance - 6 Perspectives}
	\centering
	\vspace{5pt}
	\textbf{\large How beneficial would the surfing strategy be \\ for actual plankters in their habitat?}

	\pause
	\vspace{10pt}
	\begin{itemize}
		\item Surfing performance is impacted by $\displaystyle \frac{\SwimmingVelocity}{\KolmogorovVelocityScale}$ and $\displaystyle \frac{\tau_{\mathrm{align, \NameSurfShort}}}{\KolmogorovTimeScale}$.
	\end{itemize}

	\pause
	\vspace{5pt}
	\begin{figure}
		\begin{tikzpicture}
			\visible<3> {
				\node[inner sep=0pt] (copepod) at (0,0) {
					\def\svgwidth{0.9\textwidth}
					\input{parts/relevance/schemes/typical_plankton_0.pdf_tex}
				};
			}
			\visible<4> {
				\node[inner sep=0pt] (copepod) at (0,0) {
					\def\svgwidth{0.9\textwidth}
					\input{parts/relevance/schemes/typical_plankton_1.pdf_tex}
				};
			}
			\visible<5> {
				\node[inner sep=0pt] (copepod) at (0,0) {
					\def\svgwidth{0.9\textwidth}
					\input{parts/relevance/schemes/typical_plankton_2.pdf_tex}
				};
			}
			\visible<6> {
				\node[inner sep=0pt] (copepod) at (0,0) {
					\def\svgwidth{0.9\textwidth}
					\input{parts/relevance/schemes/typical_plankton_3.pdf_tex}
				};
			}
		\end{tikzpicture}
		\captionsetup{skip=5pt, margin={0.0\textwidth, 0.0\textwidth}}
		\caption{
			Typical plankter parameters. Size ratios are respected. $\delta = d /200$.
		}
	\end{figure}
\end{frame}

%-------------------------------------------------------
% RELEVANCE: RESULTS
%-------------------------------------------------------

\begin{frame}{5 Biophysical \textbf{relevance} of the strategy}{1 Introduction - 2 Problem - 3 Surfing - 4 Performance - 6 Perspectives}
	\centering
	\vspace{-5pt}
	\begin{figure}
		\begin{tikzpicture}
			\visible<1> {
				\node[inner sep=0pt] (copepod) at (0,0) {
					\def\svgwidth{0.48\textwidth}
					\input{parts/relevance/schemes/bio_relevance_0.pdf_tex}
				};
			}
			\visible<2> {
				\node[inner sep=0pt] (copepod) at (0,0) {
					\def\svgwidth{0.48\textwidth}
					\input{parts/relevance/schemes/bio_relevance_1.pdf_tex}
				};
			}
			\visible<3> {
				\node[inner sep=0pt] (copepod) at (0,0) {
					\def\svgwidth{0.48\textwidth}
					\input{parts/relevance/schemes/bio_relevance_2.pdf_tex}
				};
			}
			\visible<4> {
				\node[inner sep=0pt] (copepod) at (0,0) {
					\def\svgwidth{0.48\textwidth}
					\input{parts/relevance/schemes/bio_relevance_3.pdf_tex}
				};
			}
			\visible<5> {
				\node[inner sep=0pt] (copepod) at (0,0) {
					\def\svgwidth{0.48\textwidth}
					\input{parts/relevance/schemes/bio_relevance_4.pdf_tex}
				};
			}
		\end{tikzpicture}
		\captionsetup{skip=20pt}%, margin={0.0\textwidth, 0.0\textwidth}}
		\caption{
			Estimation of surfing performance as a function of plankton habitat.
		}
	\end{figure}
\end{frame}

%-------------------------------------------------------
% RELEVANCE: DINSTINGUISH
%-------------------------------------------------------

\begin{frame}{5 Biophysical \textbf{relevance} of the strategy}{1 Introduction - 2 Problem - 3 Surfing - 4 Performance - 6 Perspectives}
	\centering
	\vspace{5pt}
	\textbf{\large Surfing would be beneficial for actual plankters. \\ \visible<2->{But are plankters actually surfing?}}

	\vspace{15pt}
	\begin{columns}[c]
		\column{.55\textwidth}
		\centering
		\begin{itemize}
			\large
			\setlength\itemsep{10pt}
			\item<3-> DiBenedetto et al. (2022)
			\item<3-> \textit{Crepidula fornicata} larvae
			\item<4-> migrate upward to \textbf{disperse}
			\item<5-> migrate downward to \textbf{settle}
			\item<6-> \textbf{ongoing collaboration} with \\ Michelle DiBenedetto
		\end{itemize}

		\column{.4\textwidth}
		\centering
		\visible<3->{
			\begin{figure}
				\def\svgwidth{1.0\textwidth}
				\input{parts/relevance/schemes/shell.pdf_tex}
				\caption{
					Shell of \textit{Crepidula fornicata}.
				}
			\end{figure}
		}
	\end{columns}
\end{frame}

%-------------------------------------------------------
% RELEVANCE: EXPERIMENT
%-------------------------------------------------------

\begin{frame}{5 Biophysical \textbf{relevance} of the strategy}{1 Introduction - 2 Problem - 3 Surfing - 4 Performance - 6 Perspectives}
	\centering
	\vspace{5pt}
	\textbf{\large Description of the experiment}

	\vspace{0pt}
	\begin{figure}
		\begin{tikzpicture}
			\visible<1> {
				\node[inner sep=0pt] (copepod) at (0,0) {
					\def\svgwidth{0.75\textwidth}
					\input{parts/relevance/schemes/tank_0.pdf_tex}
				};
			}
			\visible<2> {
				\node[inner sep=0pt] (copepod) at (0,0) {
					\def\svgwidth{0.75\textwidth}
					\input{parts/relevance/schemes/tank_1.pdf_tex}
				};
			}
			\visible<3> {
				\node[inner sep=0pt] (copepod) at (0,0) {
					\def\svgwidth{0.75\textwidth}
					\input{parts/relevance/schemes/tank_2.pdf_tex}
				};
			}
			\visible<4-> {
				\node[inner sep=0pt] (copepod) at (0,0) {
					\def\svgwidth{0.75\textwidth}
					\input{parts/relevance/schemes/tank_3.pdf_tex}
				};
			}
		\end{tikzpicture}
		\captionsetup{skip=5pt}%, margin={0.0\textwidth, 0.0\textwidth}}
		\caption{
			Description of the experiment of DiBenedetto et al. (2022).
		}
	\end{figure}

	\vspace{-12pt}
	\only<5->{
		\begin{itemize}
			\item Slip velocity: $\vec{V}_s = (d \ParticlePosition / dt) - \FlowVelocity(\ParticlePosition, t)$ proxy of $\SwimmingVelocity \SwimmingDirection$
		\end{itemize}
	}
\end{frame}

%-------------------------------------------------------
% RELEVANCE: DISTINGUISH
%-------------------------------------------------------

\begin{frame}{5 Biophysical \textbf{relevance} of the strategy}{1 Introduction - 2 Problem - 3 Surfing - 4 Performance - 6 Perspectives}
	\centering
	\vspace{5pt}
	\textbf{\large How to differentiate surfing from bottom-heaviness?}
	
	\centering
	\vspace{10pt}
	\begin{figure}
		\begin{tikzpicture}
			\visible<1> {
				\node[inner sep=0pt] (copepod) at (0,0) {
					\def\svgwidth{0.8\textwidth}
					\input{parts/relevance/schemes/distinguish_0.pdf_tex}
				};
			}
			\visible<2> {
				\node[inner sep=0pt] (copepod) at (0,0) {
					\def\svgwidth{0.8\textwidth}
					\input{parts/relevance/schemes/distinguish_1.pdf_tex}
				};
			}
			\visible<3> {
				\node[inner sep=0pt] (copepod) at (0,0) {
					\def\svgwidth{0.8\textwidth}
					\input{parts/relevance/schemes/distinguish_2.pdf_tex}
				};
			}
			\visible<4> {
				\node[inner sep=0pt] (copepod) at (0,0) {
					\def\svgwidth{0.8\textwidth}
					\input{parts/relevance/schemes/distinguish_3.pdf_tex}
				};
			}
			\visible<5> {
				\node[inner sep=0pt] (copepod) at (0,0) {
					\def\svgwidth{0.8\textwidth}
					\input{parts/relevance/schemes/distinguish_4.pdf_tex}
				};
			}
			\visible<6> {
				\node[inner sep=0pt] (copepod) at (0,0) {
					\def\svgwidth{0.8\textwidth}
					\input{parts/relevance/schemes/distinguish_5.pdf_tex}
				};
			}
			\visible<7> {
				\node[inner sep=0pt] (copepod) at (0,0) {
					\def\svgwidth{0.8\textwidth}
					\input{parts/relevance/schemes/distinguish_6.pdf_tex}
				};
			}
		\end{tikzpicture}
		\captionsetup{skip=10pt}%, margin={0.0\textwidth, -0.2\textwidth}}
		\caption{
			Illustration of how to distinguish surfing from bottom-heaviness.
		}
	\end{figure}
\end{frame}

%-------------------------------------------------------
% RELEVANCE: NUMERICAL AND EXPERIMENTAL RESULTS
%-------------------------------------------------------

\begin{frame}{5 Biophysical \textbf{relevance} of the strategy}{1 Introduction - 2 Problem - 3 Surfing - 4 Performance - 6 Perspectives}
	\centering
	\vspace{5pt}
	\textbf{\large Numerical and experimental results}

	\vspace{-5pt}
	\begin{figure}
		\begin{tikzpicture}
	\begin{groupplot}[
			group style={
				group size=2 by 1,
				horizontal sep=0.14\linewidth,
			},
			axis on top,
			% size
			width=0.5\textwidth,
			%ymode=log,
			% layers
			set layers ,
			% legend
			legend style={draw=none, fill=none, /tikz/every even column/.append style={column sep=4pt}, at={(1.1, 1.05)}, anchor=south},
			%legend pos=north east,
			legend cell align=right,
			legend columns=-1,
		]
		% n_{surf, x}
		\nextgroupplot[
			% x
			xlabel=$\FlowVorticityScalar_y \KolmogorovTimeScale$,
			xmin=-1,
			xmax=1,
			% y
			ylabel=$\langle \SwimmingDirection_{x} \rangle$,
			ylabel shift = -5pt,
			ymin=-1,
			ymax=1,
			ytick={-1,0,1},
		]
			% tau 2.0
			\addplot[
				ColorSurf,
				mark=square*,
				only marks,
			] table [
				x expr={\thisrowno{0}},
				y expr={\thisrowno{1}},
				col sep=comma, 
				comment chars=\#,
				unbounded coords=discard,
			] {parts/relevance/data/control_surfers__flow__n_128__re_250/control_surfer__vs_1o0__surftimeconst_2o0__omegamax_1o0__vorticity_z__p_y.csv};
			\addlegendentry{\NameSurf}
			% reorientationtime 1.0
			\addplot[
				ColorBh,
				mark=o,
				only marks,
			] table [
				x expr={\thisrowno{0}},
				y expr={\thisrowno{1}},
				col sep=comma,
				comment chars=\#,
				unbounded coords=discard,
			] {parts/relevance/data/control_surfers__flow__n_128__re_250/spherical_riser__vs_1o0__reorientationtime_1o0__vorticity_z__p_y.csv};
			\addlegendentry{\NameBh \quad\quad}
			% ref
			\addplot
	        [
		        color=white,
		        opacity=0.5,
		        solid, 
		        on layer=axis background,
		        domain=-1:1,
		        forget plot,
	        ]{0};
	        % exp
	        \addlegendimage{ColorAlt,mark=asterisk,only marks} \addlegendentry{Data of DiBenedetto et al. (2022)}
		% n_{surf, x}
		\nextgroupplot[
			% x
			xlabel=$\FlowVorticityScalar_y \KolmogorovTimeScale$,
			xmin=-1,
			xmax=1,
			% y
			ylabel=$\langle V_{s,x} \rangle / \langle V_{s,z} \rangle$,
			ylabel shift = -15pt,
			ymin=-1,
			ymax=1,
			ytick={-1,0,1},
		]
			% shade
	        \addplot[name path=A, draw=none, forget plot] table [
	            x expr={\thisrowno{0} * 0.67},
	            y expr={\thisrowno{3} / 0.069},
	            col sep=comma, 
	            comment chars=\#,
	        ]{parts/relevance/data/exp/exp_data.csv};
	        \addplot[name path=B, draw=none, forget plot] table [
	            x expr={\thisrowno{0} * 0.67},
	           	y expr={\thisrowno{2} / 0.069},
	            col sep=comma,
	            comment chars=\#,
	        ]{parts/relevance/data/exp/exp_data.csv};
	        \addplot[ColorAlt, opacity=0.4, forget plot, visible on=<2->] fill between[of=A and B];
			% tau 1.0
			\addplot[
				ColorAlt,
				mark=asterisk,
				only marks,
				visible on=<2->
			] table [
				x expr={\thisrowno{0} * 0.67},
				y expr={\thisrowno{1} / 0.069},
				col sep=comma, 
				comment chars=\#,
				unbounded coords=discard,
			] {parts/relevance/data/exp/exp_data.csv};
			% ref
			\addplot
	        [
		        color=white,
		        opacity=0.5,
		        solid, 
		        on layer=axis background,
		        domain=-1:1,
	        ]{0};
	\end{groupplot}
\end{tikzpicture}

		\captionsetup{skip=0pt}%, margin={0.0\textwidth, 0.0\textwidth}}
		\caption{
			Orientation statistics as a function of flow vorticity $\omega_y$ obtained from \textbf{(left)} simulations \textbf{(right)} experiments of DiBenedetto et al. (2022).
		}
	\end{figure}

	\vspace{-10pt}
	\begin{itemize}
		\item<3-> Might be the \textbf{first observation} of plankter ``surfing'' (\textbf{ongoing})
	\end{itemize}
\end{frame}













% %-------------------------------------------------------
% % RELEVANCE: PHOTOTACTIC SURFING
% %-------------------------------------------------------
% 
% \begin{frame}{5 Biophysical \textbf{relevance} of the strategy}{1 Introduction - 2 Problem - 3 Surfing - 4 Performance - 6 Perspectives}
	% \centering
	% \vspace{15pt}
	% \textbf{\large Many plankters are phototactic!}
% 
	% \pause
	% \vspace{30pt}
	% \def\svgwidth{0.8\textwidth}
	% \input{parts/relevance/schemes/phototactic_tank.pdf_tex}
% \end{frame}

%-------------------------------------------------------
% CONCLUSION: 
%-------------------------------------------------------

\begin{frame}{6 Summary and \textbf{perspectives}}{1 Introduction - 2 Problem - 3 Surfing - 4 Performance - 5 Relevance}
	\centering
	\vspace{15pt}
	\textbf{\Large Outline}

	\vspace{15pt}

	\large
	\begin{enumerate}
		\setlength\itemsep{10pt}
		\item \textcolor{gray}{\textbf{Introduction} to the world of plankton}
		\item \textcolor{gray}{Formulation of the \textbf{navigation problem}}
		\item \textcolor{gray}{Solution: The \textbf{surfing} strategy}
		\item \textcolor{gray}{Evaluation of surfing \textbf{performance}}
		\item \textcolor{gray}{Biophysical \textbf{relevance} of the strategy}
		\item \textcolor{white}{Summary and \textbf{perspectives}}
	\end{enumerate}
\end{frame}

\begin{frame}{6 Summary and \textbf{perspectives}}{1 Introduction - 2 Problem - 3 Surfing - 4 Performance - 5 Relevance}
	\centering
	\vspace{5pt}
	\textbf{\Large Summary}

	\vspace{10pt}
	\centering
	\begin{itemize}
		\setlength\itemsep{2pt}
		\item<1-> Formalized the \textbf{navigation problem} of plankton \textbf{vertical migration}
		\item<2-> We derived an \textbf{approximate analytical solution}
		\item<3-> We demonstrated its \textbf{performance in turbulence}
		\item<4-> We demonstrated its \textbf{relevance} for actual plankters in their habitat
	\end{itemize}

	\vspace{40pt}
	\begin{tikzpicture}
		\node<1>[inner sep=0pt] (copepod) at (0,0) {
			\def\svgwidth{0.9\textwidth}
			\input{parts/upscaling/summary_0.pdf_tex}
		};
		\node<2>[inner sep=0pt] (copepod) at (0,0) {
			\def\svgwidth{0.9\textwidth}
			\input{parts/upscaling/summary_1.pdf_tex}
		};
		\node<3>[inner sep=0pt] (copepod) at (0,0) {
			\def\svgwidth{0.9\textwidth}
			\input{parts/upscaling/summary_2.pdf_tex}
		};
		\node<4->[inner sep=0pt] (copepod) at (0,0) {
			\def\svgwidth{0.9\textwidth}
			\input{parts/upscaling/summary_3.pdf_tex}
		};
	\end{tikzpicture}
\end{frame}

\begin{frame}{6 Summary and \textbf{perspectives}}{1 Introduction - 2 Problem - 3 Surfing - 4 Performance - 5 Relevance}
	\centering
	\vspace{-10pt}
	\small
	\begin{multicols}{2}
		\begin{itemize}
			%\setlength\itemsep{2pt}
			\item \textcolor{ColorSurf}{Plankton}
				\begin{itemize}
					\scriptsize
					\item[$\bullet$] \textcolor{ColorSurf}{What are plankton}
					\item[$\bullet$] \textcolor{ColorSurf}{Role in marine food web}
					\item<2->[$\bullet$] Threat of climate change for plankton
					\item<2->[$\bullet$] Biological pump
				\end{itemize}
			\item \textcolor{ColorSurf}{Plankton navigation problem}
				\begin{itemize}
					\scriptsize
					\item[$\bullet$] \textcolor{ColorSurf}{Vertical migration}
					\item<2->[$\bullet$] Horizontal dispersion
					\item<2->[$\bullet$] Odor and light tracking
				\end{itemize}
			\item \textcolor{ColorSurf}{Surfing strategy}
				\begin{itemize}
					\scriptsize
					\item[$\bullet$] \textcolor{ColorSurf}{Derivation}
					\item<2->[$\bullet$] Characterisation in linear flows
					\item[$\bullet$] \textcolor{ColorSurf}{Taylor-Green illustration}
					\item<2->[$\bullet$] Surfing in a Poiseuille flow
				\end{itemize}
			\item \textcolor{ColorSurf}{Surfing on turbulence}
				\begin{itemize}
					\scriptsize
					\item[$\bullet$] \textcolor{ColorSurf}{Performance}
					\item<2->[$\bullet$] Flow sampled by surfers
					\item<2->[$\bullet$] Influence of $\mathit{Re}_{\lambda}$
					\item<2->[$\bullet$] Estimation of performance
				\end{itemize}
			\item \textcolor{ColorSurf}{Robustness}
				\begin{itemize}
					\scriptsize
					\item[$\bullet$] \textcolor{ColorSurf}{Reorientation time}
					\item<2->[$\bullet$] Turbulence fluctuations
					\item<2->[$\bullet$] Noise
					\item<2->[$\bullet$] Filtering
					\item<2->[$\bullet$] Partial measure
					\item<2->[$\bullet$] Processing power
				\end{itemize}
			\item<2-> Other navigation approaches
				\begin{itemize}
					\scriptsize
					\item<2->[$\bullet$] Zermelo equation
					\item<2->[$\bullet$] Dijkstra algorithm
					\item<2->[$\bullet$] Reinforcement learning
				\end{itemize}
			\item \textcolor{ColorSurf}{Biophysical relevance}
				\begin{itemize}
					\scriptsize
					\item[$\bullet$] \textcolor{ColorSurf}{Demonstration of relevance}
					\item<2->[$\bullet$] Optimality in marine ecology
				\end{itemize}
			\item<2-> Additional motion dynamics
			\item<2-> Horizontal dispersion problem
			\item<2-> Energey efficient navigation
		\end{itemize}
	\end{multicols}
\end{frame}

\begin{frame}{6 Summary and \textbf{perspectives}}{1 Introduction - 2 Problem - 3 Surfing - 4 Performance - 5 Relevance}
	\centering
	\vspace{4pt}
	\textbf{\Large Perspectives}

	\pause
	\vspace{2pt}
	\centering
	\begin{itemize}
		\setlength\itemsep{10pt}
		\item<1-> \textbf{Generalization}
			\begin{align*}
				\ControlDirectionOpt^* = \frac{\ControlDirectionOptNN^*}{\norm{\ControlDirectionOptNN^*}}, \quad \text{with} \quad \ControlDirectionOptNN^* &= \exp \left[ \TimeHorizon (\Gradients)^\top + \frac{\TimeHorizon^{2}}{2} \frac{d (\Gradients)^\top}{dt} + \cdots \right] \cdot \Direction \\
				 &= \exp \left[ \sum_{k=0}^\infty \frac{\TimeHorizon^{k+1}}{(k+1)!} \frac{d^k (\Gradients)^\top}{dt^k} \right] \cdot \Direction
			\end{align*}
		\item<1-> \st{Approximate} \textbf{Actual optimal solution}
		\item<1-> $\ControlDirectionOpt$ is a first order approximation of $\ControlDirectionOpt^*$
		\item<1-> function of \textbf{local information} contrary to the Zermelo equation or the Dijkstra algorithm $\to$ animal navigation
		\item<1-> \textbf{Positional target:} $\Direction(t)$ \quad $\Direction \to \Direction(\TimeHorizon) \approx \Direction_0 + \TimeHorizon (\partial \Direction/ \partial t)_0 + \cdots$
	\end{itemize}
\end{frame}

\begin{frame}{6 Summary and \textbf{perspectives}}{1 Introduction - 2 Problem - 3 Surfing - 4 Performance - 5 Relevance}
	\centering
	\vspace{5pt}
	\textbf{\Large Perspectives}

	\vspace{15pt}
	\centering
	\leavevmode\hidewidth\begin{tikzpicture}
		\node[inner sep=0pt] (copepod) at (0,0) {
			\def\svgwidth{1.0\textwidth}
			\input{parts/upscaling/perspectives.pdf_tex}
		};
	\end{tikzpicture}\hidewidth\null

	\vspace{0pt}
	\centering
	\begin{multicols}{2}
		\begin{itemize}
			\scriptsize
			\setlength\itemsep{2pt}
			\item \textbf{Sheld0n:} https://github.com/C0PEP0D/sheld0n
			\item \textbf{Thesis:} https://github.com/rmonthil-phd
			\item \textbf{Defense:} https://github.com/rmonthil-phd
			\item \textbf{Video-game:} https://akarius.itch.io/floward
		\end{itemize}
	\end{multicols}
\end{frame}

%-------------------------------------------------------
% MORE: 
%-------------------------------------------------------

\begin{frame}[noframenumbering]{3 Solution: The \textbf{surfing} strategy}{1 Introduction - 2 Problem - 4 Performance - 5 Relevance - 6 Perspectives}
	\centering
	\textbf{\Large Definition of the gradient}

	\Large
	\begin{center}
		\begin{equation*}
			\Gradients = \begin{pmatrix}
				\displaystyle \frac{\partial \FlowVelocityScalar_x}{\partial x} & \displaystyle \frac{\partial \FlowVelocityScalar_x}{\partial y} & \displaystyle \frac{\partial \FlowVelocityScalar_x}{\partial z} \\[20pt]
				\displaystyle \frac{\partial \FlowVelocityScalar_y}{\partial x} & \displaystyle \frac{\partial \FlowVelocityScalar_y}{\partial y} & \displaystyle \frac{\partial \FlowVelocityScalar_y}{\partial z} \\[20pt]
				\displaystyle \frac{\partial \FlowVelocityScalar_z}{\partial x} & \displaystyle \frac{\partial \FlowVelocityScalar_z}{\partial y} & \displaystyle \frac{\partial \FlowVelocityScalar_z}{\partial z}
			\end{pmatrix}
		\end{equation*}
	\end{center}
\end{frame}

%-------------------------------------------------------
% SURFING STRATEGY: INTERPRETATION
%-------------------------------------------------------

\begin{frame}[noframenumbering]{3 Solution: The \textbf{surfing} strategy}{1 Introduction - 2 Problem - 4 Performance - 5 Relevance - 6 Perspectives}
	\centering
	\textbf{\Large Physical interpretation of \textit{surfing}}
	\begin{equation*}
		\ControlDirectionOpt = \frac{\ControlDirectionOptNN}{\norm{\ControlDirectionOptNN}}, \quad \text{with} \quad
		\ControlDirectionOptNN = \Direction + \TimeHorizon \, \vec{\nabla} \FlowVelocityScalar_z + \frac{1}{2} \TimeHorizon^2 \, \vec{\nabla} ( \FlowVelocity \cdot \vec{\nabla} \FlowVelocityScalar_\DirectionScalar ) + \dotsb
	\end{equation*}

	\begin{figure}
		\begin{tikzpicture}
			\node[inner sep=0pt] (copepod) at (0,0) {
				\def\svgwidth{0.3\textwidth}
				\input{parts/surf/schemes/physical_interpretation_3.pdf_tex}
			};
		\end{tikzpicture}
		\captionsetup{skip=10pt}
		\caption{
			Illustration of the surfing strategy in a simple vortex flow.
		}
	\end{figure}

	\vskip0pt plus 1filll

	\large
	\begin{itemize}
		\item<1-> succession of \textbf{``gradient ascents''} controlled by $\TimeHorizon$
	\end{itemize}
	\vspace{15pt}
\end{frame}

% %-------------------------------------------------------
% % SURFING STRATEGY: INTERPRETATION
% %-------------------------------------------------------
% 
% \begin{frame}[noframenumbering]{3 Solution: The \textbf{surfing} strategy}{1 Introduction - 2 Problem - 4 Performance - 5 Relevance - 6 Perspectives}
	% \centering
	% \textbf{\Large Physical interpretation of \textit{surfing}}
	% \only<1>{
		% \begin{equation*}
			% \ControlDirection = \frac{\ControlDirectionOptNN}{\norm{\ControlDirectionOptNN}}, \quad \text{with} \quad \ControlDirectionOptNN = \exp \left[ \TimeHorizon (\Gradients)^\top \right] \, \Direction
		% \end{equation*}
	% }
	% % \only<2>{
		% % \begin{equation*}
			% % \ControlDirectionOpt = \frac{\ControlDirectionOptNN}{\norm{\ControlDirectionOptNN}}, \quad \text{with} \quad
			% % \ControlDirectionOptNN = \sum_{k = 0}^{\infty} \frac{\TimeHorizon^k}{k!}  \left[ (\Gradients)^\top \right]^k \cdot \Direction
		% % \end{equation*}
	% % }
	% % \only<4>{
		% % \begin{equation*}
			% % \ControlDirectionOpt = \frac{\ControlDirectionOptNN}{\norm{\ControlDirectionOptNN}}, \quad \text{with} \quad
			% % \ControlDirectionOptNN = \Direction + \TimeHorizon \, (\Gradients)^\top \cdot \Direction + \frac{1}{2} \TimeHorizon^2 [ (\Gradients)^\top ]^2 \cdot \Direction + \dotsb
		% % \end{equation*}
	% % }
	% \only<2->{
		% \begin{equation*}
			% \ControlDirectionOpt = \frac{\ControlDirectionOptNN}{\norm{\ControlDirectionOptNN}}, \quad \text{with} \quad
			% \ControlDirectionOptNN = \alt<3>{\textcolor{ColorSurf}{\Direction}}{\Direction} + \TimeHorizon \, \alt<4>{\textcolor{ColorSurf}{\vec{\nabla} \FlowVelocityScalar_z}}{\vec{\nabla} \FlowVelocityScalar_z} + \frac{1}{2} \TimeHorizon^2 \, \alt<6->{\textcolor{ColorSurf}{\vec{\nabla} ( \FlowVelocity \cdot \vec{\nabla} \FlowVelocityScalar_\DirectionScalar )}}{\vec{\nabla} ( \FlowVelocity \cdot \vec{\nabla} \FlowVelocityScalar_\DirectionScalar )} + \dotsb
		% \end{equation*}
	% }
% 
	% \visible<2->{
		% \begin{figure}
			% \begin{tikzpicture}
				% \visible<2>{
					% \node[inner sep=0pt] (copepod) at (0,0) {
						% \def\svgwidth{0.3\textwidth}
						% \input{parts/surf/schemes/physical_interpretation_0.pdf_tex}
					% };
				% }
				% \visible<3>{
					% \node[inner sep=0pt] (copepod) at (0,0) {
						% \def\svgwidth{0.3\textwidth}
						% \input{parts/surf/schemes/physical_interpretation_1.pdf_tex}
					% };
				% }
				% \visible<4>{
					% \node[inner sep=0pt] (copepod) at (0,0) {
						% \def\svgwidth{0.3\textwidth}
						% \input{parts/surf/schemes/physical_interpretation_2.pdf_tex}
					% };
				% }
				% \visible<5>{
					% \node[inner sep=0pt] (copepod) at (0,0) {
						% \def\svgwidth{0.3\textwidth}
						% \input{parts/surf/schemes/physical_interpretation_25.pdf_tex}
					% };
				% }
				% \visible<6->{
					% \node[inner sep=0pt] (copepod) at (0,0) {
						% \def\svgwidth{0.3\textwidth}
						% \input{parts/surf/schemes/physical_interpretation_3.pdf_tex}
					% };
				% }
			% \end{tikzpicture}
			% \captionsetup{skip=10pt}
			% \caption{
				% Illustration of the surfing strategy in a simple vortex flow.
			% }
		% \end{figure}
	% }
% 
	% \vskip0pt plus 1filll
% 
	% \large
	% \begin{itemize}
		% \item<7-> succession of \textbf{``gradient ascents''} controlled by $\TimeHorizon$
	% \end{itemize}
	% \vspace{15pt}
% \end{frame}

% %-------------------------------------------------------
% % SURFING STRATEGY: Taylor-Green Vortices: Definition
% %-------------------------------------------------------
% 
% \begin{frame}{3 Solution: The \textbf{surfing} strategy}{1 Introduction - 2 Problem - 4 Performance - 5 Relevance - 6 Perspectives}
	% \vspace{5pt}
	% \centering
	% \textbf{\Large Taylor-Green Vortices}
% 
	% \pause
	% \begin{equation*}\label{eq:taylor_green_vortex_velocity}
		% \FlowVelocity(\vec{x}) = \FlowVelocityScalar_{\mathrm{max}}
		% \begin{pmatrix}
			% \cos (x/L) \, \sin (y/L) \\
			% -\sin (x/L) \, \cos (y/L)
		% \end{pmatrix},
		% \quad \text{with} \quad
		% \vec{x} =
		% \begin{pmatrix}
			% x \\
			% y
		% \end{pmatrix}
	% \end{equation*}
% 
	% \input{parts/surf/plots/tgv.tex}
% 
	% \vskip0pt plus 1filll
	% 
	% \pause
	% \large
	% \begin{itemize}
		% \item $\FlowVelocityScalar_{\max}$ and $\FlowVorticityScalar_{\max} = 2 \FlowVelocityScalar_{\max} / L$ are used to \textbf{scale} the results
	% \end{itemize}
	% \vspace{5pt}
% \end{frame}


% 
% % \begin{frame}{Conclusion}
	% % \vspace{20pt}
% % %-------------------------------------------------------
	% % \begin{center}
		% % \textbf{Surfing} is a \textbf{physics-based strategy} for\\ \textbf{efficient navigation} in \textbf{turbulence}.
		% % \pause
		% % \vspace{10pt}
		% % \begin{itemize}
			% % \setlength\itemsep{10pt}
			% % \item<3-> \textbf{Relevant} for real \textbf{planktonic organisms}: $\SwimmingVelocity \lesssim \KolmogorovU$
			% % \item<4-> \textbf{Comparison} with reinforcement learning in progress.
				% % \begin{itemize}
					% % \item<5-> Surfing seems \textbf{competitive}
				% % \end{itemize}
			% % \item<6-> Influence of \textbf{physical parameters}: $Re_{\lambda}$ (universality?), predict $\TimeHorizonOpt$ and performance to properties of the flow sampled by active particles
			% % \item<7-> \textbf{Generalization} of this strategy to \textbf{other navigation problems} (dispersion, point to point navigation, odour tracking ...)
		% % \end{itemize}
	% % \end{center}
	% %
	% % %\vskip0pt plus 1filll
	% % %
	% % %\visible<4->{
	% % %    \footnotesize Alageshan et al, Phys. Rev. E (2020).
	% % %}
	% % %\vspace{10pt}
% % \end{frame}
% 
% \begin{frame}[noframenumbering]{Optimal parameter $\TimeHorizon$}{"Doppler" effect}
	% \centering
	% \vspace{20pt}
% %-------------------------------------------------------
	% \begin{columns}[c]
		% \column{.5\textwidth}
		% \centering
		% \vspace{10pt}
		% \input{plots/jhtdb_more_2.tex}
		% 
		% \column{.5\textwidth}
		% \centering
		% \vspace{7pt}
		% \input{plots/jhtdb_more_1.tex}
	% \end{columns}
% 
	% \vskip0pt plus 1filll
	% 
	% \vspace{-20pt}
	% \begin{equation*}
		% \label{eq:corr}
		% \CorrelationTime (\SwimmingVelocity) = \frac{\int \left\langle I(\omega) \right\rangle \frac{2\pi}{\omega} \, d\omega}{\int \left\langle I(\omega) \right\rangle \, d\omega} ~~ \text{with} ~~ I(\omega) = \sqrt{ \frac{d}{d\omega} \mathrm{tr} ( \Gradients^2 ) }
	% \end{equation*}
	% 
	% \vspace{10pt}
% \end{frame}
% 
% \begin{frame}[noframenumbering]{Comparison to reinforcement learning}{}
	% \centering
	% \vspace{10pt}
% %-------------------------------------------------------
	% \input{plots/comparison.tex}
% 
	% \vskip0pt plus 1filll
% 
	% \begin{itemize}
		% \item Surf: $66\%$, Alageshan et al., Phys. Rev. E (2020): $11\%$
	% \end{itemize}
	% \vspace{10pt}
% \end{frame}
% 
% \begin{frame}[noframenumbering]{Partial measure of $\Gradients*$}{Robustness}
	% \centering
	% \vspace{30pt}
% %-------------------------------------------------------
	% \begin{columns}[c]
		% \column{.5\textwidth}
		% \centering
		% \begin{equation*}
			% \SwimmingDirectionOpt* = \left[ \exp \left( \TimeHorizon \, \mathrm{sym} \Gradients* \right) \right]^T \, \Direction.
		% \end{equation*}
		% \begin{equation*}
			% \SwimmingDirectionOpt* = \left[ \exp \left( \TimeHorizon \, \mathrm{skew} \Gradients* \right) \right]^T \, \Direction.
		% \end{equation*}
% 
		% \column{.5\textwidth}
		% \pause
		% \centering
		% %\tikzexternalenable
\begin{tikzpicture}
    \begin{groupplot}[
            group style={
                group size=1 by 1,
            },
            % size
            width=\textwidth,
            height=\textwidth,
            % y
            ylabel={$\left\langle \Performance \right\rangle / \SwimmingVelocity$},
            y label style={yshift=-4pt},
            ymin=0.0,
            ymax=2.5,
            % x
            x label style={yshift=4pt},
            xmin=0.0,
            % layers
            set layers,
            % legend
            legend columns=3,
            legend style={draw=none, fill=none},
            legend cell align=left,
        ]
    % partial information
    \nextgroupplot[
        axis on top,
        % size
        %width=0.32\textwidth,
        extra y ticks={0.5, 1.5, 2.5},
        % x
        xlabel=$\TimeHorizon / \KolmogorovTimeScale$,
        xmax=30,
        % legend
        legend pos = north west,
        legend style={fill opacity=0.5, text opacity=1},
    ]
        %% full
        \addplot[name path=A, draw=none, forget plot] table [
            x index=3, 
            y expr={(\thisrowno{0} - \thisrowno{1}) / (\thisrowno{2} * 0.066)}, %u_\eta = 0.066
            col sep=comma, 
            comment chars=\#,
        ]{data/jhtdb_full/merge_average_velocity_axis_0__agent_full.csv};
        \addplot[name path=B, draw=none, forget plot] table [
            x index=3, 
            y expr={(\thisrowno{0} + \thisrowno{1}) / (\thisrowno{2} * 0.066)}, %u_\eta = 0.066
            col sep=comma, 
            comment chars=\#,
        ]{data/jhtdb_full/merge_average_velocity_axis_0__agent_full.csv};
        \addplot[ColorSurf, opacity=0.4, forget plot] fill between[of=A and B];
        %% asym
        \addplot[name path=A, draw=none, forget plot] table [
            x index=3, 
            y expr={(\thisrowno{0} - \thisrowno{1}) / (\thisrowno{2} * 0.066)}, %u_\eta = 0.066
            col sep=comma, 
            comment chars=\#,
        ]{data/jhtdb_asym/merge_average_velocity_axis_0__agent_asym.csv};
        \addplot[name path=B, draw=none, forget plot] table [
            x index=3, 
            y expr={(\thisrowno{0} + \thisrowno{1}) / (\thisrowno{2} * 0.066)}, %u_\eta = 0.066
            col sep=comma, 
            comment chars=\#,
        ]{data/jhtdb_asym/merge_average_velocity_axis_0__agent_asym.csv};
        \addplot[ColorAsym, opacity=0.4, forget plot] fill between[of=A and B];
        %% sym
        \addplot[name path=A, draw=none, forget plot] table [
            x index=3, 
            y expr={(\thisrowno{0} - \thisrowno{1}) / (\thisrowno{2} * 0.066)}, %u_\eta = 0.066
            col sep=comma, 
            comment chars=\#,
        ]{data/jhtdb_sym/merge_average_velocity_axis_0__agent_sym.csv};
        \addplot[name path=B, draw=none, forget plot] table [
            x index=3, 
            y expr={(\thisrowno{0} + \thisrowno{1}) / (\thisrowno{2} * 0.066)}, %u_\eta = 0.066
            col sep=comma,
            comment chars=\#,
        ]{data/jhtdb_sym/merge_average_velocity_axis_0__agent_sym.csv};
        \addplot[ColorSym, opacity=0.4, forget plot] fill between[of=A and B];
        % full
        \addplot
        [
        color=ColorSurf,
        opacity=1.0,
        line width=1pt,
        only marks,%solid
        mark=pentagon*
        ]
        table[
            x index=3, 
            y expr={\thisrowno{0} / (\thisrowno{2} * 0.066)}, %u_\eta = 0.066
            col sep=comma, 
            comment chars=\#,
        ]{data/jhtdb_full/merge_average_velocity_axis_0__agent_full.csv};
        \addlegendentry{$\Gradients$}
        % asym
        \addplot
        [
        color=ColorAsym,
        opacity=1.0,
        line width=1pt,
        only marks,%solid
        mark=triangle
        ]
        table[
            x index=3, 
            y expr={\thisrowno{0} / (\thisrowno{2} * 0.066)}, %u_\eta = 0.066
            col sep=comma, 
            comment chars=\#,
        ]{data/jhtdb_asym/merge_average_velocity_axis_0__agent_asym.csv};
        \addlegendentry{$\mathrm{skew} \Gradients$}
        % sym
        \addplot
        [
        color=ColorSym,
        opacity=1.0,
        line width=1pt,
        only marks,%solid
        mark=square
        ]
        table[
            x index=3, 
            y expr={\thisrowno{0} / (\thisrowno{2} * 0.066)}, %u_\eta = 0.066
            col sep=comma, 
            comment chars=\#,
        ]{data/jhtdb_sym/merge_average_velocity_axis_0__agent_sym.csv};
        \addlegendentry{$\mathrm{sym} \Gradients$}
    \end{groupplot}
\end{tikzpicture}
%\tikzexternaldisable

	% \end{columns}
% 
	% \vskip0pt plus 1filll
% 
	% \begin{itemize}
		% \item Surfing is robust to partial measure of the gradients.
	% \end{itemize}
	% \vspace{10pt}
% \end{frame}

\end{document}
